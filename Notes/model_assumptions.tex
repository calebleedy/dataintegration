\documentclass[12pt]{article}

\usepackage{amsmath, amssymb, mathrsfs, fancyhdr}
\usepackage{syntonly, lastpage, hyperref, enumitem, graphicx}
\usepackage{biblatex}
\usepackage{booktabs}
\usepackage{float}

\addbibresource{references.bib}

\hypersetup{colorlinks = true, urlcolor = black}

\headheight     15pt
\topmargin      -1.5cm   % read Lamport p.163
\oddsidemargin  -0.04cm  % read Lamport p.163
\evensidemargin -0.04cm  % same as oddsidemargin but for left-hand pages
\textwidth      16.59cm
\textheight     23.94cm
\parskip         7.2pt   % sets spacing between paragraphs
\parindent         0pt   % sets leading space for paragraphs
\pagestyle{empty}        % Uncomment if don't want page numbers
\pagestyle{fancyplain}

\newcommand{\MAP}{{\text{MAP}}}
\newcommand{\argmax}{{\text{argmax}}}
\newcommand{\argmin}{{\text{argmin}}}
\newcommand{\Cov}{{\text{Cov}}}
\newcommand{\Var}{{\text{Var}}}
\newcommand{\logistic}{{\text{logistic}}}

\renewcommand{\arraystretch}{1.2}

\begin{document}

\lhead{Caleb Leedy}
\chead{Non-monotone Missingness}
%\chead{STAT 615 - Advanced Bayesian Methods}
%\rhead{Page \thepage\ of \pageref{LastPage}}
\rhead{\today}

\section*{Introduction}

This document reports the optimality of the proposed non-monotone estimator.
It turns out that the proposed estimator is not optimal. We show a better
estimator and present a simulation study.

\section*{Setup}

\begin{itemize}
  \item Let $U = \{1, \dots N\}$ be a finite population.
  \item Consider the random variables $(X_i, Y_{1i}, Y_{2i}, \pi_i) 
    \stackrel{\text{ind}}{\sim} F$ for some unknown $F$ for all $i \in U$.
  \item We assume that $x_i$ is observed throughout the finite population.
  \item Let $I_i \stackrel{\text{ind}}{\sim} \pi_i$ for $i \in U$ be the 
    first phase sample inclusion indicator.
  \item If $i \in \{1, \dots, N\}$ is in the first phase sample ($I_i = 1$)
    then $\pi_i$ is observed. 
  \item The second phase sampling indicators are $\delta_{00}$, $\delta_{01}$,
    $\delta_{10}$, and $\delta_{11}$. These mutual exclusive variables encode 
    the following: if both $Y_1$ and $Y_2$ are observed then $\delta_{11} = 1$;
    if $Y_1$ is observed and $Y_2$ is missing then $\delta_{10} = 1$; if $Y_1$
    is missing and $Y_2$ is observed then $\delta_{01} = 1$; and if both $Y_1$
    and $Y_2$ are missing then $\delta_{00} = 1$. (For notational purposes, it
    is sometimes convienent to write $\delta_{1} = \delta_{11} + \delta_{10}$ 
    and $\delta_2 = \delta_{11} + \delta_{01}$. A similar notation will be 
    used for $\pi$. We have $\pi_{1+} = \pi_{11} + \pi_{10}$ and 
    $\pi_{2+} = \pi_{11} + \pi_{01}$.)
  \item All of the sampling indicators were drawn via \textit{Poisson}
    sampling. So the sample size of each category $A_{ij}$ is random. However,
    this does ensure independence between observations.
  \item The goal is to estimate $\theta = E[g(X, Y_1, Y_2)]$ in the population.
  \item The proposed estimator is

  \begin{align}\label{eq:prop}
    \hat \theta_{\text{eff}} 
    &= n^{-1} \sum_{i = 1}^n E[g_i \mid X_i] \nonumber \\ 
    &+ n^{-1} \sum_{i = 1}^n \frac{\delta_{1i}}{\pi_{1+}(X_i)}
      (E[g_i \mid X_i, Y_{1i}] - E[g_i \mid X_i])\nonumber  \\ 
    &+ n^{-1} \sum_{i = 1}^n \frac{\delta_{2i}}{\pi_{2+}(X_i)}
      (E[g_i \mid X_i, Y_{2i}] - E[g_i \mid X_i])\nonumber  \\ 
    &+ n^{-1} \sum_{i = 1}^n \frac{\delta_{1i} \delta_{2i}}{\pi_{11}(X_i)}
      (g_i - E[g_i \mid X_i, Y_{1i}] - E[g_i \mid X_i, Y_{2i}] + E[g_i \mid X_i])
  \end{align}
  
  \item The goal is to show that this estimator is optimal within the class
    of estimators:

    \begin{align}\label{eq:propclass}
    \hat \theta_{\text{eff}} 
    &= n^{-1} \sum_{i = 1}^n E[g_i \mid X_i]\nonumber \\ 
    &+ n^{-1} \sum_{i = 1}^n \frac{\delta_{1i}}{\pi_{1+}(X_i)}
      (b(X_i, Y_{1i}) - E[g_i \mid X_i])\nonumber \\ 
    &+ n^{-1} \sum_{i = 1}^n \frac{\delta_{2i}}{\pi_{2+}(X_i)}
      (a(X_i, Y_{2i}) - E[g_i \mid X_i])\nonumber \\ 
    &+ n^{-1} \sum_{i = 1}^n \frac{\delta_{1i} \delta_{2i}}{\pi_{11}(X_i)}
      (g_i - b_i - a_i + E[g_i \mid X_i]).
    \end{align}

\end{itemize}

\section*{Results}

The first part recaps what Dr. Fuller and I discussed previously.

\begin{itemize}
  \item Let $g = E[Y_2]$, and suppose that we know the distribution of $X$ and 
    the covariance structure of $[X, Y_1, Y_2, \pi]$. Then instead of modeling
    the relationship between $X$ and $Y_k$, we can use difference estimators:
    $Z_1 \equiv Y_1 - b_1 \tilde X$ and $Z_2 = \equiv Y_2 - b_2 \tilde X$, where
    $\tilde X = X - E[X]$.
    To minimize the variance of $Z_k$ we can choose the optimal value of $b_k$,
     which is 

     \[b_k = \frac{\Cov(Y_k, X)}{\Var(X)}.\]

    Since these covariance values are known, $b_k$ is known. This means that we
    now have the following table:

    \begin{figure}[!ht]
      \centering
      \begin{tabular}{lrr}
        \toprule
         & $Z_1$ & $Z_2$ \\
         \midrule
        $A_{11}$ & \checkmark & \checkmark \\
        $A_{10}$ & \checkmark & \\
        $A_{01}$ & & \checkmark \\
        $A_{00}$ & & \\
        \bottomrule
      \end{tabular}
    \end{figure}

    Because we assumed that the distribution of $X$ is known, the section
    $A_{00}$ contains no additional information about $Y_1$ or $Y_2$. Let
    $\mu_k = E[Y_k]$.

  \item We now consider a normal model:

    \[ 
      \begin{pmatrix}
        x_i \\ e_{1i} \\ e_{2i}
      \end{pmatrix} \stackrel{ind}{\sim}
      N \left(
      \begin{bmatrix}
        0 \\ 0 \\ 0
      \end{bmatrix},
      \begin{bmatrix}
        1 & 0 & 0 \\
        0 & \sigma_{11} & \sigma_{12} \\ 
        0 & \sigma_{12} & \sigma_{22}
      \end{bmatrix},
      \right)
    \]

    and define $y_{1i} = \theta_1 + x_i + e_{1i}$ and $y_{2i} = \theta_2 + x_i
    + e_{2i}$. Then $b_1 = b_2 = 1$. We define $\bar z_k^{(ij)}$ as the mean of
    $y_k$ in segment $A_{ij}$. This means that we have means $\bar z_1^{(11)}$,
    $\bar z_2^{(11)}$, $\bar z_1^{(10)}$, and $\bar z_2^{(01)}$. Let $W = [\bar
    z_1^{(11)}, \bar z_2^{(11)}, \bar z_1^{(10)}, \bar z_2^{(01)}]'$, then for
    $n_{ij} = |A_{ij}|$, we have

    \[Z - M \mu \sim N(\vec 0, V)\]

    where 

    \[M = 
      \begin{bmatrix}
        1 & 0 \\
        0 & 1 \\
        1 & 0 \\
        0 & 1 \\
      \end{bmatrix}
      \text{ and }
      V = 
      \begin{bmatrix}
        \frac{\sigma_{11}}{n_{11}} & \frac{\sigma_{12}}{n_{11}} & 0 & 0 \\
        \frac{\sigma_{12}}{n_{11}} & \frac{\sigma_{22}}{n_{11}} & 0 & 0 \\
        0 & 0 & \frac{\sigma_{11}}{n_{10}} & 0 \\
        0 & 0 & 0 & \frac{\sigma_{22}}{n_{01}} \\
      \end{bmatrix}.
    \]

    Thus, the BLUE for $\mu = [\mu_1, \mu_2]'$ is 

    \begin{equation}\label{eq:wls}
    \hat \mu = (M' V^{-1} M)^{-1} M' V^{-1} W.
    \end{equation}

  \item Since this is the BLUE, we would expect it to be at least as good as
    the proposed estimator. And if the proposed estimator is optimal it should
    be equivalent to the BLUE in the case that $X$, $Y_1$ and $Y_2$ are normal.
    So I ran a simulation study to test this. For $i = \{1, \dots, n = 1000\}$,

    \[
      \begin{pmatrix}
        x_i \\ e_{1i} \\ e_{2i}
      \end{pmatrix} \stackrel{ind}{\sim}
      N \left(
      \begin{bmatrix}
        0 \\ 0 \\ 0
      \end{bmatrix},
      \begin{bmatrix}
        1 & 0 & 0 \\
        0 & 1 & \sigma_{12} \\ 
        0 & \sigma_{12} & 1 
      \end{bmatrix},
      \right) \text{ and }
      y_{1i} = \theta_1 + x_i + e_{1i}, y_{2i} = \theta_2 + x_i + e_{2i}.
    \]

    Each observation $i$ was then assigned a segment $A_{11}$, $A_{10}$,
    $A_{01}$ or $A_{00}$ independently with each draw having probability
    $p_{11} = 0.4$, $p_{10} = 0.2$, $p_{01}= 0.2$ and $p_{00} = 0.2$
    respectively. This means that $\pi_{11} = 0.4$ and 
    $\pi_{1+} = 0.6 = \pi_{2+}$. We let $g = E[Y_2]$ and we test the following
    estimators:

    \begin{itemize}
      \item Oracle: This computes the average value of $Y_2$ in all 
        observations (even when $Y_2$ is not supposed to be observed).
        \[\hat \theta = n^{-1} \sum_{i = 1}^n y_{2i}.\]

      \item OracleX: This computes the average value of $Y_2 - X$ in 
        all observations (even when $Y_2$ is not supposed to be observed.)
        \[\hat \theta = n^{-1} \sum_{i = 1}^n (y_{2i} - x_i).\]

      \item CC: This computes the average value of $Y_2$ in all observations 
        in which $Y_2$ is observed.
        \[\hat \theta = \frac{\sum_{i = 1}^n \delta_{2i} y_{2i}}
          {\sum_{i = 1}^n \delta_{2i}}.\]

      \item Proposed: This is the proposed estimator from Equation~\ref{eq:prop}.
      \item WLS: This is the weighted linear estimator from Equation~\ref{eq:wls}.
        Note, since $g = E[Y_2]$ this only contains the second element from
        Equation~\ref{eq:wls}.
    \end{itemize}

    The results are shown in the table below. This simulation was run with the 
    number of observations $n = 1000$ and the Monte Carlo sample size of 
    $B = 3000$.

    \begin{table}[ht!]
      \label{tab:res}
      \caption{This table shows the estimators of $\theta = E[Y_2]$. The true 
        value of $\theta$ is $5$ and the true value of $\sigma_{12} = 0.5$.
        The bias column shows the average bias of the 
      estimator and the actual value of $\theta = 5$ across the $B = 3000$ 
      simulations. The SD column shows the average standard deviation across the 
      $B = 3000$ simulations for each algorithm. The Tstat column displays the 
      t-statistic of a t-test comparing the estimator to the actual value. 
      The value of this column is computed via $\frac{\bar{\hat{\theta}} - 
      \theta}{\sqrt{\Var{\hat \theta}/B}}$. The Pval column displays the 
      p-value of the t-statistic.}
      \centering
      \begin{tabular}[t]{lrrrr}
        \toprule
        Algorithm & Bias & SD & Tstat & Pval\\
        \midrule
        Oracle & 0.001 & 0.044 & 1.546 & 0.061\\
        OracleX & 0.000 & 0.032 & 0.479 & 0.316\\
        CC & 0.002 & 0.058 & 1.549 & 0.061\\
        WLS & 0.001 & 0.040 & 1.207 & 0.114\\
        Prop & 0.002 & 0.051 & 1.918 & 0.028\\
        \bottomrule
      \end{tabular}
    \end{table}

\end{itemize}

\section*{Computing the Variance by Hand}

Table~\ref{tab:res} demonstrates that the WLS estimator outperforms the 
Oracle estiamator. I was initially confused about why this is the case, but
realized that I could approximate the standard error of both estimators
reasonably well. It turns out that I was able to get reasonable estimates 
for the variance of all of the proposed estimators includeing $\hat
\theta_{prop}$. In this part of the report, I detail my computations and show
how the simulated results are accurate.

\begin{align}\label{eq:varor}
  \Var(\hat \theta_{Oracle}) 
  &= \Var\left(n^{-1} \sum_{i = 1}^n y_{2i}\right) \nonumber \\
  &= n^{-1} \Var(y_{2}) \nonumber \\
  &= n^{-1} (1 + \sigma_{22}) & \text{because $\Var(x) = 1$.}
\end{align}

\begin{align}\label{eq:varorx}
  \Var(\hat \theta_{OracleX}) 
  &= \Var\left(n^{-1} \sum_{i = 1}^n y_{2i} - x_i\right) \nonumber \\
  &= n^{-1} \Var(e_{2}) \nonumber \\
  &= n^{-1} \sigma_{22}.
\end{align}

Using a Taylor expansion,

\begin{align}\label{eq:varcc}
  \Var(\hat \theta_{cc})
  &= \Var\left(\theta_2 + \pi_{2+}^{-1} \left(n^{-1}\sum_{i = 1}^n \delta_{2i} y_{2i}
  - \pi_{2+}\theta_2\right) - \frac{\theta_2 \pi_{2+}}{\pi_{2+}^2} 
  \left(n^{-1} \sum_{i = 1}^n \delta_{2i} - \pi_{2+}\right) + o_p(1)\right)\nonumber  \\
  &\quad (\text{I can add more steps later if necessary.})\\
  &= \frac{1 + \sigma_{22}}{n \pi_{2+}} + o_p(1).
\end{align}

To compute the variance of $\hat \theta_{WLS}$, we have,

\begin{align}\label{eq:varwls}
  \Var(\hat \theta_{WLS})
  &= \Var((M'V^{-1}V)^{-1}M'V^{-1}Z) \nonumber \\ 
  &= E[\Var((M'V^{-1}V)^{-1}M'V^{-1}Z \mid n_1)]\nonumber  \\
  &= E[(M'V^{-1}M)]\\
\end{align}

By Slutky's Theorem, $\Var(\hat \theta_{WLS}) \stackrel{P}{\to} (M'E[V]^{-1}M)^{-1}$.

Finally, we compute the variance for the proposed estimator. First, we
estimate the variance for a similar estimator $\tilde \theta_{prop}$. This
estimator is the same as the proposed estimator except that we assume that 
the estimates from $E[Y_{2i} \mid X, Y_1]$ and $E[Y_{2i} \mid X]$ are 
population coefficients and independent of the rest of the sample. This
could occur if there is a separate sample in which we can estimate these 
expectations. Since the population coefficients of the regression 
$Y = \beta_0 + \beta_1 X$ are

\[\beta_0 = E[Y] - E[X] \Var(X)^{-1} Cov(X'Y) \text{ and } 
\beta_1 = \Var(X)^{-1}\Cov(X, Y),\]

we have that,

\begin{align*}
  E[Y_2 \mid X] &= E[Y_2] - \beta_1 E[X] + \beta_1 x = \theta_2 - x
\end{align*}

and

\begin{align*}
  E[Y_2 \mid X, Y_1] 
  &= \beta_0 + [X, Y_1][\beta_1 \beta_2]' \\ 
  &= E[Y_2] - E[[X, Y_1]] \Var([X, Y_1])^{-1}\Cov([X, Y_1], Y_2) + 
  [X, Y_1]\Var([X, Y_1])^{-1} \Cov([X, Y_1], Y_2) \\ 
  &= \theta_2 + [X, Y_1] \sigma_{11}^{-1} 
  \begin{bmatrix}
    1 + \sigma_{11} & -1 \\ -1 & 1
  \end{bmatrix}
  \begin{bmatrix}
    1 & 1 + \sigma_{12}
  \end{bmatrix} \\
  &= \theta_2 + x \left(\frac{\sigma_{11} - \sigma_{12}}{\sigma_{11}}\right) + 
  y_1 \left(\frac{\sigma_{12}}{\sigma_{11}}\right)\\
  &= \theta_2 + x(1 - \sigma_{12}) + y_1(\sigma_{12}). 
\end{align*}

This means that 

\[E[Y_2 \mid X, Y_1] - E[Y_2 \mid X] = \sigma_{12}(y_1 - x) = \sigma_{12} e_1.\]

Hence, whereas for $\theta = E[Y_2]$,

\[\hat \theta_{prop} = n^{-1} \sum_{i = 1}^n \left\{E[Y_2 \mid X] +
    \left(\frac{\delta_{1i}}{\pi_{1+}} - \frac{\delta_{11i}}{\pi_{11}}\right)
    (E[Y_2 \mid X, Y_1] - E[Y_2 \mid X]) + 
    \frac{\delta_{2i}}{\pi_{2+}}(y_{2i} - E[Y_2 \mid X])
\right\}\]

for the similar estimator we have

\[\tilde \theta_{prop} = n^{-1} \sum_{i = 1}^n \left( \theta_2 + x_i + 
    \left(\frac{\delta_{1i}}{\pi_{1+}} - \frac{\delta_{11i}}{\pi_{11}}\right)
    \sigma_{12} e_{1i} + \frac{\delta_{2i}}{\pi_{2+}} (e_{2i}) \right)
\]

This means that 

\begin{align*}
  \Var(\tilde \theta_{prop})
  &= n^{-1}\Var\left(x + 
    \left(\frac{\delta_{1}}{\pi_{1+}} - \frac{\delta_{11}}{\pi_{11}}\right)
    \sigma_{12} e_{1} + \frac{\delta_{2}}{\pi_{2+}} (e_{2}) \right)\\
  &= n^{-1} \left(\Var(x) + \Var\left(
    \left(\frac{\delta_{1}}{\pi_{1+}} - \frac{\delta_{11}}{\pi_{11}}\right)
    \sigma_{12} e_{1}
    \right) + \Var\left( \frac{\delta_{2}}{\pi_{2+}} (e_{2}) \right)\right. \\ 
  &\qquad+ \left.2\Cov\left(
    \left(\frac{\delta_{1}}{\pi_{1+}} - \frac{\delta_{11}}{\pi_{11}}\right)
    \sigma_{12} e_{1},
     \frac{\delta_{2}}{\pi_{2+}} (e_{2}) 
    \right) \right)\\
  &\equiv A + B + C + D.
\end{align*}

\begin{align*}
  A &= \Var(x) = 1.
\end{align*}

\begin{align*}
  B &= \sigma_{12}^2 \sigma_{11} \Var\left(
    \left(\frac{\delta_{1}}{\pi_{1+}} - \frac{\delta_{11}}{\pi_{11}}\right)
  \right)\\
    &=  \sigma_{12}^2 \sigma_{11} \left(\Var(\delta_1 / \pi_{1+}) + 
      \Var(\delta_{11}/\pi_{11}) - 
      2\Cov(\delta_1 / \pi_{1+}, \delta_{11} / \pi_{11}) \right)\\
    &= \sigma_{12}^2 \sigma_{11}(\pi_{1+}^{-1} + \pi_{11}^{-1} - 2 - 2(\pi_{11}
    - \pi_{1+} \pi_{11}) / (\pi_{1+}\pi_{11}))\\
    &= \sigma_{12}^2 \sigma_{11}(\pi_{11}^{-1} - \pi_{1+}^{-1}).
\end{align*}

\begin{align*}
  C &= \Var(\delta_2 e_2 / \pi_{2+})\\
    &= \pi_{2+}^{-2}(\Var(\delta_2)\Var(e_2) + \Var(\delta_2)E[e_2]^2 +
    \Var(e_2)E[\delta_2]^2)\\
    &= \pi_{2+}^{-2} (\pi_{2+}(1 - \pi_{2+}) \sigma_{22} + \sigma_{22}\pi_{2+}^2)\\
    &= \sigma_{22} / \pi_{2+}.
\end{align*}

\begin{align*}
  D &= 2\Cov\left(
    \left(\frac{\delta_{1}}{\pi_{1+}} - \frac{\delta_{11}}{\pi_{11}}\right)
    \sigma_{12} e_{1}, \frac{\delta_{2}}{\pi_{2+}} (e_{2}) \right)\\
    &= 2E\left[
    \left(\frac{\delta_{1}}{\pi_{1+}} - \frac{\delta_{11}}{\pi_{11}}\right)
    \sigma_{12} e_{1} \frac{\delta_{2}}{\pi_{2+}} (e_{2}) 
    \right] \\ 
    &= 2E\left[
    \left(\frac{\delta_{1}}{\pi_{1+}} - \frac{\delta_{11}}{\pi_{11}}\right)
    \frac{\delta_{2}}{\pi_{2+}}
    \right] \sigma_{12}^2\\
    &= 2E\left[\frac{\delta_{11}}{\pi_{1+}\pi_{2+}} - \frac{\delta_{11}}{\pi_{2+}}
    \right] \sigma_{12}^2\\
    &= 2 \frac{\pi_{11}}{\pi_{2+}} (\pi_{1+}^{-1} - 1) \sigma_{12}^2.
\end{align*}

Hence,

\begin{equation}\label{eq:varprop}
\Var(\tilde \theta_{prop}) = n^{-1}(1 +
  \sigma_{12}^2\sigma_{11}(\pi_{11}^{-1} - \pi_{1+}^{-1}) + \sigma_{22}
  \pi_{2+}^{-1} 2\sigma_{12}^2 (\pi_{1+}^{-1} - 1) \pi_{11}/ \pi_{2+}).
\end{equation}


Since $\hat \beta - \beta = O(n)$ from Theorem 2.2.1 of \cite{fuller2009sampling}, 
$\Var(\hat \theta_{prop}) \to \Var(\tilde \theta_{prop})$. (I think that 
I also need to show that the dependency between the estimated 
coefficients and the observed data is bounded and that this bound goes 
to zero as the sample size increases.)

These estimated variances are almost exactly in line with the estimated
standard deviations from the simulation study. With values of $n = 1000$,
$\sigma_{11} = 1$, $\sigma_{22} = 1$, $\sigma_{12} = 0.5$, $\theta_2 = 5$,
$\pi_{11} = 0.4$, $\pi_{10} = 0.2$, $\pi_{01} = 0.2$, and $\pi_{00} = 0.2$
the standard deviations from the previous calculations and from the Monte Carlo
simulation are listed in Table~2:
%Table~\ref{tab:calmc}:

\begin{table}[!ht]
  \centering
  \label{tab:calmc}
  \caption{This table compares the calculated and estimated standard 
  errors for each of the algorithms listed. The calculated values are 
  computed as the square root of the variance derived in the previous 
  section. We use Equations~\ref{eq:varor}, \ref{eq:varorx}, \ref{eq:varcc}
  \ref{eq:varwls}, and \ref{eq:varprop} to calculate the variance of 
  Oracle, OracleX, CC, WLS, and Prop respectively and then take the square
  root of the answer. For the estimated standard error, we take the 
  standard deviation of the $B$ estimators for each algorithm.}
  \begin{tabular}{lrr}
    \toprule
    Algorithm & Calculated Standard Error & Estimated Standard Error \\
    \midrule
    Oracle  & 0.045 & 0.044 \\
    OracleX & 0.032 & 0.032 \\
    CC      & 0.058 & 0.058 \\
    WLS     & 0.040 & 0.040 \\
    Prop    & 0.056 & 0.051 \\
    \bottomrule
  \end{tabular}
\end{table}

\newpage
  \quad
\newpage

\section*{Non-linear Estimation}

\begin{itemize}
  \item While we have previously shown that in the case of the linear 
    estimator, that the weighted least squares (WLS) estimator is superior 
    to the proposed estimator. However, we also want to see if this holds 
    for a non-linear function. Consider the estimating equation $g(X, Y_1, Y_2)
     = Y_1^2 Y_2$. Using the same setup as the previous simulation, we have the
     same definitions of $X$, $Y$, $n$, $\pi$, and $B$.

     For the definition of each estimator we have the following algorithms:

     \begin{itemize}
      \item Oracle: This computes the average value of $Y_1^2Y_2$ in all 
        observations (even when $Y_1$ or $Y_2$ is not supposed to be observed).
        \[\hat \theta = n^{-1} \sum_{i = 1}^n y_{1i}^2y_{2i}.\]

      \item OracleX: This computes the average value of $Y_1^2 (Y_2 - X)$ in 
        all observations (even when $Y_1$ or $Y_2$ is not supposed to be observed.)
        \[\hat \theta = n^{-1} \sum_{i = 1}^n y_{1i}^2(y_{2i} - x_i).\]

      \item CC: This computes the average value of $Y_1^2Y_2$ in all
        observations in which $Y_1$ and $Y_2$ are observed.
        \[\hat \theta = \frac{\sum_{i = 1}^n \delta_{1i}\delta_{2i} y_{1i}^2y_{2i}}
        {\sum_{i = 1}^n \delta_{1i}\delta_{2i}}.\]

      \item Proposed: This is the proposed estimator from Equation~\ref{eq:prop}.
        Note that since,

        \begin{align*}
          Y_1^2 Y_2 &= (x + e_1)^2 (\theta_2 + x + e_2) \\ 
                    &= x^3 + x^2 (\theta_2 + 2 e_1 + e_2) 
                    + x(2\theta_2 e_1 + 2 e_1 e_2 + e_1^2) + e_1^2 \theta_2 
                    + e_1^2 e_2
        \end{align*}

        for the conditional expectations, we have
        \begin{align*}
          E[Y_1^2 Y_2 \mid X] &= x^3 + x^2 \theta_2 + x(2\sigma_{12} + 1) + \theta_2\\
          E[Y_1^2 Y_2 \mid X, Y_1] &= Y_1^2 (x + \theta_2) \text{ and }\\
          E[Y_1^2 Y_2 \mid X, Y_2] &= (x^2 + 1) Y_2.
        \end{align*}

      \item WLS: To derive the WLS estimator we can view the sample as 
        consisting of four independent parts: $A_{11}, A_{10}, A_{01}$, and 
        $A_{00}$, where $A_{ij}$ indicates the missingness levels of $Y_1$ (if 
        i = 0) or $Y_2$ (if $j = 0$). We know that the correctly specified
        regression model for $g$ in each of these segments is the following:

        \begin{align*}
          A_{11}&: Y_1^2 Y_2 \\
          A_{10}&: Y_1^2  (x + \theta_2) \\
          A_{01}&: (x^2 + 1) Y_2\\
          A_{00}&: x^3 + x^2 \theta_2 + x(2\sigma_{12} + 1) + \theta_2
        \end{align*}

        The estimator is then to estimate $\theta_2$ using Equation~\ref{eq:wls}.
        Then, we can plug in the estimated value of $\theta_2$ into the previous 
        equations to get estimates of $g$. The overall estimate is the estimated 
        variance weighted average of the mean estimate from each segment.

      \item WLSTT: Since the previous estimator is biased, we use the same
        estimation procedure with the true value of $\theta_2$ to the estimator 
        $\hat \theta_{WLSTT}$.
     \end{itemize}

     \begin{table}[!ht]
       \caption{True g is 10. $\Cov(e_1, e_2) = 0.5$}
       \centering
       \begin{tabular}[t]{lrrrr}
         \toprule
         Algorithm & Bias & SD & Tstat & Pval\\
         \midrule
         Oracle & 0.007 & 0.529 & 0.741 & 0.229\\
         Oraclex & 0.004 & 0.475 & 0.510 & 0.305\\
         CC & 0.023 & 0.824 & 1.518 & 0.065\\
         WLS & -0.131 & 0.387 & -18.504 & 0.000\\
         WLSTT & -0.132 & 0.380 & -19.007 & 0.000\\
         Prop & 0.011 & 0.625 & 0.952 & 0.171\\
         \bottomrule
       \end{tabular}
     \end{table}

     In this scenario, the WLS estimator is biased. Other than the oracle 
     estimators, the proposed estimator is the best unbiased estimator.

\end{itemize}

\section*{Next Steps}

\begin{itemize}
  \item I am slightly unclear about how to proceed. The proposed estimator 
    seemed to work quite well for the non-linear experiment. However, it was 
    easily out-performed by the WLS estimator in the linear case. The WLS 
    estimator is a relatively intuitive estimator that is optimal in that 
    situation. However, it seems to have worse performance in the non-linear 
    case.
\end{itemize}

\printbibliography

\end{document}
