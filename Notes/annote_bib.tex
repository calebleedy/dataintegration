\documentclass[12pt]{article}

\usepackage{amsmath, amssymb, mathrsfs, fancyhdr}
\usepackage{syntonly, lastpage, hyperref, enumitem, graphicx}
\usepackage{biblatex}
\usepackage{booktabs}
\usepackage{float}

\addbibresource{references.bib}

\hypersetup{colorlinks = true, urlcolor = black}

\headheight     15pt
\topmargin      -1.5cm   % read Lamport p.163
\oddsidemargin  -0.04cm  % read Lamport p.163
\evensidemargin -0.04cm  % same as oddsidemargin but for left-hand pages
\textwidth      16.59cm
\textheight     23.94cm
\parskip         7.2pt   % sets spacing between paragraphs
\parindent         0pt   % sets leading space for paragraphs
\pagestyle{empty}        % Uncomment if don't want page numbers
\pagestyle{fancyplain}

\newcommand{\MAP}{{\text{MAP}}}
\newcommand{\argmax}{{\text{argmax}}}
\newcommand{\argmin}{{\text{argmin}}}
\newcommand{\Cov}{{\text{Cov}}}
\newcommand{\Var}{{\text{Var}}}
\newcommand{\logistic}{{\text{logistic}}}

\begin{document}

%\lhead{Caleb Leedy}
\chead{Nonmonotone Missingness}
%\chead{STAT 615 - Advanced Bayesian Methods}
%\rhead{Page \thepage\ of \pageref{LastPage}}
\rhead{April 28, 2023}

\section*{Annotated Bibliography}

\subsection*{Merkouris et al. (2023)}

\begin{description}
  \item[Title:] Combining National Surveys with Composite Calibration to Improve
    the Precision of Estimates from the United Kingdom's Living Costs and Food
    Supply

  \item[Authors:] Takis Merkouris, Paul A. Smith, Andy Fallows

  \item[Journal (Year):] JSSAM (2023)

  \item[Summary:] In some sense this is the application of ideas from Merkouris
    (2004) and Merkouris (2010) to combine information from the Living Costs and
    Food Survey (LCF) and the Labour Force Survey (LFS) in the United Kingdom.
    The overall approach is to use a linear combination of regression estimators
    from each survey:

    \[\hat Z^{CR} = B \hat Z_1^R + (I - B) Z_2^R\]

    where $B$ is a matrix, $Z_1^R$ is the regression estimator from the first
    survey and $Z_2^R$ is the regression estimator from the second survey. Like
    Merkouris (2004), they choose the optimal $B$ that minimizes the variance of
    the estimator while still having the estimation of $\hat Z$ be the same for
    each survey. To incorporate results from Merkouris (2010), they also control
    totals at the regional level (a small-area estimation problem). (More
    details regarding the derivation of the optimal $B$ is left to the summary
    of Merkouris (2004) and Merkouris (2010).) 

    Methodologically, the biggest innovation is the jackknife variance
    estimation that they use. However, even this seems relatively
    straightforward. I think that the main contribution of this paper is to show
    a dramatic reduction in the variance of regression estimates due to data
    integration.

  \item[Limitations:]\;
    \begin{itemize}
      \item They only integrate two surveys.
      \item They do not consider informative sampling. % FIXME
    \end{itemize}
  \item[Extensions:]\;
    \begin{itemize}
      \item I think that it would be interesting if the supports of several
        control variables were not the same. This would be evidence of selection
        bias and we would need to account for it.
    \end{itemize}
\end{description}

\subsection*{Merkouris (2004)}

\begin{description}
  \item[Title:] Combining Independent Regression Estimators from Multiple
    Surveys 

  \item[Author:] Takis Merkouris

  \item[Journal (Year):] JASA (2004)
  \item[Summary:] This paper develops the methodology to combine regression
    estimators from multiple surveys. Given two surveys the paper suggests
    combining the regression estimators as,

    \[\hat Z^{CR} = B\hat Z_1^R + (I - B) \hat Z_2^R\]

    where $B = \phi Z_2' L_2 Z_2 ((1 - \phi) Z_1'L_1 Z_1 + \phi Z_2'L_2
    Z_2)^{-1}$, $\phi = \frac{n_1 / d_1}{n_1 / d_1 + n_2 / d_2}$, $n_i$ is the
    sample size from survey $i$, $d_i$ is the design effects from survey $i$ of
    $z$, $Z_i$ is a matrix of common columns between the two surveys, 
    $L_i = I - X_i (X_i' X_i)^{-1} X_i'$, and $X_i$ are additional covariates
    that we have information to calibrate. In this way, the combined regression
    estimator can have minimal variance while having the estimates of $Z$ from
    each survey match. The weights also adjust for sample size and the strength
    of the correlation between $Z$ and $X$. This is the main method used in
    Merkouris (2023) except that Merkouris (2023) also adjusts for regional
    totals as developed in Merkouris (2010).

\end{description}

\subsection*{Merkouris (2010)}

\begin{description}
  \item[Title:] Combining information from multiple surveys by using regression
    for efficient small domain estimation

  \item[Author:] Takis Merkouris

  \item[Journal (Year):] JRSSB (2010)
  \item[Summary:] The bulk of this paper compares three estimators: the
    population control $\hat Y_d^R \equiv \hat Y_d + \hat \beta_d' (t_x - \hat
    X)$ where $\hat \beta = (X_s' \Lambda_s X_s)^{-1} X_s'\Lambda_s Y_{s_d}$,
    the domain control $\check Y_d^R \equiv \hat Y_d + \check \beta_d' (t_{x_d}
    - \hat X_d)$ where $\check \beta_d = (X_{s_d}' \Lambda_{s_d} X_{s_d})^{-1}
    X_{s_d}' \Lambda_{s_d} Y_{s_d}$, and the domain size control $\tilde Y_d^R
    \equiv \hat Y_d^R + \tilde \beta_d' (N_d - \hat N_d^R)$ where $\tilde
    \beta_d = (1_{s_d}' L_s 1_{s_d})^{-1} 1_{s_d}' L_s Y_{s_d}$. Notice that the
    last estimator is based off of the population control $\hat Y_d^R$ instead
    of the HT-estimator $\hat Y_d$. Since this model incorporates both the
    population variables and the domain indicators, the corresponding regression
    estimator can be thought of as an application of the augmented regression in 
    Merkouris (2004).

\end{description}

% \begin{description}
%   \item[Title:]
%   \item[Author:]
%   \item[Journal (Year):]
%   \item[Summary:]
%   \item[Limitations:]\;
%     \begin{itemize}
%       \item
%     \end{itemize}
%   \item[Extensions:]\;
%     \begin{itemize}
%       \item
%     \end{itemize}
% \end{description}

\end{document}
