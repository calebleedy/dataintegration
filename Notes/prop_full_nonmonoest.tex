\section*{Proposal: Full Nonmonotone Estimator}

\begin{itemize}

  \item When constructing
    estimates of the response model, we estimate $\pi_{11}$ using
    $\pi_{A_2 = 1|A_1 = 1}$ and we never include information about $\pi_{A_1 =
    1|A_2 = 1}$. I would like to create a full estimator where we are able to
    use this information.

  \item The current estimator is the following (See Slide 24 of Non-monotone
    Presentation):

    \begin{align*}
      \hat \theta_{eff} &= n^{-1} \sum_{i = 1}^n E[g_i \mid X_i] \\
      &+ n^{-1} \sum_{i = 1}^n \frac{R_{1i}}{\pi_{1+}(X_i)} (b_2(X_i, Y_{1i}) - 
      E[g_i \mid X_i]) \\
      &+ n^{-1} \sum_{i = 1}^n \frac{R_{2i}}{\pi_{2+}(X_i)} (a_2(X_i, Y_{2i}) -
      E[g_i \mid X_i]) \\
      &+ n^{-1} \sum_{i = 1}^n \frac{R_{1i} R_{2i}}{\pi_{11}(X_1, Y_{1i})}(g_i 
      - b_2(X_i, Y_{1i}) - a_2(X_i, Y_{2i}) + E[g_i \mid X_i])
    \end{align*}

  \item The problem with this is that we assume $\pi_{11}(X, Y_{1}) =
    \pi_{2|1} (X, Y_{1}) \pi_{1+}(X)$ when we could have $\pi_{11}(X, Y_2) =
    \pi_{1|2} (X, Y_2) \pi_{2+}(X)$. In other words, the previous result
    implicitly assumes that $\pi_{11}$ is a function of $X$ and $Y_1$ when it
    could be a function of $X$ and $Y_2$ as well. 

  \item I think that we could use the
    following (small modification) instead,

    \[\pi_{11}(X, Y_1, Y_2) = \alpha \pi_{1|2} \pi_{2+} + (1 - \alpha) \pi_{2|1}
    \pi_{1+}\]

    for some $\alpha \in [0, 1]$.
    
  \item The reasoning behind this is two-fold. First, it allows us to think of
    the nonmonotone missingness case as a linear combination of two monotone
    cases, which I think is useful. Second, we can make explicit our choice of
    $\alpha$, which in the existing estimator is simply $\alpha = 0$.

  \item The problem with adding another parameter $\alpha$ is that it makes the
    overall model unidentifiable. We cannot estimate $\alpha$ and the
    conditional and marginal distributions that we have without additional
    assumptions. So for now, I think that we should just assume that $\alpha$ is
    known and then we apply this method to a data integration problem we can
    review what $\alpha$ makes sense or figure out an additional assumption that
    can help us estimate $\alpha$ (for example if there is a variable correlated
    with $1|2$ versus $2|1$).

  \item This is not just a pure Bayes' rule.
    Formally, let $L = (X, Y_1, Y_2)$. By Bayes' rule we have

    \begin{align*}
      \Pr(R_1 &= 1 \mid R_2 = 1, L) \Pr(R_2 = 1 \mid L) \\
      &= \Pr(R_1 = 1, R_2 = 1 \mid L) \\
      & = \Pr(R_2 = 1 \mid R_1 = 1, L) \Pr(R_1 = 1 \mid L).
    \end{align*}

    Yet to estimate this model, we need to assume something like the following:
    \begin{align*}
      \Pr(R_1 &= 1 \mid R_2 = 1, X, Y_2) \Pr(R_2 = 1 \mid X, Y_2)\\
      &= \Pr(R_1 = 1, R_2 = 1 \mid L) \\
      &= \Pr(R_2 = 1 \mid R_1 = 1, X, Y_1) \Pr(R_1 = 1 \mid X, Y_1)
    \end{align*}

    in which case the two sides are clearly different.

    \item Unfortunately, early simulation results are not promising because we
      cannot distinguish points that should have $\pi_{11}$ conditional on $Y_1$
      and points that should be conditional on $Y_2$.

      \begin{table}[h!]

\caption{True Value is 0. Cor(Y1, Y2) = 0}
\centering
\begin{tabular}[t]{lrrrr}
\toprule
algorithm & bias & sd & tstat & pval\\
\midrule
oracle & 0.001 & 0.032 & 1.269 & 0.102\\
ipworacle & 0.000 & 0.078 & -0.141 & 0.444\\
prop.or & 0.002 & 0.038 & 2.201 & 0.014\\
prop.0 & 0.004 & 0.037 & 5.037 & 0.000\\
prop.half & 0.008 & 0.037 & 9.056 & 0.000\\
\addlinespace
prop.1 & 0.010 & 0.038 & 12.058 & 0.000\\
\bottomrule
\end{tabular}
\end{table}


    \item Dr. Kim, please let me know what you think about this idea and if it
      is worth pursuing more.

\end{itemize}
