% Options for packages loaded elsewhere
\PassOptionsToPackage{unicode}{hyperref}
\PassOptionsToPackage{hyphens}{url}
\PassOptionsToPackage{dvipsnames,svgnames,x11names}{xcolor}
\documentclass[
  12pt]{simple-article}% Use upquote if available, for straight quotes in verbatim environments
\IfFileExists{upquote.sty}{\usepackage{upquote}}{}
\IfFileExists{flushend.sty}{\usepackage{flushend}}{}

\makeatletter
\@ifundefined{KOMAClassName}{% if non-KOMA class
  \IfFileExists{parskip.sty}{%
    \usepackage{parskip}
  }{% else
    \setlength{\parindent}{0pt}
    \setlength{\parskip}{6pt plus 2pt minus 1pt}}
}{% if KOMA class
  \KOMAoptions{parskip=half}}
\makeatother


\setlength{\emergencystretch}{3em} % prevent overfull lines
\setcounter{secnumdepth}{5}
% Make \paragraph and \subparagraph free-standing
\ifx\paragraph\undefined\else
  \let\oldparagraph\paragraph
  \renewcommand{\paragraph}[1]{\oldparagraph{#1}\mbox{}}
\fi
\ifx\subparagraph\undefined\else
  \let\oldsubparagraph\subparagraph
  \renewcommand{\subparagraph}[1]{\oldsubparagraph{#1}\mbox{}}
\fi


\providecommand{\tightlist}{%
  \setlength{\itemsep}{0pt}\setlength{\parskip}{0pt}}\usepackage{longtable,booktabs,array}
\usepackage{calc} % for calculating minipage widths
% Correct order of tables after \paragraph or \subparagraph
\usepackage{etoolbox}
\makeatletter
\patchcmd\longtable{\par}{\if@noskipsec\mbox{}\fi\par}{}{}
\makeatother
% Allow footnotes in longtable head/foot
\IfFileExists{footnotehyper.sty}{\usepackage{footnotehyper}}{\usepackage{footnote}}
\makesavenoteenv{longtable}
\usepackage{graphicx}
\makeatletter
\def\maxwidth{\ifdim\Gin@nat@width>\linewidth\linewidth\else\Gin@nat@width\fi}
\def\maxheight{\ifdim\Gin@nat@height>\textheight\textheight\else\Gin@nat@height\fi}
\makeatother
% Scale images if necessary, so that they will not overflow the page
% margins by default, and it is still possible to overwrite the defaults
% using explicit options in \includegraphics[width, height, ...]{}
\setkeys{Gin}{width=\maxwidth,height=\maxheight,keepaspectratio}
% Set default figure placement to htbp
\makeatletter
\def\fps@figure{htbp}
\makeatother


\newcommand{\MAP}{{\text{MAP}}}
\newcommand{\argmax}{{\text{argmax}}}
\newcommand{\argmin}{{\text{argmin}}}
\newcommand{\Cov}{{\text{Cov}}}
\newcommand{\Var}{{\text{Var}}}
\newcommand{\logistic}{{\text{logistic}}}
\usepackage{authblk}
\usepackage{orcidlink}
\definecolor{mypink}{RGB}{219, 48, 122}
\makeatletter
\makeatother
\makeatletter
\makeatother
\makeatletter
\@ifpackageloaded{caption}{}{\usepackage{caption}}
\AtBeginDocument{%
\ifdefined\contentsname
  \renewcommand*\contentsname{Table of contents}
\else
  \newcommand\contentsname{Table of contents}
\fi
\ifdefined\listfigurename
  \renewcommand*\listfigurename{List of Figures}
\else
  \newcommand\listfigurename{List of Figures}
\fi
\ifdefined\listtablename
  \renewcommand*\listtablename{List of Tables}
\else
  \newcommand\listtablename{List of Tables}
\fi
\ifdefined\figurename
  \renewcommand*\figurename{Figure}
\else
  \newcommand\figurename{Figure}
\fi
\ifdefined\tablename
  \renewcommand*\tablename{Table}
\else
  \newcommand\tablename{Table}
\fi
}
\@ifpackageloaded{float}{}{\usepackage{float}}
\floatstyle{ruled}
\@ifundefined{c@chapter}{\newfloat{codelisting}{h}{lop}}{\newfloat{codelisting}{h}{lop}[chapter]}
\floatname{codelisting}{Listing}
\newcommand*\listoflistings{\listof{codelisting}{List of Listings}}
\makeatother
\makeatletter
\@ifpackageloaded{caption}{}{\usepackage{caption}}
\@ifpackageloaded{subcaption}{}{\usepackage{subcaption}}
\makeatother
\makeatletter
\@ifpackageloaded{tcolorbox}{}{\usepackage[skins,breakable]{tcolorbox}}
\makeatother
\makeatletter
\@ifundefined{shadecolor}{\definecolor{shadecolor}{rgb}{.97, .97, .97}}
\makeatother
\makeatletter
\makeatother
\makeatletter
\makeatother
\ifLuaTeX
  \usepackage{selnolig}  % disable illegal ligatures
\fi
\usepackage[]{natbib}
\bibliographystyle{simple-article}
\IfFileExists{bookmark.sty}{\usepackage{bookmark}}{\usepackage{hyperref}}
\IfFileExists{xurl.sty}{\usepackage{xurl}}{} % add URL line breaks if available
\urlstyle{same} % disable monospaced font for URLs

\hypersetup{
  pdftitle={The F Test},
  pdfauthor={Caleb Leedy},
  colorlinks=true,
  linkcolor={Blue},
  filecolor={Blue},
  citecolor={Blue},
  urlcolor={Maroon},
  pdfcreator={LaTeX via pandoc}}


\title{The F Test}


\newif\ifnoAffil\noAffilfalse
\newif\ifoneAffil\oneAffilfalse

\newcommand{\setNumAffil}[1]{%
    \ifnum#1=0
        \noAffiltrue
    \else
        \ifnum#1=1
            \oneAffiltrue
        \fi
    \fi
}

\setNumAffil{0}

\ifnoAffil
%% No affiliation given....
\author{Caleb Leedy}
\else
\ifoneAffil
%% Single affiliation given...
\author{Caleb Leedy}
\else
%% Multiple affiliations given
\author[]{Caleb Leedy}
\fi
\fi

\date{19 January 2024}

 
 
\fancyhead[C]{Caleb Leedy $\bullet$ The F Test}


\begin{document}


\maketitle

\ifdefined\Shaded\renewenvironment{Shaded}{\begin{tcolorbox}[boxrule=0pt, breakable, frame hidden, interior hidden, sharp corners, borderline west={3pt}{0pt}{shadecolor}, enhanced]}{\end{tcolorbox}}\fi




\hypertarget{computation}{%
\section{Computation}\label{computation}}

To compute an F test we can use Equation~\ref{eq-fstat}.

\begin{equation}\protect\hypertarget{eq-fstat}{}{ F = \frac{(SSE_{Reduced} - SSE_{Full}) / (DFE_{Reduced} - DFE_{Full})}{
SSE_{Full} / DFE_{Full}}. }\label{eq-fstat}\end{equation}

Let's assess what this would mean to compare models
\texttt{Parametric\ 1} and \texttt{Outcome\ Robust} from
Table~\ref{tbl-mods}.

\hypertarget{tbl-mods}{}
\begin{longtable}[]{@{}
  >{\raggedright\arraybackslash}p{(\columnwidth - 2\tabcolsep) * \real{0.2941}}
  >{\raggedright\arraybackslash}p{(\columnwidth - 2\tabcolsep) * \real{0.7059}}@{}}
\caption{\label{tbl-mods}This table identifies the different constraints
for each model type.}\tabularnewline
\toprule\noalign{}
\begin{minipage}[b]{\linewidth}\raggedright
Type
\end{minipage} & \begin{minipage}[b]{\linewidth}\raggedright
Constraints
\end{minipage} \\
\midrule\noalign{}
\endfirsthead
\toprule\noalign{}
\begin{minipage}[b]{\linewidth}\raggedright
Type
\end{minipage} & \begin{minipage}[b]{\linewidth}\raggedright
Constraints
\end{minipage} \\
\midrule\noalign{}
\endhead
\bottomrule\noalign{}
\endlastfoot
Parametric 1 & \(\sum_{k, t} c_{kt} = 1\) \\
Parametric 2 &
\(\sum_{k, t: (k, t) \neq (4, 4)} c_{kt} = 0, c_{44} = 1\) \\
Outcome Robust &
\(c_{11} + c_{21} + c_{31} + c_{41} = 0, c_{22} + c_{42} = 0, c_{33} + c_{43} = 0, \text{ and } c_{44} = 1.\) \\
Response Robust &
\(c_{11} = \pi_{00}, c_{21} + c_{22} = \pi_{10}, c_{31} + c_{33} = \pi_{01}, \text{ and } c_{41} + c_{42} + c_{43} + c_{44} = \pi_{11}\) \\
Double Robust &
\(c_{11} + c_{21} + c_{31} + c_{41} = 0, c_{22} + c_{42} = 0, c_{33} + c_{43} = 0, c_{44} = 1, c_{11} = \pi_{00}, c_{21} + c_{22} = \pi_{10}, c_{31} + c_{33} = \pi_{01}, \text{ and } c_{41} + c_{42} + c_{43} + c_{44} = \pi_{11}\) \\
\end{longtable}

Consider the case where we run estimate each model on a data set with
\(n = 1000\) observations, and define the following notation: let
\(\hat \theta^{(P)}\) and \(\hat \theta^{(OR)}\) be the estimated values
of \(\theta = E[Y_2]\) for the \texttt{Parametric\ 1} model and the
\texttt{Outcome\ Robust} model respectively. Define the estimated
coefficients to be \(\hat c_j^{(P)}\) and \(\hat c_j^{(OR)}\) where
\(j = 1, \dots 9\) for the \texttt{Parametric\ 1} and
\texttt{Outcome\ Robust} models respectively. The one can compute the
SSE with the following:

\[ SSE = n^{-1} \sum_{i = 1}^n (\hat y_{2i} - \theta)^2 \text{ where }
\hat y_{2i} = \sum_{j = 1}^9 \hat c_j \hat \gamma_j \]

and
\(\hat \gamma_0 := \hat \gamma_{00} = \frac{\delta_{00i}}{\pi_{00}} E[Y_2 \mid x_i]\),
\(\hat \gamma_1 := \hat \gamma_{11} = \frac{\delta_{10i}}{\pi_{10}} E[Y_2 \mid x_i, y_{1i}]\),
etc. Note that this is different from the previous estimate of the
variance which used the Monte Carlo variance (standard deviation)
defined by
\[ \frac{1}{n - 1} \sum_{b = 1}^B (\hat \theta_b - \bar \theta_B)^2 \]
where \(B\) is the number of Monte Carlo estimates and
\(\bar \theta_B = \frac{1}{B} \sum_{b = 1}^B \hat \theta_b\).

Likewise, one can compute the degrees of freedom by noticing that each
model has a degrees of freedom equal to nine minus the number of
constraints. This means that we can compute the model degrees of freedom
and error degrees of freedom for each model type, which we do in
Table~\ref{tbl-df}

\hypertarget{tbl-df}{}
\begin{longtable}[]{@{}lll@{}}
\caption{\label{tbl-df}This table displays the degrees of freedom for
each model}\tabularnewline
\toprule\noalign{}
Model & Model Degrees of Freedom & Error Degrees of Freedom \\
\midrule\noalign{}
\endfirsthead
\toprule\noalign{}
Model & Model Degrees of Freedom & Error Degrees of Freedom \\
\midrule\noalign{}
\endhead
\bottomrule\noalign{}
\endlastfoot
Parametric 1 & \(9 - 1 = 8\) & \(n - 1 - 8 = n - 9\) \\
Parametric 2 & \(9 - 1 = 8\) & \(n - 1 - 8 = n - 9\) \\
Outcome Robust & \(9 - 4 = 5\) & \(n - 1 - 5 = n - 6\) \\
Response Robust & \(9 - 4 = 5\) & \(n - 1 - 5 = n - 6\) \\
Double Robust & \(9 - 8 = 1\) & \(n - 1 - 1 = n - 2\) \\
\end{longtable}

So continuing our first example, if \(SSE^{(P)} = 993\) and
\(SSE^{(OR)} = 3428\) with \(n = 1000\) then the F statistic is

\[F = \frac{(3428 - 993) / (3)}{993 / (1000 - 9)} = 810.\]

The critical value we want to compare this with is the \(0.95\) quantile
of \(F_{3, 993}\) which is \(2.61\). So there \emph{is} a significant
difference between the fit of these two model. (The p-value is basically
zero.)



\end{document}
