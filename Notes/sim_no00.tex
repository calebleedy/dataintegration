% Options for packages loaded elsewhere
\PassOptionsToPackage{unicode}{hyperref}
\PassOptionsToPackage{hyphens}{url}
\PassOptionsToPackage{dvipsnames,svgnames,x11names}{xcolor}
%
\documentclass[
  letterpaper,
  DIV=11,
  numbers=noendperiod]{scrartcl}

\usepackage{amsmath,amssymb}
\usepackage{iftex}
\ifPDFTeX
  \usepackage[T1]{fontenc}
  \usepackage[utf8]{inputenc}
  \usepackage{textcomp} % provide euro and other symbols
\else % if luatex or xetex
  \usepackage{unicode-math}
  \defaultfontfeatures{Scale=MatchLowercase}
  \defaultfontfeatures[\rmfamily]{Ligatures=TeX,Scale=1}
\fi
\usepackage{lmodern}
\ifPDFTeX\else  
    % xetex/luatex font selection
\fi
% Use upquote if available, for straight quotes in verbatim environments
\IfFileExists{upquote.sty}{\usepackage{upquote}}{}
\IfFileExists{microtype.sty}{% use microtype if available
  \usepackage[]{microtype}
  \UseMicrotypeSet[protrusion]{basicmath} % disable protrusion for tt fonts
}{}
\makeatletter
\@ifundefined{KOMAClassName}{% if non-KOMA class
  \IfFileExists{parskip.sty}{%
    \usepackage{parskip}
  }{% else
    \setlength{\parindent}{0pt}
    \setlength{\parskip}{6pt plus 2pt minus 1pt}}
}{% if KOMA class
  \KOMAoptions{parskip=half}}
\makeatother
\usepackage{xcolor}
\setlength{\emergencystretch}{3em} % prevent overfull lines
\setcounter{secnumdepth}{-\maxdimen} % remove section numbering
% Make \paragraph and \subparagraph free-standing
\ifx\paragraph\undefined\else
  \let\oldparagraph\paragraph
  \renewcommand{\paragraph}[1]{\oldparagraph{#1}\mbox{}}
\fi
\ifx\subparagraph\undefined\else
  \let\oldsubparagraph\subparagraph
  \renewcommand{\subparagraph}[1]{\oldsubparagraph{#1}\mbox{}}
\fi


\providecommand{\tightlist}{%
  \setlength{\itemsep}{0pt}\setlength{\parskip}{0pt}}\usepackage{longtable,booktabs,array}
\usepackage{calc} % for calculating minipage widths
% Correct order of tables after \paragraph or \subparagraph
\usepackage{etoolbox}
\makeatletter
\patchcmd\longtable{\par}{\if@noskipsec\mbox{}\fi\par}{}{}
\makeatother
% Allow footnotes in longtable head/foot
\IfFileExists{footnotehyper.sty}{\usepackage{footnotehyper}}{\usepackage{footnote}}
\makesavenoteenv{longtable}
\usepackage{graphicx}
\makeatletter
\def\maxwidth{\ifdim\Gin@nat@width>\linewidth\linewidth\else\Gin@nat@width\fi}
\def\maxheight{\ifdim\Gin@nat@height>\textheight\textheight\else\Gin@nat@height\fi}
\makeatother
% Scale images if necessary, so that they will not overflow the page
% margins by default, and it is still possible to overwrite the defaults
% using explicit options in \includegraphics[width, height, ...]{}
\setkeys{Gin}{width=\maxwidth,height=\maxheight,keepaspectratio}
% Set default figure placement to htbp
\makeatletter
\def\fps@figure{htbp}
\makeatother


\newcommand{\MAP}{{\text{MAP}}}
\newcommand{\argmax}{{\text{argmax}}}
\newcommand{\argmin}{{\text{argmin}}}
\newcommand{\Cov}{{\text{Cov}}}
\newcommand{\Var}{{\text{Var}}}
\newcommand{\logistic}{{\text{logistic}}}
\renewcommand{\arraystretch}{1.3}
\KOMAoption{captions}{tableheading}
\makeatletter
\@ifpackageloaded{caption}{}{\usepackage{caption}}
\AtBeginDocument{%
\ifdefined\contentsname
  \renewcommand*\contentsname{Table of contents}
\else
  \newcommand\contentsname{Table of contents}
\fi
\ifdefined\listfigurename
  \renewcommand*\listfigurename{List of Figures}
\else
  \newcommand\listfigurename{List of Figures}
\fi
\ifdefined\listtablename
  \renewcommand*\listtablename{List of Tables}
\else
  \newcommand\listtablename{List of Tables}
\fi
\ifdefined\figurename
  \renewcommand*\figurename{Figure}
\else
  \newcommand\figurename{Figure}
\fi
\ifdefined\tablename
  \renewcommand*\tablename{Table}
\else
  \newcommand\tablename{Table}
\fi
}
\@ifpackageloaded{float}{}{\usepackage{float}}
\floatstyle{ruled}
\@ifundefined{c@chapter}{\newfloat{codelisting}{h}{lop}}{\newfloat{codelisting}{h}{lop}[chapter]}
\floatname{codelisting}{Listing}
\newcommand*\listoflistings{\listof{codelisting}{List of Listings}}
\makeatother
\makeatletter
\makeatother
\makeatletter
\@ifpackageloaded{caption}{}{\usepackage{caption}}
\@ifpackageloaded{subcaption}{}{\usepackage{subcaption}}
\makeatother
\ifLuaTeX
  \usepackage{selnolig}  % disable illegal ligatures
\fi
\usepackage{bookmark}

\IfFileExists{xurl.sty}{\usepackage{xurl}}{} % add URL line breaks if available
\urlstyle{same} % disable monospaced font for URLs
\hypersetup{
  pdftitle={Non-Monotone GLS Simulation: No A\_\{00\}},
  pdfauthor={Caleb Leedy},
  colorlinks=true,
  linkcolor={blue},
  filecolor={Maroon},
  citecolor={Blue},
  urlcolor={Blue},
  pdfcreator={LaTeX via pandoc}}

\title{Non-Monotone GLS Simulation: No \(A_{00}\)}
\author{Caleb Leedy}
\date{1 February 2024}

\begin{document}
\maketitle

\section{Introduction}\label{introduction}

The goal of this report is to conduct a simulation study to show the
validity of using a GLS estimator when intermediate models do not also
estimate the parameter in question. This setup is different from the
previous setup is a couple of ways:

\begin{enumerate}
\def\labelenumi{\arabic{enumi}.}
\tightlist
\item
  We do not use \(E[g \mid G_r(Z)]\) as the \(g\)-functions. Instead, we
  have \(g = (g_1,
  g_2, g_3)' = (X, Y_1, Y_2)'\).
\item
  We have fewer comparison estimators. Since we changed the
  \(g\)-functions it makes less sense to compare then to other
  estimators that are using different intermediate estimators.
\item
  We use a simple random sample (SRS) instead of a Poisson sample. For
  this setup, we have each segment being totally independent of each
  other. Each segment also has a fixed sample size of 250 instead of
  having segments with random sample sizes with a total observation
  count of 1000.
\item
  The GLS estimator is now only estimating \(\theta = E[Y_2] = E[g_3]\).
\end{enumerate}

\section{Notation and Setup}\label{notation-and-setup}

Let \(Z = (X, Y_1, Y_2)'\). We want to estimate the parameter
\(\theta = E[Y_2]\) where we may not always observe \(Y_1\) and \(Y_2\).
Define segments that contain observations in which the same variables
are observed as in Table~\ref{tbl-vars1}.

\begin{longtable}[]{@{}ll@{}}
\caption{This table identifies which variables are observed in each
segment. Since \(X\) is always observed, the subscript for each segment
identifies which of variables \(Y_1\) and \(Y_2\) are in the segment
based on the position of a 1.}\label{tbl-vars1}\tabularnewline
\toprule\noalign{}
Segment & Variables Observed \\
\midrule\noalign{}
\endfirsthead
\toprule\noalign{}
Segment & Variables Observed \\
\midrule\noalign{}
\endhead
\bottomrule\noalign{}
\endlastfoot
\(A_{00}\) & \(X\) \\
\(A_{10}\) & \(X, Y_1\) \\
\(A_{01}\) & \(X, Y_2\) \\
\(A_{11}\) & \(X, Y_1, Y_2\) \\
\end{longtable}

Let \(\delta_{i, j_1, j_2}\) be the sample inclusion indicator for
observation \(i\) in segment \(A_{j_1, j_2}\), and let
\(\pi_{j_1, j_2}\) be the probability of selecting an element into
\(A_{j_1, j_2}\).

We consider the vector \(g(Z) = (g_1(X), g_2(Y_1), g_3(Y_2))'\) and
\emph{for this simulation setup}, let \(g_1\), \(g_2\), and \(g_3\) all
be the identity function \(I()\). This means that
\(g(Z) = (X, Y_1, Y_2)'\). Notice, that we have
\(\theta = E[Y_2] = E[g_3]\). In each segment \(A_{j_1, j_2}\) we can
obtain estimators of some of the \(g\) elements. We have the following:

\begin{align*}
g_1^{(11)} &= n^{-1} \sum_{i = 1}^n \frac{\delta_{11}}{\pi_{11}} g_1(x_i) \\
g_2^{(11)} &= n^{-1} \sum_{i = 1}^n \frac{\delta_{11}}{\pi_{11}} g_2(y_{1i}) \\
g_3^{(11)} &= n^{-1} \sum_{i = 1}^n \frac{\delta_{11}}{\pi_{11}} g_3(y_{2i}) \\
g_1^{(10)} &= n^{-1} \sum_{i = 1}^n \frac{\delta_{10}}{\pi_{10}} g_1(x_i) \\
g_2^{(10)} &= n^{-1} \sum_{i = 1}^n \frac{\delta_{10}}{\pi_{10}} g_2(y_{1i}) \\
g_1^{(01)} &= n^{-1} \sum_{i = 1}^n \frac{\delta_{01}}{\pi_{01}} g_1(x_i) \\
g_3^{(01)} &= n^{-1} \sum_{i = 1}^n \frac{\delta_{01}}{\pi_{01}} g_2(y_{2i}) \\
g_1^{(00)} &= n^{-1} \sum_{i = 1}^n \frac{\delta_{00}}{\pi_{00}} g_1(x_i) \\
\end{align*}

This yields the following linear estimator,

\[ \hat g = Zg + e \]

where

\[\hat g = 
\begin{bmatrix}
g_1^{(11)} \\
g_2^{(11)} \\
g_3^{(11)} \\
g_1^{(10)} \\
g_2^{(10)} \\
g_1^{(01)} \\
g_3^{(01)} \\
g_1^{(00)} \\
\end{bmatrix},
Z = 
\begin{bmatrix}
1 & 0 & 0 \\
0 & 1 & 0 \\
0 & 0 & 1 \\
1 & 0 & 0 \\
0 & 1 & 0 \\
1 & 0 & 0 \\
0 & 0 & 1 \\
1 & 0 & 0 \\
\end{bmatrix},
E[e] = 0,
\text{ and } 
\Var(e) = n^{-1}
\begin{bmatrix}
V_{11} & 0 & 0 & 0 \\
0 & V_{10} & 0 & 0 \\
0 & 0 & V_{01} & 0 \\
0 & 0 & 0 & V_{00} \\
\end{bmatrix}.
\]

Here, we also have

\[
V_{11} = 
\begin{bmatrix}
\frac{1}{\pi_{11}} E[g_1^2] - E[g_1]^2 & \frac{1}{\pi_{11}} E[g_1g_2] -
E[g_1]E[g_2] & \frac{1}{\pi_{11}} E[g_1g_3] - E[g_1]E[g_3] \\
\frac{1}{\pi_{11}} E[g_1g_2] - E[g_1]E[g_2] & \frac{1}{\pi_{11}} E[g_2^2] -
E[g_2]^2 & \frac{1}{\pi_{11}} E[g_2g_3] - E[g_2]E[g_3] \\
\frac{1}{\pi_{11}} E[g_1g_3] - E[g_1]E[g_3] & \frac{1}{\pi_{11}} E[g_2g_3] -
E[g_2]E[g_3] & \frac{1}{\pi_{11}} E[g_3^2] - E[g_3]^2 \\
\end{bmatrix},
\] \[
V_{10} = 
\begin{bmatrix}
\frac{1}{\pi_{10}}E[g_1^2] - E[g_1]^2 & \frac{1}{\pi_{10}}E[g_1g_2] - E[g_1]E[g_2]\\
\frac{1}{\pi_{10}}E[g_1g_2] - E[g_1]E[g_2] & \frac{1}{\pi_{10}}E[g_2^2] - E[g_2]^2 \\
\end{bmatrix},
\] \[
V_{01} = 
\begin{bmatrix}
\frac{1}{\pi_{01}}E[g_1^2] - E[g_1]^2 & \frac{1}{\pi_{01}}E[g_1g_3] - E[g_1]E[g_3]\\
\frac{1}{\pi_{01}}E[g_1g_3] - E[g_1]E[g_3] & \frac{1}{\pi_{01}}E[g_3^2] - E[g_3]^2 \\
\end{bmatrix}, \text{ and }
V_{00} = 
\begin{bmatrix}
\frac{1}{\pi_{00}}E[g_1^2] - E[g_1]^2
\end{bmatrix}.
\]

\newpage{}

\section{Simulation}\label{simulation}

We use the following simulation setup

\begin{align*}
  \begin{bmatrix} x \\ e_1 \\ e_2 \end{bmatrix} 
  &\stackrel{ind}{\sim} N\left(\begin{bmatrix} 0 \\ 0 \\ 0 \end{bmatrix}, 
  \begin{bmatrix} 1 & 0 & 0 \\ 0 & 1 & \rho \\ 0 & \rho & 1 \end{bmatrix}\right) \\
  y_1 &= x + e_1 \\
  y_2 &= \mu + x + e_2 \\
\end{align*}

This yields outcome variables \(Y_1\) and \(Y_2\) that are correlated
both with \(X\) and additionally with each other. To generate the
missingness pattern, we select 250 observations into the four segments
independently.

\begin{longtable}[]{@{}lrrrr@{}}
\caption{Results from simulations study with independent equally sized
segments \(A_{11}\), \(A_{10}\), and \(A_{01}\) all of size \(n = 250\).
In this simulation we have the true mean of \(Y_2\) equal to \(\mu = 5\)
and the covariance between \(e_1\) and \(e_2\) is \(\rho = 0.5\). The
goal is to estimate \(E[Y_2] = \mu\). For the GLS estimation, we use the
true known covariance matrix (using the true values of \(\mu\) and
\(\rho\)), and we use g-functions \(g_1 = X\), \(g_2 = Y_1\) and
\(g_3 = Y_2\).}\tabularnewline
\toprule\noalign{}
Algorithm & Bias & SD & Tstat & Pval \\
\midrule\noalign{}
\endfirsthead
\toprule\noalign{}
Algorithm & Bias & SD & Tstat & Pval \\
\midrule\noalign{}
\endhead
\bottomrule\noalign{}
\endlastfoot
Oracle & -0.001 & 0.049 & -0.818 & 0.207 \\
CC & -0.004 & 0.089 & -1.317 & 0.094 \\
IPW & -0.004 & 0.089 & -1.317 & 0.094 \\
GLS & -0.002 & 0.053 & -1.490 & 0.068 \\
\end{longtable}

A quick comparison between the standard deviation of the GLS estimator
and the standard deviation of the GLS estimator in \texttt{gls\_sim.qmd}
shows that this standard deviation is slightly larger, which means that
we do lose efficiency by not including \(A_{00}\).



\end{document}
