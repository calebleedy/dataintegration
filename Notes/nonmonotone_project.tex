\documentclass[12pt]{article}

\usepackage{amsmath, amssymb, mathrsfs, fancyhdr}
\usepackage{syntonly, lastpage, hyperref, enumitem, graphicx}
\usepackage{biblatex}
\usepackage{booktabs}
\usepackage{float}

\addbibresource{references.bib}

\hypersetup{colorlinks = true, urlcolor = black}

\headheight     15pt
\topmargin      -1.5cm   % read Lamport p.163
\oddsidemargin  -0.04cm  % read Lamport p.163
\evensidemargin -0.04cm  % same as oddsidemargin but for left-hand pages
\textwidth      16.59cm
\textheight     23.94cm
\parskip         7.2pt   % sets spacing between paragraphs
\parindent         0pt   % sets leading space for paragraphs
\pagestyle{empty}        % Uncomment if don't want page numbers
\pagestyle{fancyplain}

\newcommand{\MAP}{{\text{MAP}}}
\newcommand{\argmax}{{\text{argmax}}}
\newcommand{\argmin}{{\text{argmin}}}
\newcommand{\Cov}{{\text{Cov}}}
\newcommand{\logistic}{{\text{logistic}}}

\begin{document}

%\lhead{Caleb Leedy}
\chead{Nonmonotone Missingness}
%\chead{STAT 615 - Advanced Bayesian Methods}
%\rhead{Page \thepage\ of \pageref{LastPage}}
\rhead{April 4, 2023}

\section*{Overview:}

The goal of this project is to outperform existing techniques in the literature
related to nonmonotone missing data.

\section*{Completed:}

\begin{itemize}
  \item \textbf{Implemented simulation of monotone MAR data}: This is 
  correspondingly easier
  than the subsequent nonmonotone MAR simulation. For this simulation we use the 
  following approach:
  \begin{enumerate}
      \item Generate $X$, $Y_1$, and $Y_2$ for elements $i = 1, \dots, n$.
      \item Using the covariate $X$, determine the probability $p_1$ of $Y_1$
      being observed for each element $i$.
      \item Based on $p_1$, determine if $R_1 = 1$.
      \item If $R_1 = 0$, then $R_2 = 0$. Otherwise, using variables $X$ and
        $Y_1$, determine the probability $p_{12}$.
      \item Based on $p_{12}$ determine if $R_2 = 1$.
  \end{enumerate}
  At the end of the algorithm, we have determined the values of binary variables
    $R_1$ and $R_2$ for each $i$ and if either of them are equal to 1, the
    corresponding level of $Y_{k}$. As is common in this literature, the values
    of $R_1$ and $R_2$ determine if the corresponding variable $Y_1$ or $Y_2$ is
    missing or observed with $R = 1$ indicating $Y$ being observed.
  
  \item \textbf{Implemented simulation of nonmonotone MAR data}: 
  Following the approach
    of \cite{robins1997non}, I construct a nonmonotone MAR simulation with two 
    response variables $Y_1$ and $Y_2$ and one covariate $X$. The algorithm to
    generate the data is the following:
    \begin{enumerate}
        \item Generate $X$, $Y_1$, and $Y_2$ for elements $i = 1, \dots, n$.
        \item Using the covariate $X_i$, generate probabilities for each element
          $i$ $p_0$, $p_1$, and $p_2$ such that $p_0 + p_1 + p_2 = 1$. 
        \item Select one option based on the three probabilities for each
          element $i$. If 0 is selected: $R_1 = 0$ and $R_2 = 0$. If 1 is
          selected $R_1 = 1$. If 2 is selected, $R_2 = 1$.
        \item We take the next step in multiple cases. If 0 was selected, we are
          done. If 1 was selected, we generate probabilities $p_{12}$ based on
          $X$ and $Y_1$. Then based on this probability, we determine if $R_2 =
          1$. In the same manner, if 2 was selected in the previous step, we
          generate probabilities $p_{21}$ based on $X$ and $Y_2$. Then based on
          this probability, we determine if $R_2 = 1$.
    \end{enumerate}
    Like the monotone MAR simulation this algorithm produces similar final
    results with the determination of binary variables $R_1$ and $R_2$ and
    variables $X$, $Y_1$, and $Y_2$. Unlike the monotone MAR case, the
    nonmonotone MAR includes observations with $Y_2$ observed and $Y_1$ missing.
    
    \item \textbf{Analyzed simulation from monotone MAR}:
    Following the algorithm described in the monotone MAR simulation bullet, 
    we first generate data from the following distributions:
    \begin{align*}
        X_i &\stackrel{iid}{\sim} N(0, 1) \\
        Y_{1i} &\stackrel{iid}{\sim} N(0, 1)\\
        Y_{2i} &\stackrel{iid}{\sim} N(\theta, 1)
    \end{align*}

    Then, we create the probabilities $p_1 = \logistic(x_i)$ and 
    $p_{12} = \logistic(y_{1i})$.
    Since, both $x_i$ and $y_1$ are standard normal distributions, each of these
    probabilities is approximately $0.5$ in expectation.

    The goal of this simulation is to estimate $\theta$. Alternatively, we can
    express this as solving the estimating equation:

    \[g(\theta) \equiv Y_2 - \theta = 0.\]

    We estimate $\theta$ using the following procedures:

    \begin{itemize}
        \item Oracle: This computes $\bar Y$ using \textit{both} the observed
          and missing data.
        \item IPW-Oracle: This is an IPW estimator using only the observed
          values of $Y_2$. The weights (inverse probabilities) use the actual
          probabilities.
        \item IPW-Est: This is an IPW estimator using the probabilities that
          have been estimated by a logistic model.
        \item Semi: This is the monotone semiparametric efficient estimator from
          Slide 11 (Equation 2) of Dr. Kim's Nonmonotone Missingness
          presentation.
    \end{itemize}

    We run this simulation with different values of $\theta$, sample size of
    1000, and 1000 Monte Carlo replications. Each algorithm for each replication
    generates $\hat \theta$. In the subsequent tables, we compute the bias,
    standard deviation (sd), t-statistic (where we test for a significant
    difference between the Monte Carlo mean $\hat \theta$ and the true $\theta$)
    and the p-value of the t-statistic.
    
    \newpage 

    \begin{table}[h!]
\caption{True Value is -5}
\centering
\begin{tabular}[t]{lrrrr}
\toprule
algorithm & bias & sd & tstat & pval\\
\midrule
oracle & 0.001 & 0.033 & 0.680 & 0.248\\
ipworacle & -0.012 & 0.392 & -0.973 & 0.165\\
ipwest & 0.007 & 0.186 & 1.178 & 0.120\\
semi & 0.001 & 0.074 & 0.538 & 0.295\\
\bottomrule
\end{tabular}
\end{table}

    \begin{table}[h!]
\caption{True Value is 0}
\centering
\begin{tabular}[t]{lrrrr}
\toprule
algorithm & bias & sd & tstat & pval\\
\midrule
oracle & -0.001 & 0.031 & -1.091 & 0.138\\
ipworacle & -0.001 & 0.085 & -0.201 & 0.420\\
ipwest & 0.000 & 0.085 & -0.029 & 0.488\\
semi & 0.000 & 0.079 & 0.112 & 0.455\\
\bottomrule
\end{tabular}
\end{table}

    \begin{table}[h!]
\caption{True Value is 5}
\centering
\begin{tabular}[t]{lrrrr}
\toprule
algorithm & bias & sd & tstat & pval\\
\midrule
oracle & 0.000 & 0.033 & -0.468 & 0.320\\
ipworacle & 0.010 & 0.383 & 0.857 & 0.196\\
ipwest & -0.006 & 0.176 & -1.020 & 0.154\\
semi & 0.000 & 0.077 & -0.049 & 0.481\\
\bottomrule
\end{tabular}
\end{table}


    Overall, these results are mostly what I would have expected. All of the
    algorithms estimate the true value of $\theta$ correctly in each case, with
    the oracle estimate having the smallest variance followed by the
    semiparametric algorithm. If there is anything surprising it is that the IPW
    estimator has better performance with the estimated weights compared to the
    true weights. However, I think that this is a known phenomenon.

    \newpage 
    
    \item \textbf{Analyzed simulation from nonmonotone MAR}:
    Like the monotone simulation, we generate data using:
    \begin{align*}
        X_i &\stackrel{iid}{\sim} N(0, 1) \\
        Y_{1i} &\stackrel{iid}{\sim} N(0, 1)\\
        Y_{2i} &\stackrel{iid}{\sim} N(\theta, 1)
    \end{align*}

    However, because we have nonmonotone data, our ``Stage 1'' probabilities are
    different. We compute the true Stage 1 probabilities being proportional to
    the following values:
    
    \begin{align*}
        p_0 &\propto |x - 2|\\
        p_1 &\propto |x|\\
        p_2 &\propto |x + 2|.
    \end{align*}
    
   However, we keep the same structure for the Stage 2 probabilities with:
   $p_{12} = \logistic(y_1)$ and $p_{21} = \logistic(y_2)$. The goal remains to
   estimate $\theta$. We continue to use the Oracle algorithm and the IPW-Oracle
   algorithm. Since we have nonmonotone MAR data, we use the ``Proposed''
   algorithm that is described on Slide 25 (Equation 12) of Dr. Kim's
   presentation. The response models and outcome models were estimated using
   logistic regression and OLS and correctly specified. This yields the
   following results:
   
    \begin{table}[h!]
\caption{True Value is -5}
\centering
\begin{tabular}[t]{lrrrr}
\toprule
algorithm & bias & sd & tstat & pval\\
\midrule
oracle & 0.000 & 0.033 & 0.144 & 0.443\\
ipworacle & 0.026 & 0.882 & 0.925 & 0.178\\
proposed & 0.008 & 0.065 & 4.061 & 0.000\\
\bottomrule
\end{tabular}
\end{table}

    \begin{table}[h!]
\caption{True Value is 0}
\centering
\begin{tabular}[t]{lrrrr}
\toprule
algorithm & bias & sd & tstat & pval\\
\midrule
oracle & -0.001 & 0.032 & -1.112 & 0.133\\
ipworacle & 0.001 & 0.074 & 0.462 & 0.322\\
proposed & 0.001 & 0.051 & 0.444 & 0.329\\
\bottomrule
\end{tabular}
\end{table}

    \begin{table}[h!]
\caption{True Value is 5}
\centering
\begin{tabular}[t]{lrrrr}
\toprule
algorithm & bias & sd & tstat & pval\\
\midrule
oracle & 0.001 & 0.031 & 0.609 & 0.271\\
ipworacle & -0.001 & 0.199 & -0.181 & 0.428\\
proposed & -0.001 & 0.050 & -0.343 & 0.366\\
\bottomrule
\end{tabular}
\end{table}


    \newpage
    
    While most of these numbers are to be expected, the proposed algorithm
    exhibits considerable bias with the true value of $\theta = -5$. Why does
    this happen? When $\theta = -5$, $p_{21} = \logistic(y_2)$ is very small
    ($\logistic(-5) = 0.007$). This means that in many replications every single
    observation that is initially selected to have $R_2 = 1$ chooses to have
    $R_1 = 0$. I think that this is causing the problem, but I am not totally
    sure how to check.

    \item \textbf{Simulation 2 with Nonmonotone MAR}:
    We also want to simulate data that is correlated. For this simulation, we
    focus on $\Cov(Y_1, Y_2)$. While the probabilities are generated the same as
    the previous simulation the data generating process is 

    \[\begin{bmatrix}
    X_i \\ Y_{1i} \\ Y_{2i}
    \end{bmatrix} \stackrel{iid}{\sim}
    N\left(
    \begin{bmatrix}
        0 \\ 0 \\ \theta
    \end{bmatrix},
    \begin{bmatrix}
        1 & 0 & 0 \\
        0 & 1 & \sigma_{yy}\\
        0 & \sigma_{yy} & 1
    \end{bmatrix}
    \right).\]

    We are still interested in $\bar Y_2$ and we still run 1000 simulation with
    1000 observations. In all of the next simulations the true value of $\theta
    = 0$. The results are the following:

    \begin{table}[h!]
\caption{True Value is 0. Cor(Y1, Y2) = 0.1}
\centering
\begin{tabular}[t]{lrrrr}
\toprule
algorithm & bias & sd & tstat & pval\\
\midrule
oracle & -0.002 & 0.032 & -2.056 & 0.020\\
ipworacle & 0.002 & 0.077 & 0.913 & 0.181\\
proposed & 0.002 & 0.048 & 0.992 & 0.161\\
\bottomrule
\end{tabular}
\end{table}


    \begin{table}[h!]
\caption{True Value is 0. Cor(Y1, Y2) = 0.5}
\centering
\begin{tabular}[t]{lrrrr}
\toprule
algorithm & bias & sd & tstat & pval\\
\midrule
oracle & -0.001 & 0.032 & -0.766 & 0.222\\
ipworacle & 0.000 & 0.079 & -0.018 & 0.493\\
proposed & -0.006 & 0.050 & -3.534 & 0.000\\
\bottomrule
\end{tabular}
\end{table}


    \begin{table}[h!]
\caption{True Value is 0. Cor(Y1, Y2) = 0.9}
\centering
\begin{tabular}[t]{lrrrr}
\toprule
algorithm & bias & sd & tstat & pval\\
\midrule
oracle & 0.000 & 0.033 & 0.221 & 0.413\\
ipworacle & -0.002 & 0.085 & -0.593 & 0.277\\
proposed & -0.035 & 0.053 & -20.992 & 0.000\\
\bottomrule
\end{tabular}
\end{table}



    \newpage

    While a small correlation (0.1) in Table 7 (also see Table 5 for no
    correlation), shows that the proposed algorithm with an insignificant level
    of bias, with a stronger correlation the bias increases.
\end{itemize}


{\bf Commments from JK}
\begin{itemize}
    \item Three variables $(X, Y_1, Y_2)$ should be correlated. 
    \item I think the proposed method is doubly robust. So, you may make a
      simulation setup testing DR property. I will explain it further. 
\end{itemize}

\newpage

\printbibliography

\end{document}

