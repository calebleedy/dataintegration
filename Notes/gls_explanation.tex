% Options for packages loaded elsewhere
\PassOptionsToPackage{unicode}{hyperref}
\PassOptionsToPackage{hyphens}{url}
\PassOptionsToPackage{dvipsnames,svgnames,x11names}{xcolor}
%
\documentclass[
  letterpaper,
  DIV=11,
  numbers=noendperiod]{scrartcl}

\usepackage{amsmath,amssymb}
\usepackage{iftex}
\ifPDFTeX
  \usepackage[T1]{fontenc}
  \usepackage[utf8]{inputenc}
  \usepackage{textcomp} % provide euro and other symbols
\else % if luatex or xetex
  \usepackage{unicode-math}
  \defaultfontfeatures{Scale=MatchLowercase}
  \defaultfontfeatures[\rmfamily]{Ligatures=TeX,Scale=1}
\fi
\usepackage{lmodern}
\ifPDFTeX\else  
    % xetex/luatex font selection
\fi
% Use upquote if available, for straight quotes in verbatim environments
\IfFileExists{upquote.sty}{\usepackage{upquote}}{}
\IfFileExists{microtype.sty}{% use microtype if available
  \usepackage[]{microtype}
  \UseMicrotypeSet[protrusion]{basicmath} % disable protrusion for tt fonts
}{}
\makeatletter
\@ifundefined{KOMAClassName}{% if non-KOMA class
  \IfFileExists{parskip.sty}{%
    \usepackage{parskip}
  }{% else
    \setlength{\parindent}{0pt}
    \setlength{\parskip}{6pt plus 2pt minus 1pt}}
}{% if KOMA class
  \KOMAoptions{parskip=half}}
\makeatother
\usepackage{xcolor}
\setlength{\emergencystretch}{3em} % prevent overfull lines
\setcounter{secnumdepth}{-\maxdimen} % remove section numbering
% Make \paragraph and \subparagraph free-standing
\ifx\paragraph\undefined\else
  \let\oldparagraph\paragraph
  \renewcommand{\paragraph}[1]{\oldparagraph{#1}\mbox{}}
\fi
\ifx\subparagraph\undefined\else
  \let\oldsubparagraph\subparagraph
  \renewcommand{\subparagraph}[1]{\oldsubparagraph{#1}\mbox{}}
\fi


\providecommand{\tightlist}{%
  \setlength{\itemsep}{0pt}\setlength{\parskip}{0pt}}\usepackage{longtable,booktabs,array}
\usepackage{calc} % for calculating minipage widths
% Correct order of tables after \paragraph or \subparagraph
\usepackage{etoolbox}
\makeatletter
\patchcmd\longtable{\par}{\if@noskipsec\mbox{}\fi\par}{}{}
\makeatother
% Allow footnotes in longtable head/foot
\IfFileExists{footnotehyper.sty}{\usepackage{footnotehyper}}{\usepackage{footnote}}
\makesavenoteenv{longtable}
\usepackage{graphicx}
\makeatletter
\def\maxwidth{\ifdim\Gin@nat@width>\linewidth\linewidth\else\Gin@nat@width\fi}
\def\maxheight{\ifdim\Gin@nat@height>\textheight\textheight\else\Gin@nat@height\fi}
\makeatother
% Scale images if necessary, so that they will not overflow the page
% margins by default, and it is still possible to overwrite the defaults
% using explicit options in \includegraphics[width, height, ...]{}
\setkeys{Gin}{width=\maxwidth,height=\maxheight,keepaspectratio}
% Set default figure placement to htbp
\makeatletter
\def\fps@figure{htbp}
\makeatother


\newcommand{\MAP}{{\text{MAP}}}
\newcommand{\argmax}{{\text{argmax}}}
\newcommand{\argmin}{{\text{argmin}}}
\newcommand{\Cov}{{\text{Cov}}}
\newcommand{\Var}{{\text{Var}}}
\newcommand{\logistic}{{\text{logistic}}}
\renewcommand{\arraystretch}{1.4}
\KOMAoption{captions}{tableheading}
\makeatletter
\makeatother
\makeatletter
\makeatother
\makeatletter
\@ifpackageloaded{caption}{}{\usepackage{caption}}
\AtBeginDocument{%
\ifdefined\contentsname
  \renewcommand*\contentsname{Table of contents}
\else
  \newcommand\contentsname{Table of contents}
\fi
\ifdefined\listfigurename
  \renewcommand*\listfigurename{List of Figures}
\else
  \newcommand\listfigurename{List of Figures}
\fi
\ifdefined\listtablename
  \renewcommand*\listtablename{List of Tables}
\else
  \newcommand\listtablename{List of Tables}
\fi
\ifdefined\figurename
  \renewcommand*\figurename{Figure}
\else
  \newcommand\figurename{Figure}
\fi
\ifdefined\tablename
  \renewcommand*\tablename{Table}
\else
  \newcommand\tablename{Table}
\fi
}
\@ifpackageloaded{float}{}{\usepackage{float}}
\floatstyle{ruled}
\@ifundefined{c@chapter}{\newfloat{codelisting}{h}{lop}}{\newfloat{codelisting}{h}{lop}[chapter]}
\floatname{codelisting}{Listing}
\newcommand*\listoflistings{\listof{codelisting}{List of Listings}}
\makeatother
\makeatletter
\@ifpackageloaded{caption}{}{\usepackage{caption}}
\@ifpackageloaded{subcaption}{}{\usepackage{subcaption}}
\makeatother
\makeatletter
\@ifpackageloaded{tcolorbox}{}{\usepackage[skins,breakable]{tcolorbox}}
\makeatother
\makeatletter
\@ifundefined{shadecolor}{\definecolor{shadecolor}{rgb}{.97, .97, .97}}
\makeatother
\makeatletter
\makeatother
\makeatletter
\makeatother
\ifLuaTeX
  \usepackage{selnolig}  % disable illegal ligatures
\fi
\IfFileExists{bookmark.sty}{\usepackage{bookmark}}{\usepackage{hyperref}}
\IfFileExists{xurl.sty}{\usepackage{xurl}}{} % add URL line breaks if available
\urlstyle{same} % disable monospaced font for URLs
\hypersetup{
  pdftitle={GLS Explanation},
  pdfauthor={Caleb Leedy},
  colorlinks=true,
  linkcolor={blue},
  filecolor={Maroon},
  citecolor={Blue},
  urlcolor={Blue},
  pdfcreator={LaTeX via pandoc}}

\title{GLS Explanation}
\author{Caleb Leedy}
\date{31 January 2024}

\begin{document}
\maketitle
\ifdefined\Shaded\renewenvironment{Shaded}{\begin{tcolorbox}[sharp corners, interior hidden, frame hidden, borderline west={3pt}{0pt}{shadecolor}, boxrule=0pt, enhanced, breakable]}{\end{tcolorbox}}\fi

\hypertarget{introduction}{%
\section{Introduction}\label{introduction}}

This is a write up explaning the dual form of generalized least squares
(GLS). We think of GLS as an extension of ordinary least squares (OLS),
so I will discuss this first and then comment on how these are connected
and the dual formulation for both.

\hypertarget{the-gauss-markov-theorem}{%
\section{The Gauss-Markov Theorem}\label{the-gauss-markov-theorem}}

It all comes back to the Gauss-Markov Theorem, which states that for a
linear model in the form,

\[ Y = X\beta + \varepsilon\]

with \(E[\varepsilon] = 0\) and \(\Var(\varepsilon) = \sigma^2 I_n\),
the BLUE (best linear unbiased estimator) of \(C\beta\) for \(C = AX\)
for some matrix \(A\) is (assuming \(X'X\) has an inverse),

\[ C\hat \beta = C(X'X)^{-1} X'Y = AX(X'X)^{-1}X'Y.\]

\hypertarget{gls-extension}{%
\subsection{GLS Extension}\label{gls-extension}}

To extend this to the GLS setting consider the case where
\(\Var(\varepsilon) = \sigma^2 V\) for some known positive-definite
matrix \(V\). In this case, we can rewrite \(Y = X\beta + \varepsilon\)
as

\[ Z = W\beta + \tilde \varepsilon\]

where \(Z = V^{-1/2} Y\), \(W = V^{-1/2}X\) and
\(\tilde \varepsilon = V^{-1/2} \varepsilon\). But this formulation
implies that \(E[\tilde \varepsilon] = 0\) and
\(\Var(\tilde \varepsilon) = \sigma^2 I_n\) and so hence by the
Gauss-Markov Theorem, the BLUE is

\[ (W'W)^{-1} W'Z = (X'V^{-1}V)^{-1} X'V^{-1}Y.\]

\hypertarget{dual-form}{%
\section{Dual Form}\label{dual-form}}

All of this comes back to the fact that we want to estimate the BLUE.
However, we could do this directly. By definition, the BLUE is the
minimum variance linear unbiased estimator. This means that we consider
a class of estimators \(CW'Y\) such that \(E[CW'Y] = C\beta\) that
minimizes the variance \(CW'\Var(Y) WC' \propto CW'WC'\). Since we
assume the model form of \(Y = X\beta + \varepsilon\), the unbiasedness
condition is equivalent to,

\[ E[CW'Y] = CW'E[Y] = CW'X\beta = C\beta. \]

Since this is true for all \(\beta\), it must be the case that

\[ W'X = I_p \text{ for } p = \dim(\beta).\]

Hence, we have a Lagrangian

\[L(W) \propto CW'WC' - (W'X - I_p)\lambda.\]

Thus, the first-order conditions are \(W - X\lambda = 0\) and
\(W'X - I_p = 0\). This yields a solution of \(W = X(X'X)^{-1}\) which
is exactly what we get from OLS because the final result is
\(CW'Y = C(X'X)^{-1} X'Y\). To get the GLS form, we can keep the same
constraint except minimize the variance \(W'VW\).

\hypertarget{estimating-the-variance}{%
\section{Estimating the Variance}\label{estimating-the-variance}}

When estimating GLS, we assume that the covariance matrix \(V\) is fully
known. What do we do if we have to estimate \(V\)? In this case, we need
to use (finite-dimensional) semiparametric theory to estimate the
optimal \(W\) such that \(W\) belongs to the orthogonal nuisance tangent
space of \(\hat V\). This is something that we could explore further if
we want to.



\end{document}
