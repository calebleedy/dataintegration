\documentclass[12pt]{article}

\usepackage{amsmath, amssymb, mathrsfs, fancyhdr}
\usepackage{syntonly, lastpage, hyperref, enumitem, graphicx}
\usepackage{biblatex}
\usepackage{booktabs}
\usepackage{float}

\addbibresource{references.bib}

\hypersetup{colorlinks = true, urlcolor = black}

\headheight     15pt
\topmargin      -1.5cm   % read Lamport p.163
\oddsidemargin  -0.04cm  % read Lamport p.163
\evensidemargin -0.04cm  % same as oddsidemargin but for left-hand pages
\textwidth      16.59cm
\textheight     23.94cm
\parskip         7.2pt   % sets spacing between paragraphs
\parindent         0pt   % sets leading space for paragraphs
\pagestyle{empty}        % Uncomment if don't want page numbers
\pagestyle{fancyplain}

\newcommand{\MAP}{{\text{MAP}}}
\newcommand{\argmax}{{\text{argmax}}}
\newcommand{\argmin}{{\text{argmin}}}
\newcommand{\Cov}{{\text{Cov}}}
\newcommand{\Var}{{\text{Var}}}
\newcommand{\logistic}{{\text{logistic}}}

\begin{document}

%\lhead{Caleb Leedy}
\chead{Nonmonotone Missingness}
%\chead{STAT 615 - Advanced Bayesian Methods}
%\rhead{Page \thepage\ of \pageref{LastPage}}
\rhead{April 4, 2023}

\section*{Overview:}

The goal of this project is to outperform existing techniques in the literature
related to nonmonotone missing data.

\section*{Initial Simulations:}
% Tables generated from Simulations/nonmonotone.R

\begin{itemize}
  \item \textbf{Implemented simulation of monotone MAR data}: 
    This is correspondingly easier than the subsequent nonmonotone MAR
    simulation. For this simulation we use the following approach:
  \begin{enumerate}
      \item Generate $X$, $Y_1$, and $Y_2$ for elements $i = 1, \dots, n$.
      \item Using the covariate $X$, determine the probability $p_1$ of $Y_1$
      being observed for each element $i$.
      \item Based on $p_1$, determine if $R_1 = 1$.
      \item If $R_1 = 0$, then $R_2 = 0$. Otherwise, using variables $X$ and
        $Y_1$, determine the probability $p_{12}$.
      \item Based on $p_{12}$ determine if $R_2 = 1$.
  \end{enumerate}
    At the end of the algorithm, we have determined the values of binary
    variables $R_1$ and $R_2$ for each $i$ and if either of them are equal to 1,
    the corresponding level of $Y_{k}$. As is common in this literature, the
    values of $R_1$ and $R_2$ determine if the corresponding variable $Y_1$ or
    $Y_2$ is missing or observed with $R = 1$ indicating $Y$ being observed.
  
  \item \textbf{Implemented simulation of nonmonotone MAR data}: 
    Following the approach of \cite{robins1997non}, I construct a nonmonotone
    MAR simulation with two response variables $Y_1$ and $Y_2$ and one covariate
    $X$. The algorithm to generate the data is the following:
    \begin{enumerate}
        \item Generate $X$, $Y_1$, and $Y_2$ for elements $i = 1, \dots, n$.
        \item Using the covariate $X_i$, generate probabilities for each element
          $i$ $p_0$, $p_1$, and $p_2$ such that $p_0 + p_1 + p_2 = 1$. 
        \item Select one option based on the three probabilities for each
          element $i$. If 0 is selected: $R_1 = 0$ and $R_2 = 0$; if 1 is
          selected $R_1 = 1$; if 2 is selected, $R_2 = 1$.
        \item We take the next step in multiple cases. If 0 was selected, we are
          done. If 1 was selected, we generate probabilities $p_{12}$ based on
          $X$ and $Y_1$. Then based on this probability, we determine if $R_2 =
          1$. In the same manner, if 2 was selected in the previous step, we
          generate probabilities $p_{21}$ based on $X$ and $Y_2$. Then based on
          this probability, we determine if $R_2 = 1$.
    \end{enumerate}
    Like the monotone MAR simulation this algorithm produces similar final
    results with the determination of binary variables $R_1$ and $R_2$ and
    variables $X$, $Y_1$, and $Y_2$. Unlike the monotone MAR case, the
    nonmonotone MAR includes observations with $Y_2$ observed and $Y_1$ missing.
    
  \item \textbf{Simulation 1 with Monotone MAR}:
    Following the algorithm described in the monotone MAR simulation bullet, 
    we first generate data from the following distributions:
    \begin{align*}
        X_i &\stackrel{iid}{\sim} N(0, 1) \\
        Y_{1i} &\stackrel{iid}{\sim} N(0, 1)\\
        Y_{2i} &\stackrel{iid}{\sim} N(\theta, 1)
    \end{align*}

    Then, we create the probabilities $p_1 = \logistic(x_i)$ and 
    $p_{12} = \logistic(y_{1i})$.
    Since, both $x_i$ and $y_1$ are standard normal distributions, each of these
    probabilities is approximately $0.5$ in expectation.

    The goal of this simulation is to estimate $\theta$. Alternatively, we can
    express this as solving the estimating equation:

    \[g(\theta) \equiv Y_2 - \theta = 0.\]

    We estimate $\theta$ using the following procedures:

    \begin{itemize}
        \item Oracle: This computes $\bar Y$ using \textit{both} the observed
          and missing data.
        \item IPW-Oracle: This is an IPW estimator using only the observed
          values of $Y_2$. The weights (inverse probabilities) use the actual
          probabilities.
        \item IPW-Est: This is an IPW estimator using the probabilities that
          have been estimated by a logistic model.
        \item Semi: This is the monotone semiparametric efficient estimator from
          Slide 11 (Equation 2) of Dr. Kim's Nonmonotone Missingness
          presentation.
    \end{itemize}

    We run this simulation with different values of $\theta$, sample size of
    2000, and 2000 Monte Carlo replications. Each algorithm for each replication
    generates $\hat \theta$. In the subsequent tables, we compute the bias,
    standard deviation (sd), t-statistic (where we test for a significant
    difference between the Monte Carlo mean $\hat \theta$ and the true $\theta$)
    and the p-value of the t-statistic.
    
    \newpage 

    \begin{table}[h!]
\caption{True Value is -5}
\centering
\begin{tabular}[t]{lrrrr}
\toprule
algorithm & bias & sd & tstat & pval\\
\midrule
oracle & 0.001 & 0.033 & 0.680 & 0.248\\
ipworacle & -0.012 & 0.392 & -0.973 & 0.165\\
ipwest & 0.007 & 0.186 & 1.178 & 0.120\\
semi & 0.001 & 0.074 & 0.538 & 0.295\\
\bottomrule
\end{tabular}
\end{table}

    \begin{table}[h!]
\caption{True Value is 0}
\centering
\begin{tabular}[t]{lrrrr}
\toprule
algorithm & bias & sd & tstat & pval\\
\midrule
oracle & -0.001 & 0.031 & -1.091 & 0.138\\
ipworacle & -0.001 & 0.085 & -0.201 & 0.420\\
ipwest & 0.000 & 0.085 & -0.029 & 0.488\\
semi & 0.000 & 0.079 & 0.112 & 0.455\\
\bottomrule
\end{tabular}
\end{table}

    \begin{table}[h!]
\caption{True Value is 5}
\centering
\begin{tabular}[t]{lrrrr}
\toprule
algorithm & bias & sd & tstat & pval\\
\midrule
oracle & 0.000 & 0.033 & -0.468 & 0.320\\
ipworacle & 0.010 & 0.383 & 0.857 & 0.196\\
ipwest & -0.006 & 0.176 & -1.020 & 0.154\\
semi & 0.000 & 0.077 & -0.049 & 0.481\\
\bottomrule
\end{tabular}
\end{table}


    Overall, these results are mostly what I would have expected. All the
    algorithms estimate the true value of $\theta$ correctly in each case, with
    the oracle estimate having the smallest variance followed by the
    semiparametric algorithm. If there is anything surprising it is that the IPW
    estimator has better performance with the estimated weights compared to the
    true weights. However, I think that this is a known phenomenon.

    \newpage 
    
    \item \textbf{Simulation 1 with Nonmonotone MAR}:

    We generate variables $(X, Y_1, Y_2)$ using the following setup:

    \[\begin{bmatrix}
    X_i \\ \varepsilon_{1i} \\ \varepsilon_{2i}
    \end{bmatrix} \stackrel{iid}{\sim}
    N\left(
    \begin{bmatrix}
        0 \\ 0 \\ \theta
    \end{bmatrix},
    \begin{bmatrix}
        1 & 0 & 0 \\
        0 & 1 & \sigma_{yy}\\
        0 & \sigma_{yy} & 1
    \end{bmatrix}
    \right).\]

    Then, 

    \[y_{1i} = x_i + \varepsilon_{1i} \text{ and } 
    y_{2i} = x_i + \varepsilon_{2i}.\]

    Since we have nonmonotone data, our ``Stage 1'' probabilities are
    different. We compute the true Stage 1 probabilities being proportional to
    the following values:
    
    \begin{align*}
        p_0 &= 0.2 \\
        p_1 &= 0.4 \\
        p_2 &= 0.4 \\
    \end{align*}
    
   However, we keep the same structure for the Stage 2 probabilities with:
   $p_{12} = \logistic(y_1)$ and $p_{21} = \logistic(y_2)$. The goal remains to
   estimate $\theta$. We continue to use the Oracle algorithm and the IPW-Oracle
   algorithm. Since we have nonmonotone MAR data, we use the ``Proposed''
   algorithm that is described on Slide 25 (Equation 12) of Dr. Kim's
   presentation. The outcome models were estimated using
   logistic regression and OLS and correctly specified. The response model used
   the oracle estimates of the probabilities. This yields the
   following results:
   
    \begin{table}[h!]

\caption{True Value is -5. Cor(Y1, Y2) = 0}
\centering
\begin{tabular}[t]{lrrrr}
\toprule
algorithm & bias & sd & tstat & pval\\
\midrule
oracle & 0.000 & 0.032 & 0.285 & 0.388\\
ipworacle & -0.003 & 0.381 & -0.318 & 0.375\\
proposed & 0.000 & 0.038 & 0.492 & 0.311\\
\bottomrule
\end{tabular}
\end{table}

    \begin{table}[h!]

\caption{True Value is 0. Cor(Y1, Y2) = 0}
\centering
\begin{tabular}[t]{lrrrr}
\toprule
algorithm & bias & sd & tstat & pval\\
\midrule
oracle & 0.000 & 0.032 & 0.285 & 0.388\\
ipworacle & 0.000 & 0.076 & -0.237 & 0.406\\
proposed & 0.001 & 0.038 & 0.894 & 0.186\\
\bottomrule
\end{tabular}
\end{table}

    \begin{table}[h!]

\caption{True Value is 5. Cor(Y1, Y2) = 0}
\centering
\begin{tabular}[t]{lrrrr}
\toprule
algorithm & bias & sd & tstat & pval\\
\midrule
oracle & 0.000 & 0.032 & 0.285 & 0.388\\
ipworacle & -0.001 & 0.098 & -0.479 & 0.316\\
proposed & 0.000 & 0.037 & 0.505 & 0.307\\
\bottomrule
\end{tabular}
\end{table}


    \newpage
    
    \item \textbf{Simulation 2 with Nonmonotone MAR}:
    We also want to simulate data that is correlated. For this simulation, we
    focus on $\Cov(Y_1, Y_2)$. The data generating process now has $\sigma_{yy}
    \neq 0$. We are still interested in $\bar Y_2$ and we still run 2000
    simulation with 2000 observations. In all the next simulations the true
    value of $\theta = 0$. The results are the following:

    \begin{table}[h!]

\caption{True Value is 0. Cor(Y1, Y2) = 0.1}
\centering
\begin{tabular}[t]{lrrrr}
\toprule
algorithm & bias & sd & tstat & pval\\
\midrule
oracle & 0.001 & 0.031 & 1.623 & 0.052\\
ipworacle & 0.001 & 0.077 & 0.762 & 0.223\\
proposed & 0.001 & 0.037 & 1.366 & 0.086\\
\bottomrule
\end{tabular}
\end{table}

    \begin{table}[h!]

\caption{True Value is 0. Cor(Y1, Y2) = 0.5}
\centering
\begin{tabular}[t]{lrrrr}
\toprule
algorithm & bias & sd & tstat & pval\\
\midrule
oracle & 0.001 & 0.032 & 1.486 & 0.069\\
ipworacle & 0.004 & 0.086 & 1.890 & 0.029\\
proposed & 0.000 & 0.041 & 0.172 & 0.432\\
\bottomrule
\end{tabular}
\end{table}

    \begin{table}[h!]

\caption{True Value is 0. Cor(Y1, Y2) = 0.9}
\centering
\begin{tabular}[t]{lrrrr}
\toprule
algorithm & bias & sd & tstat & pval\\
\midrule
oracle & 0.001 & 0.032 & 0.706 & 0.240\\
ipworacle & 0.003 & 0.098 & 1.395 & 0.082\\
proposed & -0.002 & 0.062 & -1.339 & 0.090\\
\bottomrule
\end{tabular}
\end{table}


    \newpage

  \item \textbf{Simulation 3 with Nonmonotone MAR}:
    This simulation aims to see if the proposed algorithm is doubly robust.
    First, we check with a misspecified outcome model. In this case the data
    generating procedure is the following:

    \[\begin{bmatrix}
    X_i \\ \varepsilon_{1i} \\ \varepsilon_{2i}
    \end{bmatrix} \stackrel{iid}{\sim}
    N\left(
    \begin{bmatrix}
        0 \\ 0 \\ \theta
    \end{bmatrix},
    \begin{bmatrix}
        1 & 0 & 0 \\
        0 & 1 & \sigma_{yy}\\
        0 & \sigma_{yy} & 1
    \end{bmatrix}
    \right).\]

    Then, the true outcome model is,

    \[y_{1i} = x_i + x_i^2 \varepsilon_{1i} \text{ and } 
    y_{2i} = -x_i + x_i^3 + \varepsilon_{2i}.\]

    This procedure causes $X$ to influence both $Y_1$ and $Y_2$ and we still
    have correlation in the error terms of $Y_1$ and $Y_2$. However, since
    neither $Y_1$ nor $Y_2$ are linear in $X$, the model will be misspecified.
    The response mechanisms are first generated MCAR with a probability of
    either $Y_1$ or $Y_2$ being the first variable observed to be $0.4$. (There
    is a $0.2$ probability neither is observed.) Then the probability of the
    other variable being observed is proportional to $\logistic(y_k)$ where
    $y_k$ is the $y$ that has been observed. To ensure that the proposed method
    has the correct propensity score we use the oracle probabilities instead of
    estimating them. This yields the following:

    \begin{table}[h!]

\caption{True Value is 0. Cor(Y1, Y2) = 0}
\centering
\begin{tabular}[t]{lrrrr}
\toprule
algorithm & bias & sd & tstat & pval\\
\midrule
oracle & 0.000 & 0.075 & 0.014 & 0.494\\
ipworacle & 0.002 & 0.107 & 0.876 & 0.191\\
proposed & -0.002 & 0.084 & -1.063 & 0.144\\
\bottomrule
\end{tabular}
\end{table}

    \begin{table}[h!]

\caption{True Value is 0. Cor(Y1, Y2) = 0.1}
\centering
\begin{tabular}[t]{lrrrr}
\toprule
algorithm & bias & sd & tstat & pval\\
\midrule
oracle & -0.002 & 0.074 & -1.479 & 0.070\\
ipworacle & 0.000 & 0.106 & -0.196 & 0.422\\
proposed & -0.003 & 0.083 & -1.464 & 0.072\\
\bottomrule
\end{tabular}
\end{table}

    \begin{table}[h!]

\caption{True Value is 0. Cor(Y1, Y2) = 0.5}
\centering
\begin{tabular}[t]{lrrrr}
\toprule
algorithm & bias & sd & tstat & pval\\
\midrule
oracle & -0.003 & 0.074 & -1.567 & 0.059\\
ipworacle & -0.002 & 0.108 & -0.818 & 0.207\\
proposed & -0.003 & 0.083 & -1.633 & 0.051\\
\bottomrule
\end{tabular}
\end{table}


    \newpage

    Thus, the proposed method is unbiased with a misspecified outcome model.
    We now show a simulation where the outcome model is correctly specified, but
    the response model is not.

  \item \textbf{Simulation 4 with Nonmonotone MAR}:
    Continuing to test if the proposed algorithm is doubly robust, this
    simulation checks a misspecified response model. Instead of using oracle
    weights as in Simulation 3, we estimate the weights for the proposed method.
    However, unlike the true probabilities of being proportional to
    $\logistic(y_k)$, this simulation has the true probabilities being
    proportional to $\logistic(x_i)$. Thus, the true response model is the
    following:

    \begin{enumerate}
      \item (Stage 1) Choose a variable observe. We choose $Y_1$ with
        probability $0.4$, $Y_2$ with probability $0.4$ and neither with
        probability $0.2$. If neither, $R_1 = 0$ and $R_2 = 0$. Otherwise,
        continue to Step 2.
      \item (Stage 2) With probability $p_i \propto \logistic(x_i)$, choose to
        observe the other $Y$ variable. 
    \end{enumerate}

    This sequence generates the missingness indicators $R_1$ and $R_2$. Since,
    the Stage 1 probabilities are fixed and known and the Stage 2 probabilities
    only depend on $x_i$, the missingness is MAR and only a function of $x_i$.
    The algorithms to which we compare still use the oracle weights.

    \begin{table}[h!]

\caption{True Value is 0. Cor(Y1, Y2) = 0}
\centering
\begin{tabular}[t]{lrrrr}
\toprule
algorithm & bias & sd & tstat & pval\\
\midrule
oracle & 0.000 & 0.032 & -0.318 & 0.375\\
ipworacle & -0.001 & 0.079 & -0.475 & 0.317\\
proposed & 0.000 & 0.038 & 0.174 & 0.431\\
\bottomrule
\end{tabular}
\end{table}

    \begin{table}[h!]

\caption{True Value is 0. Cor(Y1, Y2) = 0.1}
\centering
\begin{tabular}[t]{lrrrr}
\toprule
algorithm & bias & sd & tstat & pval\\
\midrule
oracle & 0.000 & 0.031 & 0.394 & 0.347\\
ipworacle & 0.001 & 0.082 & 0.560 & 0.288\\
proposed & 0.008 & 0.037 & 9.204 & 0.000\\
\bottomrule
\end{tabular}
\end{table}

    \begin{table}[h!]

\caption{True Value is 0. Cor(Y1, Y2) = 0.5}
\centering
\begin{tabular}[t]{lrrrr}
\toprule
algorithm & bias & sd & tstat & pval\\
\midrule
oracle & 0.000 & 0.031 & 0.318 & 0.375\\
ipworacle & 0.001 & 0.093 & 0.683 & 0.247\\
proposed & 0.000 & 0.042 & 0.182 & 0.428\\
\bottomrule
\end{tabular}
\end{table}


    The previous version of this simulation used the Stage 2 probability $p_i
    \propto \logistic(y_i)$ where $y_i$ was the observed $Y$ value in Stage 1.
    However, under this setup, our method is biased. This is because

    \[ E\left[E\left[\frac{R_1 R_2}{\pi_{11}(X, Y_1)} Y_1 \mid X\right]\right] 
    \neq
    E\left[\frac{E[Y_1 \mid X]}{\pi_{11}(X, Y_1)} E[R_1 R_2 \mid X]\right] \]

    but

    \[ E\left[E\left[\frac{R_1 R_2}{\pi_{11}(X)} Y_1 \mid X\right]\right] 
    =
    E\left[\frac{E[Y_1 \mid X]}{\pi_{11}(X)} E[R_1 R_2 \mid X]\right].\]

    Thus, there is strong evidence that the proposed method is 
    doubly robust because it is robust to both misspecification in the outcome
    and response model.
\end{itemize}

\newpage

\section*{Missingness Mechanism}

\begin{itemize}
  \item First I am going to reproduce the proof of double robustness that we
    talked about during our last meeting. I think it is insightful for future
    comments:

    \begin{align*}
      E[\hat \theta_{eff} - \theta_n] 
      &= E\left[n^{-1} \sum_{i = 1}^n E[g_i \mid X_i] - g_i\right]\\
      &\quad+ E\left[n^{-1} \sum_{i = 1}^n \frac{R_{1i}}{\pi_{1+}(X_i)}
      (b_2(X_i, Y_{1i}) - E[g_i \mid X_i])\right]\\
      &\quad+ E\left[n^{-1} \sum_{i = 1}^n \frac{R_{2i}}{\pi_{2+}(X_i)}
      (a_2(X_i, Y_{2i}) - E[g_i \mid X_i])\right]\\
      &\quad+ E\left[n^{-1} \sum_{i = 1}^n \frac{R_{1i} R_{2i}}{\pi_{11}(X_i)} 
      (g_i - a_2(X_i, Y_{2i}) - b_2(X_i, Y_{1i}) + E[g_i \mid X_i])\right]\\
      &= n^{-1} \sum_{i = 1}^n (E[E[g_i \mid X_i]] - E[g_i])\\
      &\quad+ n^{-1} \sum_{i = 1}^n E\left[E\left[\frac{R_{1i}}{\pi_{1+}(X_i)} 
      (b_2(X_i, Y_{1i}) - E[g_i \mid X_i]) \mid X_i\right]\right] \\ 
      &\quad+ n^{-1} \sum_{i = 1}^n E\left[E\left[\frac{R_{2i}}{\pi_{2+}(X_i)} 
      (a_2(X_i, Y_{2i}) - E[g_i \mid X_i]) \mid X_i\right]\right] \\
      &\quad+ n^{-1} \sum_{i = 1}^n E\left[E\left[\frac{R_{1i}
      R_{2i}}{\pi_{11}(X_i)} (g_i - a_2(X_i, Y_{2i}) - b_2(X_i, Y_{1i}) + E[g_i
      \mid X_i]) \mid X_i \right]\right]\\
      \intertext{Since $R_{1i} \perp Y_{1i} \mid X_i$, $R_{2i} \perp Y_{2i} \mid
      X_i$, $(R_{1i}, R_{2i}) \perp (Y_{1i}, Y_{2i}) \mid X_i$ and $\pi_{1+},
      \pi_{2+}$ and $\pi_{11}$ are all free of $Y_{1i}$ and $Y_{2i}$.}
      &= n^{-1} \sum_{i = 1}^n E\left[\frac{R_{1i}}{\pi_{1+}(X_i)} E\left[
        (E[g_i \mid X_i, Y_{1i}] - E[g_i \mid X_i]) \mid X_i\right]\right] \\ 
      &\quad+ n^{-1} \sum_{i = 1}^n E\left[\frac{R_{2i}}{\pi_{2+}(X_i)} 
      E\left[E[g_i \mid X_i, Y_{2i}] - E[g_i \mid X_i] \mid X_i\right]\right] \\
      &\quad+ n^{-1} \sum_{i = 1}^n E\left[\frac{R_{1i}
      R_{2i}}{\pi_{11}(X_i)} E\left[(g_i - E[g_i \mid X_i, Y_{2i}] - 
      E[g_i \mid X_i, Y_{1i}]+ E[g_i \mid X_i]) \mid X_i \right]\right]\\
      \intertext{Since $E[E[g_i \mid X_i, Y_{ki}] \mid X_i] = E[g_i \mid X_i] =
      0$,}
      &= 0.
    \end{align*}

    Thus, if the outcome models are correctly specified $\hat \theta_{eff}$ is
    unbiased. If the response models are correctly specified it is easy to see
    that $\hat \theta_{eff}$ is also unbiased. This means that $\hat
    \theta_{eff}$ is doubly robust.

  \item However, one of the key steps is that \textit{all} the response
    models are free of $Y$. In a previous iteration of Simulation 4, we had
    adopted the framework of \cite{robins1997non} where we first to observe the
    first variable and see if we observe the second variable. In this case, the
    second step can depend on the result of the first step and this is what we
    did. However, this makes it the case that $\pi_11$ is a function of $X_i$
    and $Y_1$ and $Y_2$. In this case $\hat \theta_{eff}$ is not unbiased if the
    response model is misspecified (or even just estimated).

  \item If we modify Simulation 4, such that the second step of observed the
    second variable is proportional to $\logistic(y_k)$ then we get the same
    result as before:

    \begin{table}[h!]

\caption{True Value is 0. Cor(Y1, Y2) = 0}
\centering
\begin{tabular}[t]{lrrrr}
\toprule
algorithm & bias & sd & tstat & pval\\
\midrule
oracle & 0.000 & 0.032 & -0.318 & 0.375\\
ipworacle & -0.001 & 0.079 & -0.475 & 0.317\\
proposed & 0.002 & 0.037 & 2.851 & 0.002\\
\bottomrule
\end{tabular}
\end{table}

    \begin{table}[h!]

\caption{True Value is 0. Cor(Y1, Y2) = 0.1}
\centering
\begin{tabular}[t]{lrrrr}
\toprule
algorithm & bias & sd & tstat & pval\\
\midrule
oracle & 0.000 & 0.031 & 0.394 & 0.347\\
ipworacle & 0.001 & 0.082 & 0.560 & 0.288\\
proposed & 0.008 & 0.037 & 9.204 & 0.000\\
\bottomrule
\end{tabular}
\end{table}

    \begin{table}[h!]

\caption{True Value is 0. Cor(Y1, Y2) = 0.5}
\centering
\begin{tabular}[t]{lrrrr}
\toprule
algorithm & bias & sd & tstat & pval\\
\midrule
oracle & 0.000 & 0.031 & 0.318 & 0.375\\
ipworacle & 0.001 & 0.093 & 0.683 & 0.247\\
proposed & 0.017 & 0.039 & 19.062 & 0.000\\
\bottomrule
\end{tabular}
\end{table}

  
\end{itemize}

\newpage

% \section*{Estimating the Response Model}

\begin{itemize}
  \item In most (critically \textit{not} Simulation 4) of the simulations of the
    previous section, we used the oracle weights when estimating our proposed
    method. The reasoning for this was straightforward{--}we wanted to ensure
    that the response model was correctly specified and the best case scenario
    is to use the oracle weights which we did.
    
  \item This section focuses more on estimating these response weights. Instead
    of focusing on the proposed method, we will actually be working on
    estimation of the complete case IPW estimator. This model is less complex
    and will thus make it easier to understand where we are making mistakes.

  \item We use the following simulation study:
    \begin{align*}
      x_i &\stackrel{iid}{\sim} N(0, 1)\\
      \varepsilon_{1i} &\stackrel{iid}{\sim} N(0, 1)\\
      \varepsilon_{2i} &\stackrel{iid}{\sim} N(0, 1)\\
      y_{1i} &= x_i + \varepsilon_{1i} \\
      y_{2i} &= x_i + \varepsilon_{2i} \\
    \end{align*}

    To select a missingness pattern for each $i$, we have sequence: first, we
    select the first variable to observe (or neither), then we either select the
    second variable or we do not. In the first step, we select $R_{1i} = 1$ with
    probability $0.4$, $R_{2i} = 1$ with probability $0.4$ and neither variable
    with probability $0.2$. For the second step, we have the probability of
    observing the other variable be $\logistic(x_i)$. This yields the following:

    \begin{table}[h!]

\caption{True Value is 0. Cor(Y1, Y2) = 0}
\centering
\begin{tabular}[t]{lrrrr}
\toprule
algorithm & bias & sd & tstat & pval\\
\midrule
oracle & 0.000 & 0.032 & -0.318 & 0.375\\
ipw.or & 0.002 & 0.067 & 1.140 & 0.127\\
ipw.0 & 0.083 & 0.032 & 116.911 & 0.000\\
\bottomrule
\end{tabular}
\end{table}

    % Extension?: estimate IPW with \pi_{2|1} and \pi_{1|2}.

    The problem with this is that our estimate is biased. For one particular
    realization of the simulation, here is the distribution of the difference
    between the estimated and true probabilities of $\pi_{11}$:

    \includegraphics[width = 0.7\linewidth]{diffhist.png}
  
\end{itemize}

\newpage



\section*{Minimizing the Variance}

\begin{itemize}
  \item The goal of this section is to find optimal values of $b_2(X, Y_1)$ and
    $a_2(X, Y_2)$ such that the variance of $\hat \theta_{eff}$ is minimized.

  \item Recall:

    \begin{align*}
      \hat \theta_{eff} - \hat \theta_n 
      &= n^{-1} \sum_{i = 1}^n E[g_i \mid X_i] \left(1 - \frac{R_{1i}}{\pi_{1+}}
      - \frac{R_{2i}}{\pi_{2+}} + \frac{R_{1i}R_{2i}}{\pi_{11}}\right) \\
      &+ n^{-1} \sum_{i = 1}^n b_2(X_i, Y_{1i}) \left(\frac{R_{1i}}{\pi_{1+}} -
      \frac{R_{1i}R_{2i}}{\pi_{11}}\right)\\
      &+ n^{-1} \sum_{i = 1}^n a_2(X_i, Y_{2i}) \left(\frac{R_{2i}}{\pi_{2+}} -
      \frac{R_{1i}R_{2i}}{\pi_{11}}\right)\\
      &+ n^{-1} \sum_{i = 1}^n g_i \left(\frac{R_{1i}R_{2i}}{\pi_{11}} -
      1\right) \\
      &\equiv A + B + C + D.
    \end{align*}

  \item Notice that we will suppress the fact that response models are functions
    of $X$ (i.e. we write $\pi_{11}$ instead of $\pi_{11}(X)$).
  
  \item To compute the variance, we first solve for each covariance combination.
    Basically, all these computations rely on the following ideas. First, we
    assume that the response model is correctly specified. Consequently,
    $E[A] = E[B] = E[C] = E[D] = 0$ and things work out better. This helps when
    we take the covariance conditional on $X$ because the inner expectations are
    zero. The second key insight it to notice that $E[R_{j}^k] = E[R_j]$ for $j
    \in \{1, 2\}$ and $k \in \mathbb{N}$. This is because $R$ is a binary
    variable. Third, since we assume that the response models are correctly
    specified, we have $E[R_1 \mid X] = \pi_{1+}$, $E[R_2 \mid X] = \pi_{2+}$,
    and $E[R_1, R_2 \mid X] = \pi_{11}$.

    The overall approach to each of these computations is the following:
    (1) take conditional expectations with respect to $X$ (the $\Cov(E[\cdot])$
    term is zero), (2) expand the covariance to $E[XY] - E[X]E[Y]$ (the second
    term is also zero), (3) by the MAR assumption $g, a_2, b_2$ are independent
    of $R_1$ and $R_2$ and we can take the latter out of the expectation, (4)
    evaluate and simplify expressions involving $E[R]$.
  
    \begin{align*}
      \Cov(A, B) &= n^{-2} \sum_{i = 1}^n E\left[\Cov\left(E[g_i \mid X] \left(1
      - \frac{R_{1i}}{\pi_{1+}} - \frac{R_{2i}}{\pi_{2+}} +
      \frac{R_{1i}R_{2i}}{\pi_{11}}\right) \mid X_i,\right. \right.\\
      &\qquad \qquad \qquad \qquad b_2(X_i, Y_{1i}) \left. \left.
      \left(\frac{R_{1i}}{\pi_{1+}} - \frac{R_{1i}R_{2i}}{\pi_{11}}\right) \mid
      X_i\right)\right]\\
      &= n^{-1} E\left[E\left[E[g \mid X] \left(1 - \frac{R_{1i}}{\pi_{1+}} -
      \frac{R_{2i}}{\pi_{2+}} + \frac{R_{1i}R_{2i}}{\pi_{11}}\right)
      b_2(X_i, Y_{1i}) \left(\frac{R_{1i}}{\pi_{1+}} -
      \frac{R_{1i}R_{2i}}{\pi_{11}}\right) \mid X \right]\right] \\
      &= n^{-1} E\left[E[g \mid X] E[b_2(X, Y_1) \mid X]
      \left(\frac{1}{\pi_{1+}} + \frac{1}{\pi_{2+}} - \frac{1}{\pi_{11}} -
      \frac{\pi_{11}}{\pi_{1+} \pi_{2+}}\right)\right].
    \end{align*}

    By symmetry,
    \begin{align*}
      \Cov(A, C) 
      &= n^{-1} E\left[E[g \mid X] E[a_2(X, Y_2) \mid X]
      \left(\frac{1}{\pi_{1+}} + \frac{1}{\pi_{2+}} - \frac{1}{\pi_{11}} -
      \frac{\pi_{11}}{\pi_{1+} \pi_{2+}}\right)\right].
    \end{align*}

    \begin{align*}
      \Cov(A, D) &= n^{-1} E\left[E\left[E[g \mid X] \left(1 -
      \frac{R_{1i}}{\pi_{1+}} - \frac{R_{2i}}{\pi_{2+}} +
      \frac{R_{1i}R_{2i}}{\pi_{11}}\right) g \left(\frac{R_1 R_2}{\pi_{11}} -
      1\right) \mid X\right] \right]\\
      &= n^{-1} E\left[E[g \mid X]^2 \left(\frac{-1}{\pi_{1+}} -
      \frac{1}{\pi_{2+}} + 2\right)\right].
    \end{align*}

    \begin{align*}
      \Cov(B, C) &= n^{-1} E\left[E[b_2(X, Y_1) \mid X] E[a_2(X, Y_2) \mid X]
      E\left[\left(\frac{R_1}{\pi_{1+}} - \frac{R_1 R_2}{\pi_{11}}\right)
      \left(\frac{R_2}{\pi_{2+}} - \frac{R_1 R_2}{\pi_{11}}\right) \mid
      X\right]\right]\\
      &= n^{-1} E\left[E[b_2(X, Y_1) \mid X] E[a_2(X, Y_2) \mid X]
      \left(\frac{\pi_{11}}{\pi_{1+} \pi_{2+}} - \frac{1}{\pi_{1+}} -
      \frac{1}{\pi_{2+}} + \frac{1}{\pi_{11}}\right)\right].
    \end{align*}

    \begin{align*}
      \Cov(B, D) &= n^{-1} E\left[ E[b_2(X, Y_1) \mid X] E[g \mid X]
      E\left[\left(\frac{R_1}{\pi_{1+}} - \frac{R_1 R_2}{\pi_{11}}\right)
      \left(\frac{R_1 R_2}{\pi_{11}} - 1\right) \mid X\right] \right]\\
      &= n^{-1} E\left[ E[b_2(X, Y_1) \mid X] E[g \mid X]
      \left(\frac{1}{\pi_{1+}} - \frac{1}{\pi_{11}}\right)\right].
    \end{align*}

    By symmetry,
    \begin{align*}
      \Cov(C, D) 
      &= n^{-1} E\left[ E[a_2(X, Y_2) \mid X] E[g \mid X]
      \left(\frac{1}{\pi_{2+}} - \frac{1}{\pi_{11}}\right)\right].
    \end{align*}

    We also compute the variance terms,

    \begin{align*}
      \Cov(A, A) &= n^{-1} E\left[ E\left[ E[g \mid X]^2 \left(1
      - \frac{R_{1i}}{\pi_{1+}} - \frac{R_{2i}}{\pi_{2+}} +
      \frac{R_{1i}R_{2i}}{\pi_{11}}\right)^2 \mid X\right] \right]\\
      &= n^{-1} E \left[ E[g \mid X]^2 \left(-1 + \frac{2 \pi_{11}}{\pi_{1+}
      \pi_{2+}} - \frac{1}{\pi_{1+}} - \frac{1}{\pi_{2+}} +
      \frac{1}{\pi_{11}}\right)\right].
    \end{align*}

    \begin{align*}
      \Cov(B, B) &= n^{-1} E\left[ E\left[b(X, Y_1)^2 \left(\frac{R_1}{\pi_{1+}}
      - \frac{R_1 R_2}{\pi_{11}}\right)^2 \mid X\right] \right]\\
      &= n^{-1} E\left[ E[b_2(X, Y_1)^2 \mid X] \left(\frac{-1}{\pi_{1+}} +
      \frac{1}{\pi_{11}}\right)\right].
    \end{align*}
    
    \begin{align*}
      \Cov(C, C) &= n^{-1} E\left[ E[a_2(X, Y_2)^2 \mid X]
      \left(\frac{-1}{\pi_{2+}} + \frac{1}{\pi_{11}}\right) \right].
    \end{align*}

    \begin{align*}
      \Cov(D, D) &= n^{-1} E\left[ E\left[g_i^2 \left(\frac{R_1 R_2}{\pi_{11}} -
      1\right)^2 \mid X\right] \right]\\
      &= n^{-1} E\left[ E[g^2 \mid X] \left(\frac{1}{\pi_{11}} -
      1\right)\right].
    \end{align*}

    This means that
    \begin{align*}
      \Var(\hat \theta_{eff} - \hat \theta_n) 
      &= \Cov(A, A) + 2 \Cov(A, B) + 2 \Cov(A, C) + 2 \Cov(A, D) + \Cov(B, B) \\
      &+ 2 \Cov(B, C) + 2 \Cov(B, D) + \Cov(C, C) + 2 \Cov(C, D) + \Cov(D, D) \\
      &= n^{-1} E \left[ E[g \mid X]^2 \left(-1 + \frac{2 \pi_{11}}{\pi_{1+}
      \pi_{2+}} - \frac{1}{\pi_{1+}} - \frac{1}{\pi_{2+}} +
      \frac{1}{\pi_{11}}\right)\right] \\
      &+ 2 n^{-1} E\left[E[g \mid X] E[b_2(X, Y_1) \mid X]
      \left(\frac{1}{\pi_{1+}} + \frac{1}{\pi_{2+}} - \frac{1}{\pi_{11}} -
      \frac{\pi_{11}}{\pi_{1+} \pi_{2+}}\right)\right]\\
      &+ 2 n^{-1} E\left[E[g \mid X] E[a_2(X, Y_2) \mid X]
      \left(\frac{1}{\pi_{1+}} + \frac{1}{\pi_{2+}} - \frac{1}{\pi_{11}} -
      \frac{\pi_{11}}{\pi_{1+} \pi_{2+}}\right)\right]\\
      &+ 2 n^{-1} E\left[E[g \mid X]^2 \left(\frac{-1}{\pi_{1+}} -
      \frac{1}{\pi_{2+}} + 2\right)\right]\\
      &+ n^{-1} E\left[ E[b_2(X, Y_1)^2 \mid X] \left(\frac{-1}{\pi_{1+}} +
      \frac{1}{\pi_{11}}\right)\right]\\
      &+ 2 n^{-1} E\left[E[b_2(X, Y_1) \mid X] E[a_2(X, Y_2) \mid X]
      \left(\frac{\pi_{11}}{\pi_{1+} \pi_{2+}} - \frac{1}{\pi_{1+}} -
      \frac{1}{\pi_{2+}} + \frac{1}{\pi_{11}}\right)\right]\\
      &+ 2 n^{-1} E\left[ E[b_2(X, Y_1) \mid X] E[g \mid X]
      \left(\frac{1}{\pi_{1+}} - \frac{1}{\pi_{11}}\right)\right]\\
      &+  n^{-1} E\left[ E[a_2(X, Y_2)^2 \mid X]
      \left(\frac{-1}{\pi_{2+}} + \frac{1}{\pi_{11}}\right) \right]\\
      &+ 2 n^{-1} E\left[ E[a_2(X, Y_2) \mid X] E[g \mid X]
      \left(\frac{1}{\pi_{2+}} - \frac{1}{\pi_{11}}\right)\right]\\
      &+ n^{-1} E\left[ E[g^2 \mid X] \left(\frac{1}{\pi_{11}} -
      1\right)\right].
    \end{align*}

    Differentiating yields:

    \begin{align*}
      \frac{\partial}{\partial a_2} \Var(\hat \theta_{eff} - \hat \theta_n)
      &= E\left[E[g \mid X] \left(\frac{1}{\pi_{1+}} + \frac{2}{\pi_{2+}} -
      \frac{2}{\pi_{11}} - \frac{\pi_{11}}{\pi_{1+} \pi_{2+}}\right)\right] \\
      &+ E\left[E[b_2(X, Y_1) \mid X] \left(\frac{\pi_{11}}{\pi_{1+} \pi_{2+}} -
      \frac{1}{\pi_{1+}} - \frac{1}{\pi_{2+}} +
      \frac{1}{\pi_{11}}\right)\right]\\
      &+ E\left[ E[a_2(X, Y_2) \mid X] \left(\frac{-1}{\pi_{2+}} +
      \frac{1}{\pi_{11}}\right)\right]\\
      &\equiv 0, \text{ and }\\
      \frac{\partial}{\partial b_2} \Var(\hat \theta_{eff} - \hat \theta_n)
      &= E\left[E[g \mid X] \left(\frac{2}{\pi_{1+}} + \frac{1}{\pi_{2+}} -
      \frac{2}{\pi_{11}} - \frac{\pi_{11}}{\pi_{1+}\pi_{2+}}\right)\right]\\
      &+ E\left[E[a_2(X, Y_2) \mid X] \left(\frac{\pi_{11}}{\pi_{1+} \pi_{2+}} -
      \frac{1}{\pi_{1+}} - \frac{1}{\pi_{2+}} +
      \frac{1}{\pi_{11}}\right)\right]\\
      &+ E\left[ E[b_2(X, Y_2) \mid X] \left(\frac{-1}{\pi_{1+}} +
      \frac{1}{\pi_{11}}\right)\right]\\
      &\equiv 0. 
    \end{align*}


    Substitution shows that these constraints are equivalent to:

    \begin{align*}
      E&\left[E[g \mid X] \left(\frac{-1}{\pi_{1+}} +
    \frac{1}{\pi_{2+}}\right)\right] + E\left[ E[b_2(X, Y_1) \mid X]
    \left(\frac{\pi_{11}}{\pi_{1+} \pi_{2+}} -\frac{1}{\pi_{2+}}\right)\right]\\
      &- E\left[E[a_2(X, Y_2) \mid X] \left(\frac{\pi_{11}}{\pi_{1+} \pi_{2+}} - 
    \frac{1}{\pi_{1+}}\right)\right] \equiv 0
    \end{align*}

    where is the same as,

    \begin{align*}
      E&\left[ E[b_2(X, Y_1) \mid X] \left(\frac{\pi_{11}}{\pi_{1+} \pi_{2+}} -
    \frac{1}{\pi_{2+}}\right) \right] + E\left[E[g \mid X]
      \left(\frac{1}{\pi_{2+}}\right)\right] \\
      &=
    E\left[ E[a_2(X, Y_2) \mid X] \left(\frac{\pi_{11}}{\pi_{1+} \pi_{2+}} -
    \frac{1}{\pi_{1+}}\right) \right] + E\left[E[g \mid X]
    \left(\frac{1}{\pi_{1+}}\right)\right].
    \end{align*}

  \item These constraints can be satisfied (this is sufficient but maybe not
    necessary) if 

    \begin{align*}
      E&\left[(E[b_2(X, Y_1) \mid X] - E[a_2(X, Y_2) \mid X])
      \left(\frac{\pi_{11}}{\pi_{1+} \pi_{2+}}\right) \right] = 0\\
      E&\left[\left(\frac{1}{\pi_{1+}} - \frac{1}{\pi_{2+}}\right)(E[a_2(X, Y_2)
      \mid X] + E[b_2(X, Y_1) \mid X] - 2E[g \mid X])\right] = 0.
    \end{align*}

\end{itemize}

%\textbf{Questions for Dr. Kim}
%\begin{itemize}
%  \item Where do we go from here?
%  \item This is a functional expectation that I need to calibrate. Should I use
%    some sort of basis spline to try to estimate these expectations and then set
%    them equal to zero? I don't think that I have worked with this kind of
%    constraint before.
%\end{itemize}

%\section*{Proposal: Full Nonmonotone Estimator}

\begin{itemize}

  \item When constructing
    estimates of the response model, we estimate $\pi_{11}$ using
    $\pi_{A_2 = 1|A_1 = 1}$ and we never include information about $\pi_{A_1 =
    1|A_2 = 1}$. I would like to create a full estimator where we are able to
    use this information.

  \item The current estimator is the following (See Slide 24 of Non-monotone
    Presentation):

    \begin{align*}
      \hat \theta_{eff} &= n^{-1} \sum_{i = 1}^n E[g_i \mid X_i] \\
      &+ n^{-1} \sum_{i = 1}^n \frac{R_{1i}}{\pi_{1+}(X_i)} (b_2(X_i, Y_{1i}) - 
      E[g_i \mid X_i]) \\
      &+ n^{-1} \sum_{i = 1}^n \frac{R_{2i}}{\pi_{2+}(X_i)} (a_2(X_i, Y_{2i}) -
      E[g_i \mid X_i]) \\
      &+ n^{-1} \sum_{i = 1}^n \frac{R_{1i} R_{2i}}{\pi_{11}(X_1, Y_{1i})}(g_i 
      - b_2(X_i, Y_{1i}) - a_2(X_i, Y_{2i}) + E[g_i \mid X_i])
    \end{align*}

  \item The problem with this is that we assume $\pi_{11}(X, Y_{1}) =
    \pi_{2|1} (X, Y_{1}) \pi_{1+}(X)$ when we could have $\pi_{11}(X, Y_2) =
    \pi_{1|2} (X, Y_2) \pi_{2+}(X)$. In other words, the previous result
    implicitly assumes that $\pi_{11}$ is a function of $X$ and $Y_1$ when it
    could be a function of $X$ and $Y_2$ as well. 

  \item I think that we could use the
    following (small modification) instead,

    \[\pi_{11}(X, Y_1, Y_2) = \alpha \pi_{1|2} \pi_{2+} + (1 - \alpha) \pi_{2|1}
    \pi_{1+}\]

    for some $\alpha \in [0, 1]$.
    
  \item The reasoning behind this is two-fold. First, it allows us to think of
    the nonmonotone missingness case as a linear combination of two monotone
    cases, which I think is useful. Second, we can make explicit our choice of
    $\alpha$, which in the existing estimator is simply $\alpha = 0$.

  \item The problem with adding another parameter $\alpha$ is that it makes the
    overall model unidentifiable. We cannot estimate $\alpha$ and the
    conditional and marginal distributions that we have without additional
    assumptions. So for now, I think that we should just assume that $\alpha$ is
    known and then we apply this method to a data integration problem we can
    review what $\alpha$ makes sense or figure out an additional assumption that
    can help us estimate $\alpha$ (for example if there is a variable correlated
    with $1|2$ versus $2|1$).

  \item This is not just a pure Bayes' rule.
    Formally, let $L = (X, Y_1, Y_2)$. By Bayes' rule we have

    \begin{align*}
      \Pr(R_1 &= 1 \mid R_2 = 1, L) \Pr(R_2 = 1 \mid L) \\
      &= \Pr(R_1 = 1, R_2 = 1 \mid L) \\
      & = \Pr(R_2 = 1 \mid R_1 = 1, L) \Pr(R_1 = 1 \mid L).
    \end{align*}

    Yet to estimate this model, we need to assume something like the following:
    \begin{align*}
      \Pr(R_1 &= 1 \mid R_2 = 1, X, Y_2) \Pr(R_2 = 1 \mid X, Y_2)\\
      &= \Pr(R_1 = 1, R_2 = 1 \mid L) \\
      &= \Pr(R_2 = 1 \mid R_1 = 1, X, Y_1) \Pr(R_1 = 1 \mid X, Y_1)
    \end{align*}

    in which case the two sides are clearly different.

    \item Unfortunately, early simulation results are not promising because we
      cannot distinguish points that should have $\pi_{11}$ conditional on $Y_1$
      and points that should be conditional on $Y_2$.

      \begin{table}[h!]

\caption{True Value is 0. Cor(Y1, Y2) = 0}
\centering
\begin{tabular}[t]{lrrrr}
\toprule
algorithm & bias & sd & tstat & pval\\
\midrule
oracle & 0.001 & 0.032 & 1.269 & 0.102\\
ipworacle & 0.000 & 0.078 & -0.141 & 0.444\\
prop.or & 0.002 & 0.038 & 2.201 & 0.014\\
prop.0 & 0.004 & 0.037 & 5.037 & 0.000\\
prop.half & 0.008 & 0.037 & 9.056 & 0.000\\
\addlinespace
prop.1 & 0.010 & 0.038 & 12.058 & 0.000\\
\bottomrule
\end{tabular}
\end{table}


    \item Dr. Kim, please let me know what you think about this idea and if it
      is worth pursuing more.

\end{itemize}


\newpage


\section*{Comparison with Calibration Estimator}

% Outline
% * [X] Regression estimator comparison with calibration in monotone case
% * [X] Equivalent calibration in nonmonotone case

\subsection*{Monotone Case}

\begin{itemize}
  \item In the monotone case the efficient estimator is 

    \begin{align*}
      \hat \theta_{eff} &= n^{-1} \sum_{i = 1}^n E[g_i \mid X_i] \\
      &+ n^{-1} \sum_{i = 1}^n \frac{R_{1i}}{\pi_{1+}(X_i)}(
        E[g_i \mid X_i, Y_{1i}] - E[g_i \mid X_i]) \\
      &+ n^{-1} \sum_{i = 1}^n \frac{R_{1i} R_{2i}}{\pi_{11}(X_i)} (
      E[g_i \mid X_i, Y_{1i}, Y_{2i}] - E[g_i \mid X_i, Y_{1i}]). 
    \end{align*}

\item HT estimator of $\theta=E(Y_2)$: 
$$ \hat{\theta}_{\rm HT} = \frac{1}{n} \sum_{i=1}^n \frac{R_{1i} R_{2i}}{ \pi_{11}(X_i) } y_{2i}  $$
\item The three-phase regression estimator of $\theta$:
\begin{eqnarray*} 
\hat{\theta}_{\rm reg} &=& \frac{1}{n} \sum_{i \in A_2}\frac{1}{ \pi_{2i}  }\left\{ y_{i}  - \hat{E} ( Y \mid x_i, z_{i} ) \right\} \\
&+& \frac{1}{n} \sum_{i \in A_1}\frac{1}{ \pi_{1i}  }\left\{ \hat{E} ( Y \mid x_i, z_{i} ) - \hat{E} ( Y \mid x_i ) \right\} + \frac{1}{n} \sum_{i \in U} \hat{E} ( Y \mid x_i ) \\
&=& \bar{x}_0' \hat{\beta} + \left( \bar{x}_1' \hat{\gamma}_x + \bar{z}_1' \hat{\gamma}_z - \bar{x}_1' \hat{\beta} \right) + \{ \bar{y}_2- (\bar{x}_2' \hat{\gamma}_x + \bar{z}_2' \hat{\gamma}_z ) \} \\
&=& \bar{y}_2 + \{ \bar{x}_1' \hat{\gamma}_x + \bar{z}_1' \hat{\gamma}_z  -  (\bar{x}_2' \hat{\gamma}_x + \bar{z}_2' \hat{\gamma}_z ) \} +\left(  \bar{x}_0' \hat{\beta} - \bar{x}_1' \hat{\beta} \right) \end{eqnarray*} 
\item We can view the above three-phase regression estimator as a projection (= mass imputation) estimator of Kim and Rao (2012, Biometrika). 
  \item This should be very similar to the following calibration estimator, for
    $\sum_{i = 1}^n w_i y_{2i}$

    \begin{align*}
      \argmin_w \sum_{i = 1}^n w_i^2& \text{ such that }\\
      \sum_{i = 1}^n x_i &= \sum_{i = 1}^n R_{1i} w_{1i} x_i \\
      \sum_{i = 1}^n w_{1i} (x_i, y_{1i}) &= \sum_{i = 1}^n R_{1i} R_{2i} w_{2i}
      (x_i, y_{1i}) \\
    \end{align*}
  
  \item The reason that these should be the same is because they are similar in
    relationship to a calibration and regression estimator which are exactly the
    same.
    
  \item To test the idea that the monotone regression estimator is similar to
    the calibration estimator we run several simulation studies. In the monotone
    case data is generating in the following steps:

    \begin{enumerate}
      \item The variables $X$, $Y_1$, and $Y_2$ are simulated from the following
        distributions:
        \begin{align*}
          X_i &\stackrel{iid}{\sim} N(0, 1) \\
          Y_{1i} &\stackrel{iid}{\sim} N(0, 1) \\
          Y_{2i} &\stackrel{iid}{\sim} N(\theta, 1).
        \end{align*}

      \item After the variables have been simulated, we see which variables are
        observed. We always observe $X_i$. We observed $Y_1$ with
        probability $p_{1i} \propto \logistic(x_i)$. If $Y_{1i}$ is observed,
        then we observe $Y_{2i}$ with probability $p_{2i} \propto
        \logistic(y_{1i})$. If $Y_{1i}$ is not observed, we do not observe
        $Y_{2i}$.
    \end{enumerate}

  \item The goal of this simulation study is the estimate $\theta = E[Y_2]$. We
    use the previous monotone data generating process with different true values
    of $\theta$ and compute the bias, standard deviation, T-statistic and
    p-value. (The T-statistic and p-value test if the estimated value of $\hat
    \theta$ is significantly different from the true value of $\theta$.)

    \begin{table}

\caption{True Value is -5}
\centering
\begin{tabular}[t]{lrrrr}
\toprule
algorithm & bias & sd & tstat & pval\\
\midrule
oracle & 0.001 & 0.032 & 1.295 & 0.098\\
ipworacle & -0.006 & 0.406 & -0.434 & 0.332\\
ipwest & -0.009 & 0.187 & -1.575 & 0.058\\
semi & -0.001 & 0.077 & -0.251 & 0.401\\
calib & 0.000 & 0.076 & 0.070 & 0.472\\
\bottomrule
\end{tabular}
\end{table}
    \begin{table}

\caption{True Value is 0}
\centering
\begin{tabular}[t]{lrrrr}
\toprule
algorithm & bias & sd & tstat & pval\\
\midrule
oracle & 0.000 & 0.032 & 0.288 & 0.387\\
ipworacle & -0.002 & 0.081 & -0.619 & 0.268\\
ipwest & -0.002 & 0.083 & -0.686 & 0.246\\
semi & -0.002 & 0.074 & -0.705 & 0.240\\
calib & -0.001 & 0.074 & -0.626 & 0.266\\
\bottomrule
\end{tabular}
\end{table}
    \begin{table}[h!]

\caption{True Value is 5}
\centering
\begin{tabular}[t]{lrrrr}
\toprule
algorithm & bias & sd & tstat & pval\\
\midrule
oracle & 0.000 & 0.032 & -0.446 & 0.328\\
ipworacle & 0.004 & 0.392 & 0.298 & 0.383\\
ipwest & 0.000 & 0.185 & 0.079 & 0.468\\
semi & -0.004 & 0.077 & -1.649 & 0.050\\
calib & -0.004 & 0.075 & -1.702 & 0.045\\
\bottomrule
\end{tabular}
\end{table}


\end{itemize}

\newpage

\subsection*{Nonmonotone Case}

\begin{itemize}
  \item Similar to the monotone case, we have an idea of the efficient
    estimator. Now we want to show that it is similar to a calibration equation.
    Unlike the monotone case where $R_{1i} = 0$ implies $R_{2i} = 0$, the
    nonmonotone case does not have this relationship. Instead we believe that we
    have the following calibration equations:

    \begin{align*}
      \sum_{i = 1}^n E[g_i \mid X_i] &= \sum_{i = 1}^n R_{1i} w_{1i} E[g_i \mid
      X_i]\\
      \sum_{i = 1}^n E[g_i \mid X_i] &= \sum_{i = 1}^n R_{2i} w_{2i} E[g_i \mid
      X_i]\\
      \sum_{i = 1}^n R_{1i} w_{1i} E[g_i \mid X_i, Y_{1i}] &= \sum_{i = 1}^n
      R_{1i} R_{2i} w_{ci} E[g_i \mid X_i, Y_{1i}]\\
      \sum_{i = 1}^n R_{2i} w_{2i} E[g_i \mid X_i, Y_{1i}] &= \sum_{i = 1}^n
      R_{1i} R_{2i} w_{ci} E[g_i \mid X_i, Y_{2i}]\\
      \sum_{i = 1}^n E[g_i \mid X_i] &= \sum_{i = 1}^n R_{1i} R_{2i} w_{ci}
      E[g_i \mid X_i].
    \end{align*}
  
  \item We still have the same goal of the simulation study: estimate $\theta =
    E[Y_2]$. We use the previous nonmonotone data generating process with
    different true values of $\theta$ to estimate the bias, standard deviation,
    T-statistic, and p-value. For clarity here is a reminder of the simulation
    setup.

    \begin{enumerate}
      \item Generate $X_i$, $\varepsilon_{1i}$, and $\varepsilon_{2i}$ from the
        following distributions:

        \begin{align*}
          x_i &\stackrel{iid}{\sim} N(0, 1)\\
          \varepsilon_{1i} &\stackrel{iid}{\sim} N(0, 1)\\
          \varepsilon_{2i} &\stackrel{iid}{\sim} N(\theta, 1)\\
        \end{align*}

        Then we have
        \[y_{1i} = x_i + \varepsilon_{1i} \text{ and } y_{2i} = x_i +
        \varepsilon_{2i}.\]

      \item Then we have to select the variables to observe. We always observe
        $X_i$. Then we choose to either observe $Y_1$ with probability $0.4$,
        $Y_2$ with probability $0.4$ or neither with probability $0.2$.

      \item If neither then $R_{1i} = 0$ and $R_{2i} = 0$. If we observe $Y_1$
        then $R_1 = 1$ and if we observe $Y_2$ then $R_2 = 1$.

      \item If we observe either $Y_1$ or $Y_2$ then with probability $p \propto
        \logistic(Y_k)$ where $Y_k$ is the observed $Y$ variable we choose to
        observe the other $Y$ variable.

      \item If the other $Y$ variable is observed then the corresponding $R_k =
        1$. Otherwise, $R_k = 0$.
    \end{enumerate}

  \item For this simulation setup, we estimate $\theta = E[Y_2]$. Like the
    previous nonmonotone simulations, we compare this calibration estimator to
    the oracle estimator which uses the average value of $Y_2$ if $R_2 = 0$ or
    $R_2 = 1$, an IPW estimator with the correct weights, and the proposed
    regression estimator. These are currently run with a sample size of $n =
    1000$ with the number of Monte Carlo simulations of $B = 1000$.

    \begin{table}[h!]

\caption{True Value is -5. Cor(Y1, Y2) = 0}
\centering
\begin{tabular}[t]{lrrrr}
\toprule
algorithm & bias & sd & tstat & pval\\
\midrule
oracle & -0.001 & 0.045 & -0.953 & 0.170\\
ipworacle & 0.011 & 0.552 & 0.627 & 0.266\\
proposed & -0.002 & 0.055 & -0.873 & 0.191\\
calib & -0.002 & 0.054 & -1.312 & 0.095\\
\bottomrule
\end{tabular}
\end{table}

    \begin{table}[h!]

\caption{True Value is 0. Cor(Y1, Y2) = 0}
\centering
\begin{tabular}[t]{lrrrr}
\toprule
algorithm & bias & sd & tstat & pval\\
\midrule
oracle & -0.001 & 0.044 & -0.945 & 0.173\\
ipworacle & 0.001 & 0.112 & 0.178 & 0.429\\
proposed & -0.001 & 0.053 & -0.363 & 0.358\\
reg2p & 0.004 & 0.069 & 1.809 & 0.035\\
reg3p & 0.005 & 0.069 & 2.372 & 0.009\\
\addlinespace
calib & -0.001 & 0.052 & -0.508 & 0.306\\
\bottomrule
\end{tabular}
\end{table}

    \begin{table}[h!]

\caption{True Value is 5. Cor(Y1, Y2) = 0}
\centering
\begin{tabular}[t]{lrrrr}
\toprule
algorithm & bias & sd & tstat & pval\\
\midrule
oracle & -0.002 & 0.045 & -1.358 & 0.087\\
ipworacle & -0.002 & 0.141 & -0.409 & 0.341\\
proposed & -0.002 & 0.051 & -1.531 & 0.063\\
reg2p & -0.003 & 0.052 & -1.589 & 0.056\\
reg3p & -0.003 & 0.052 & -1.565 & 0.059\\
\addlinespace
calib & -0.002 & 0.051 & -1.401 & 0.081\\
\bottomrule
\end{tabular}
\end{table}


\end{itemize}

\newpage




\section*{Efficiency of Proposed Estimator}

One of our main goals is to show that we have an efficient estimator. We
want to ensure that our estimator is more efficient (lower MSE or really zero
bias and a smaller variance) than other competing estimators. We have already
demonstrated superior performance compared to the IPW estimator with known
weights. Now, we want to compare our estimator with two-phase and three-phase 
regression estimators.

\subsection*{Two and Three Phase Regression Estimators}

\begin{itemize}
  \item The simulation setup that we have been running generates monotone and
    nonmonotone missing patterns because we always observe $X$, yet we observe
    $Y_1$ and $Y_2$ with missingness. This creates the need to extend the
    traditional two-phase estimator to become a three-phase estimator. The
    two-phase regression estimator is really just a regression estimator with
    the ``finite population'' being Phase 1 of the sample. Thus, a valid
    two-phase estimator for $\theta = E[Y_2]$ is

    \[\hat \theta = \bar y_2 + (\bar x_1 - \bar x_2) \hat \beta\]

    where $\bar y_2 = n_2^{-1} \sum_{i \in U} I(i \in A_2) y_2$ and $\bar x_k =
    n_k^{-1} \sum_{i \in U} I(i \in A_k) x_i$ where $n_k$ is the number of
    elements in $A_k$ where $A_k$ is the Phase $k$ sample. In this case, $\hat
    \beta$ solves the following equation for $\beta_0$ and $\beta_1$,

    \[\sum_{i \in A_2} (y_{2i} - \beta_0 - \beta_1 x_i)^2 = 0.\]

  \item Notice that the previous construction of the two-phase estimator ignored
    the variable $Y_1$. We can incorporate this into the model using a 
    three-phase estimator. From \cite{fuller2009sampling}, the three-phase
    estimator is 

    \begin{align*}
      \bar y_{2, 2p} &= \bar y_2 + (\bar x_0 - \bar x_{2}) \hat \beta_1 + 
      (\bar y_{1, reg} - \bar y_{1, 2p}) \hat \beta_2
    \end{align*}

    where 
    \begin{align*}
      \bar x_0 &= n^{-1} \sum_{i \in U} x_i \\
      \bar x_1 &= \left(\sum_{i \in A_1} \pi_{1i}^{-1}\right)^{-1} 
                  \sum_{i \in A_1} \pi_{1i}^{-1} x_i\\
      \bar x_2 &= \left(\sum_{i \in A_2} \pi_{2i}^{-1}\right)^{-1} 
                  \sum_{i \in A_2} \pi_{2i}^{-1} x_i\\
      \bar y_{1, reg} &= \bar y_{1, 1p} + (\bar x_0 - \bar x_1)\hat \beta_{1p}\\
      \bar y_{1, 1p} &= \left(\sum_{i \in A_1} \pi_{1i}^{-1}\right)^{-1} 
                        \sum_{i \in A_1} \pi_{1i}^{-1} y_{1i}\\
      \bar y_{1, 2p} &= \left(\sum_{i \in A_2} \pi_{2i}^{-1}\right)^{-1} 
                        \sum_{i \in A_2} \pi_{2i}^{-1} y_{1i}\\
      \hat \beta_{1p} &= \left(\sum_{i \in A_1} (x_i - \bar x_1)^2
      \pi_{1i}^{-1}\right)^{-1} \sum_{i \in A_1} (x_i - \bar x_1)\pi_{1i}^{-1}
      (y_{1i} - \bar y_{1, 1p})\\
      \hat \beta_1 &= \left(\sum_{i \in A_2} (x_i - \bar x_2, y_{1i} - \bar
      y_{1, 2p})' \pi_{2i}^{-1} (x_i - \bar x_2, y_{1i} - \bar y_{1,
      2p})\right)^{-1} \sum_{i \in A_2} (x_i - \bar x_2) \pi_{2i}^{-1} (y_{2i} -
      \bar y_2)\\
      \hat \beta_2 &= \left(\sum_{i \in A_2} (x_i - \bar x_2, y_{1i} - \bar
      y_{1, 2p})' \pi_{2i}^{-1} (x_i - \bar x_2, y_{1i} - \bar y_{1,
      2p})\right)^{-1} \sum_{i \in A_2} (y_{1i} - \bar y_{1, 2p}) \pi_{2i}^{-1} 
      (y_{2i} - \bar y_2)\\
    \end{align*}

  \item Notice that the three-phase estimator implicitly assumes a monotone 
    missingness model. It ignores the values of $y_2$ that have an unobserved
    $y_1$.

\end{itemize}

\subsection*{Monotone Results}

\begin{itemize}
  \item First, we test the two-phase and three-phase regression estimators with
    a monotone simulation. Like the previous monotone simulations,
    we first generate data from the following distributions:
    \begin{align*}
        X_i &\stackrel{iid}{\sim} N(0, 1) \\
        Y_{1i} &\stackrel{iid}{\sim} N(0, 1)\\
        Y_{2i} &\stackrel{iid}{\sim} N(\theta, 1)
    \end{align*}

    Then, we create the probabilities $p_1 = \logistic(x_i)$ and 
    $p_{12} = \logistic(y_{1i})$. Including the calibration estimator from the
    previous section (note, these are the same as the previous monotone tables)
    this yields,

    \begin{table}

\caption{True Value is -5}
\centering
\begin{tabular}[t]{lrrrr}
\toprule
algorithm & bias & sd & tstat & pval\\
\midrule
oracle & 0.001 & 0.032 & 1.295 & 0.098\\
ipworacle & -0.006 & 0.406 & -0.434 & 0.332\\
ipwest & -0.009 & 0.187 & -1.575 & 0.058\\
semi & -0.001 & 0.077 & -0.251 & 0.401\\
calib & 0.000 & 0.076 & 0.070 & 0.472\\
\bottomrule
\end{tabular}
\end{table}
    \begin{table}

\caption{True Value is 0}
\centering
\begin{tabular}[t]{lrrrr}
\toprule
algorithm & bias & sd & tstat & pval\\
\midrule
oracle & 0.000 & 0.032 & 0.288 & 0.387\\
ipworacle & -0.002 & 0.081 & -0.619 & 0.268\\
ipwest & -0.002 & 0.083 & -0.686 & 0.246\\
semi & -0.002 & 0.074 & -0.705 & 0.240\\
calib & -0.001 & 0.074 & -0.626 & 0.266\\
\bottomrule
\end{tabular}
\end{table}
    \begin{table}[h!]

\caption{True Value is 5}
\centering
\begin{tabular}[t]{lrrrr}
\toprule
algorithm & bias & sd & tstat & pval\\
\midrule
oracle & 0.000 & 0.032 & -0.446 & 0.328\\
ipworacle & 0.004 & 0.392 & 0.298 & 0.383\\
ipwest & 0.000 & 0.185 & 0.079 & 0.468\\
semi & -0.004 & 0.077 & -1.649 & 0.050\\
calib & -0.004 & 0.075 & -1.702 & 0.045\\
\bottomrule
\end{tabular}
\end{table}


  \item The proposed semiparametric estimator and the calibration estimator
    seem to both outperform the regression estimators. With smaller bias and
    smaller variance our estimators do better.

  \item It may be puzzling to see that the three-phase estimator does worse than
    the two-phase; however, I think that there is a good reason for this. In the
    simulation, $y_2 = x + \varepsilon$. This is the regression from the
    two-phase estimator. However, the three-phase estimator measures $y_2 \sim x
    + y_1 + \varepsilon$, which just adds noise. This is why I think the
    standard deviation of the three-phase regression estimator is larger.
  
\end{itemize}

\subsection*{Nonmonotone Results}

\begin{itemize}
  \item Similar to the monotone results, we also use the same simulation as the
    nonmonotone calibration estimators. Repeating the simulation outline, we are
    trying to estimate $\theta = E[y_2]$ and we have

    \begin{enumerate}
      \item Generate $X_i$, $\varepsilon_{1i}$, and $\varepsilon_{2i}$ from the
        following distributions:

        \begin{align*}
          x_i &\stackrel{iid}{\sim} N(0, 1)\\
          \varepsilon_{1i} &\stackrel{iid}{\sim} N(0, 1)\\
          \varepsilon_{2i} &\stackrel{iid}{\sim} N(\theta, 1)\\
        \end{align*}

        Then we have
        \[y_{1i} = x_i + \varepsilon_{1i} \text{ and } y_{2i} = x_i +
        \varepsilon_{2i}.\]

      \item Then we have to select the variables to observe. We always observe
        $X_i$. Then we choose to either observe $Y_1$ with probability $0.4$,
        $Y_2$ with probability $0.4$ or neither with probability $0.2$.

      \item If neither then $R_{1i} = 0$ and $R_{2i} = 0$. Otherwise, if we 
        observe $Y_1$ then $R_1 = 1$ and if we observe $Y_2$ then $R_2 = 1$.

      \item If we observe either $Y_1$ or $Y_2$ then with probability $p \propto
        \logistic(Y_k)$ where $Y_k$ is the observed $Y$ variable we choose to
        observe the other $Y$ variable.

      \item If the other $Y$ variable is observed then the corresponding $R_k =
        1$. Otherwise, $R_k = 0$.
    \end{enumerate}

    \begin{table}[h!]

\caption{True Value is -5. Cor(Y1, Y2) = 0}
\centering
\begin{tabular}[t]{lrrrr}
\toprule
algorithm & bias & sd & tstat & pval\\
\midrule
oracle & -0.001 & 0.045 & -0.953 & 0.170\\
ipworacle & 0.011 & 0.552 & 0.627 & 0.266\\
proposed & -0.002 & 0.055 & -0.873 & 0.191\\
calib & -0.002 & 0.054 & -1.312 & 0.095\\
\bottomrule
\end{tabular}
\end{table}

    \begin{table}[h!]

\caption{True Value is 0. Cor(Y1, Y2) = 0}
\centering
\begin{tabular}[t]{lrrrr}
\toprule
algorithm & bias & sd & tstat & pval\\
\midrule
oracle & -0.001 & 0.044 & -0.945 & 0.173\\
ipworacle & 0.001 & 0.112 & 0.178 & 0.429\\
proposed & -0.001 & 0.053 & -0.363 & 0.358\\
reg2p & 0.004 & 0.069 & 1.809 & 0.035\\
reg3p & 0.005 & 0.069 & 2.372 & 0.009\\
\addlinespace
calib & -0.001 & 0.052 & -0.508 & 0.306\\
\bottomrule
\end{tabular}
\end{table}

    \begin{table}[h!]

\caption{True Value is 5. Cor(Y1, Y2) = 0}
\centering
\begin{tabular}[t]{lrrrr}
\toprule
algorithm & bias & sd & tstat & pval\\
\midrule
oracle & -0.002 & 0.045 & -1.358 & 0.087\\
ipworacle & -0.002 & 0.141 & -0.409 & 0.341\\
proposed & -0.002 & 0.051 & -1.531 & 0.063\\
reg2p & -0.003 & 0.052 & -1.589 & 0.056\\
reg3p & -0.003 & 0.052 & -1.565 & 0.059\\
\addlinespace
calib & -0.002 & 0.051 & -1.401 & 0.081\\
\bottomrule
\end{tabular}
\end{table}


  \item Overall, it seems that the proposed estimator and calibration estimator
    outperform the two regression estimators. The regression estimators display
    slight bias (perhaps because of the nonmonotonicity).

\end{itemize}

\newpage


\printbibliography

\end{document}


