\documentclass[12pt]{article}

\usepackage{amsmath, amssymb, amsthm, mathrsfs, fancyhdr}
\usepackage{syntonly, lastpage, hyperref, enumitem, graphicx}
%\usepackage[style=authoryear]{biblatex}
\usepackage{booktabs}
\usepackage{float}

\usepackage{amsmath, amsthm, mathtools}
\usepackage{graphics, color}
\usepackage{latexsym}
\usepackage{amssymb, amsfonts, bm}
\usepackage{mathrsfs}

\usepackage{graphicx}
\usepackage{epsfig}
\usepackage{makeidx}
\usepackage{fullpage}
\usepackage{booktabs, arydshln}
\usepackage{comment} 
%\makeindex

\hbadness=10000 \tolerance=10000 \hyphenation{en-vi-ron-ment
in-ven-tory e-num-er-ate char-ac-ter-is-tic}

\usepackage[round]{natbib}
%\bibliographystyle{apalike2}
% \bibliographystyle{jmr}


\newcommand{\biblist}{\begin{list}{}
{\listparindent 0.0cm \leftmargin 0.50cm \itemindent -0.50 cm
\labelwidth 0 cm \labelsep 0.50 cm
\usecounter{list}}\clubpenalty4000\widowpenalty4000}
\newcommand{\ebiblist}{\end{list}}

\newcounter{list}

%\usepackage{setspace}

%\usepackage{hangpar}
\newcommand{\lbl}[1]{\label{#1}{\ensuremath{^{\fbox{\tiny\upshape#1}}}}}
% remove % from next line for final copy
\renewcommand{\lbl}[1]{\label{#1}}

\newtheorem{theorem}{Theorem}
\newtheorem{corollary}{Corollary}
\newtheorem{definition}{Definition}
\newtheorem{example}{Example}
\newtheorem{remark}{Remark}
\newtheorem{result}{Result}

\newtheorem{lemma}{Lemma}

\DeclareMathOperator*{\argmin}{arg\,min}
\DeclareMathOperator*{\argmax}{arg\,max}
\newcommand{\MAP}{{\text{MAP}}}
\newcommand{\Cov}{{\text{Cov}}}
\newcommand{\Var}{{\text{Var}}}
\newcommand{\logistic}{{\text{logistic}}}

\newcommand{\bx}{\mathbf{x}}
\newcommand{\R}{\mathbb{R}}
\renewcommand{\bf}[1]{\mathbf{#1}}

\begin{document}

\title{Debiased Calibration for Generalized Two-Phase Sampling}
\author{Caleb Leedy, Jae-Kwang Kim}
\maketitle 

\baselineskip .3in

\begin{abstract}
Generalized entropy calibration \cite{gneiting2007strictly} is a nonparametric
method to estimate survey weights without using the design weights in the
specified loss function. While \cite{kwon2024debiased} explore generalized
entropy calibration to estimate survey weights with known population totals, by
using a two-phase sampling framework, we extend their method to allow for
estimated population totals. We then extend this two-phase sampling framework to
two additional cases: first, when we have two non-nested samples, and
second, when we want to combine three or more samples. 
We apply our method to the Medicare Current Beneficiary Survey for which we
get more efficient estimates by combining this data set with population level
controls obtained from the Center of Medicare and Medicaid Services as well as
survey data from the U.S. Census' American Community Survey.
\end{abstract}

\section{Introduction}

Combining information from multiple samples is an important practical problem
(\cite{yang2020statistical}, \cite{yang2023elastic}, \cite{dagdoug2023model}).
We want to incorporate information from external data sources to reduce the bias
in our estimates and/or improve the estimator's efficiency. For many problems, the
additional information consists of summary statistics with standard errors. The
goal of this project is to incorporate external information with existing data 
to create more efficient estimators using calibration weighting.

To model this scenario, we formulate the problem as a generalized two-phase
sample. The first phase sample consists of data from multiple sources, and the
second phase sample contains the existing data. To motivate this setup, we
take the following approach: first, we consider the classical two-phase
sampling setup where the second phase sample is a subset of the first phase
sample; then, we extend this setup to consider non-nested two-phase samples;
and finally, we consider the more general approach of having multiple data sets.


\textcolor{red}{The overall structure needs to be changed. You can change the current section 2 to introduce the notation and review the basic properties of the regression estimator in two phase sampling. Section 3 is the main proposal of the debiasing calibration for two-phase sampling. Section 4 is an extension of the proposal to non-nested two-phase sampling and multi-source data integration. The simulation studies in the subsections of Sections 2-4 should be moved to Section 5. The real data analysis in Section 5 should be changed to Section 6. Concluding remarks are made in Section 7. In summary, here is my suggestion for the structure.
\begin{enumerate}
\item Introduction
\item Basic setup: introduce the notation, two-phase regression estimator and its calibration form
\item Main Proposal: introduce debiased calibration for two-phase sampling and discuss its statistical properties 
\item Extensions: \begin{enumerate}
\item Extension to non-nested two-phase sampling 
\item Extension to multi-source data integration
\end{enumerate}
\item Simulation study 
\item Real data analysis 
\item Concluding Remarks 
\end{enumerate}
}

\section{Topic 1: Classical Two-Phase Sampling}

\subsection{Background}

Consider a finite population of size $N$ containing elements $(\bf X_i, Y_i)$ where
an initial (Phase 1) sample of size $n_1$ is selected and $\bf X_i$ is observed.
From the Phase 1 sample of elements, a (Phase 2) sample of size $n_2 < n_1$ is
selected and $Y_i$ is observed. This is two-phase sampling. (See 
\cite{fuller2009sampling}, \cite{kim2024statistics} for general references.) The
goal of two-phase sampling is to construct an estimator of $\bar Y_N$ 
that uses both the observed information in the Phase 2 sample and also the extra
auxiliary information from $\bf X$ in the Phase 1 sample.
The challenge is doing this efficiently.

An easy-to-implement unbiased estimator in the spirit of a Horvitz-Thompson (HT)
estimator (\cite{horvitz1952generalization}, \cite{narain1951sampling}) is the
$\pi^*$-estimator. Let $\pi_i^{(2)}$ be the response probability of element $i$
being observed in the Phase 2 sample. Then, allowing the elements in the Phase 1
sample to be represented by $A_1$ and the elements in the Phase 2 sample to be
denoted as $A_2$,
we define $\pi_{2i | 1} = \sum_{A_2: i \in A_2} \Pr(A_2 \mid A_1)$ and
$\pi_{1i} = \sum_{A_1: i \in A_1} \Pr(A_1)$ then,

$$ \pi_i^{(2)}(A_1) = \pi_{2i | 1} \pi_{1i}.$$

This means that we can define the $\pi^*$-estimator as the following design
unbiased estimator:

$$ \hat Y_{\pi^*} = \sum_{i \in A_2} \frac{y_i}{\pi_{2i | 1} \pi_{1i}}.$$

While unbiased (see \cite{kim2024statistics}), the $\pi^*$-estimator does not
account for the additional information contained in the auxiliary Phase 1
variable $\bf X$. The two-phase regression estimator $\hat Y_{reg, tp}$ does
incorporate information for $\bf X$ by using the estimate $\hat{\bf X_1}$ from
the Phase 1 sample. This is one way we can leverage the external information of 
$\hat{\bf{X_1}}$ to improve the initial $\pi^*$-estimator in the second phase
sample. The two-phase regression estimator has the form,

$$ 
\hat Y_{reg, tp} 
= \sum_{i \in A_1} \frac{1}{\pi_{1i}} \bf x_i \hat{\bm \beta_q} + 
\sum_{i \in A_2} \frac{1}{\pi_{1i}\pi_{2i|1}} (y_i - \bf x_i \hat{\bm \beta_q})
$$

where for $q_i = q(\bf x_i)$ and is a function of $\bf x_i$,

$$
\hat{\bm \beta_q} = \left(\sum_{i \in A_2} 
  \frac{\bf x_i \bf x_i'}{\pi_{1i} q_i}\right)^{-1} 
\sum_{i \in A_2} \frac{\bf x_i y_i}{\pi_{1i} q_i}.
$$ 

The regression estimator is the minimum variance design consistent linear
estimator which is easily shown to be the case because $\hat Y_{reg, tp} =
\sum_{i \in A_2} \hat w_{2i} y_i / \pi_{1i}$ where 

$$
\hat w_{2i} = \argmin_{w} \sum_{i \in A_2} (w_{2i} - \pi_{2i|1}^{-1})^2 q_i
\text{ such that } \sum_{i \in A_2} w_{2i} \bf x_i / \pi_{1i} = \sum_{i \in A_1}
\bf x_i / \pi_{1i}.
$$

This means that $\hat Y_{reg, tp}$ is also a calibration estimator. The idea
that regression estimation is a form of calibration was noted by
\cite{deville1992calibration} and extended by them to consider loss functions
other than just squared loss. Their generalized loss function minimizes
$\sum_i G(w_i, d_i)q_i$ for weights $w_i$ and design-weights $d_i$ where
$G(\cdot)$ is a non-negative, strictly convex function with respect to $w$,
defined on an interval containing $d_i$, with $g(w_i, d_i) = \partial G /
\partial w$ continuous.\footnote{The \cite{deville1992calibration} paper
considers regression estimators for a single phase setup, which we apply to our
two-phase example.} This
generalization includes empirical likelihood estimation, and maximum entropy
estimation among others. The variance estimation is based on a linearization
that shows that minimizing the generalized loss function subject to the
calibration constraints is asymptotically equivalent to a regression estimator.

Furthermore, the regression estimator
has a nice feature that its two terms can be thought about as minimizing the
variance and bias correction,

$$ 
\hat Y_{reg, tp} 
= \underbrace{\sum_{i \in A_1} \frac{\bf x_i \hat{\bm \beta_q}}{\pi_{1i}}}_{
  \text{ Minimizing the variance}} + \underbrace{\sum_{i \in A_2}
\frac{1}{\pi_{1i}\pi_{2i|1}} (y_i - \bf x_i \hat{\bm \beta_q})}_{
\text{Bias correction}}.
$$

The \cite{deville1992calibration} method incorporates the design weights into
the loss function, which is the part minimizing the variance. We would rather
separate have bias calibration separate from the minimizing the variance so that
we can control each in isolation. In
\cite{kwon2024debiased}, the authors show that for a generalized entropy
function $G(w)$, including a term of $g(\pi_{2i|1}^{-1})$ into the calibration
for $g = \partial G / \partial w$ not only creates a design consistent
estimator, but it also has better efficiency than the generalized regression
estimators of \cite{deville1992calibration}.

However, the method of \cite{kwon2024debiased} requires known finite population 
calibration levels. It does not handle the
two-phase setup where we need to estimate the finite population total of $\bf x$
from the Phase 1 sample. In the rest of the section, we extend this method to 
two-phase sampling so that we construct asymptotically valid confidence
intervals when using estimate finite population totals derived from the Phase 1
sample.

\subsection{Methodology}

We follow the approach of \cite{kwon2024debiased} for the debiased calibration
method. We consider maximizing a generalized entropy function
\cite{gneiting2007strictly},

\begin{equation}\label{eq:primalloss}
  H(w) = - \sum_{i \in A_2} \frac{1}{\pi_{1i}} G(w_{2i}) q_i
\end{equation}

where $G: \mathcal{V} \to \R$ is strictly convex, differentiable function
subject to the constraints:

\begin{equation}\label{eq:calconst1}
  \sum_{i \in A_2} \frac{\bf x_i w_{2i}q_i}{\pi_{1i}} = 
\sum_{i \in A_1} \frac{\bf x_iq_i}{\pi_{1i}}
\end{equation}

and 

\begin{equation}\label{eq:calconst2}
  \sum_{i \in A_2} \frac{g(\pi_{2i|1}^{-1})w_{2i}q_i}{\pi_{1i}} = 
  \sum_{i \in A_1} \frac{g(\pi_{2i|1}^{-1})q_i}{\pi_{1i}}, 
\end{equation}
where $g(w) = \partial G / \partial w$. 

The first constraint is the existing calibration constraint and the second
ensures that design consistency is achieved. 
The original method of \cite{kwon2024debiased} only considers known finite
population quantities on the right hand side of \eqref{eq:calconst1}.

By writing $w_{1i} = \pi_{1i}^{-1}$, the the goal is to solve

\begin{equation}\label{eq:primal}
  \argmin_{w_{2|1}} \sum_{i \in A_2} w_{1i} G(w_{2i}) q_i 
  \text{ such that}
  \sum_{i \in A_2} w_{1i} w_{2i|1} \bf z_i q_i = \sum_{i \in A_1} w_{1i} \bf z_i q_i
\end{equation}

where  $\bf z_i = (\bf x_i / q_i, g(\pi_{2i|1}^{-1}))$. 
\textcolor{red}{Did you use $w_{2i \mid 1}$ to denote $w_{2i}$? Why do you introduce new notation?  } 
Let $\hat w_{2i|1}$ be the solution to Equation~\eqref{eq:primal}. The resulting 
estimator of $Y_N$ is $\hat Y_{DCE} = \sum_{i \in A_2} w_{1i} \hat w_{2i|1} y_i$.
We can get an estimator using the method of Lagrange multipliers. We need to
minimize the Lagrangian function

\begin{equation}\label{eq:legragedc1}
  L(w_{2i|1}, \bm \lambda) = \sum_{i \in A_2} w_{1i} G(w_{2i|1}) q_i 
  + \bm \lambda \left( \sum_{i \in A_1} w_{1i} \bf z_i q_i -
    \sum_{i \in A_2} w_{1i} w_{2i|1} \bf z_i q_i\right)
\end{equation}

where $\bm \lambda$ is a vector of Lagrange multipliers.
Differentiating with respect to $w_{2i|1}$ and setting this expression equal to
zero, yields the fact that $\hat w_{2i|1}$ satisfies 

$$ \hat w_{2i|1}(\hat{\bm \lambda}) = g^{-1}(\hat{\bm \lambda}^T \bf z_i) $$

where $\hat{\bm \lambda}$ is the solution to

\begin{equation}\label{eq:lamdc1}
  \left( \sum_{i \in A_1} w_{1i} \bf z_i q_i -
  \sum_{i \in A_2} w_{1i} w_{2i|1}(\hat{\bm \lambda}) \bf z_i q_i\right) = 0.
\end{equation}

\subsection{Theoretical Results}

The first result that we show is that the construction of the weights in
Equation~\ref{eq:primal} leads to a design consistent estimator $\hat Y$.

\begin{theorem}[Design Consistency]\label{thm:dc1}
  Let $\bm \lambda^*$ be the probability limit of $\hat{\bm \lambda}$.
  Under some regularity conditions,

  $$\hat Y_{DCE} = \hat Y_\ell(\bm \lambda^*, \bm \phi^*) + O_p(N / n_2)$$

  where

  %% FIXME: This needs to be defined in terms of \hat Y_{\pi^*}
  $$
  \hat Y_{\ell}(\bm \lambda, \bm \phi^*) = \hat Y_{DCE}(\hat{\bm \lambda}) + 
  \left(\sum_{i \in A_1} w_{1i} \bf z_i q_i - \sum_{i \in A_2} w_{1i} \hat w_{2i|1}(
  \bm \lambda) \bf z_i q_i\right)\bm \phi^*
  $$

  and

  $$
  \bm \phi^* = 
  \left[\sum_{i \in U} \frac{\pi_{2i|1}\bf z_i \bf z_i^T q_i}{g'(d_{2i|1})}\right]^{-1}
  \sum_{i \in U} \frac{\pi_{2i|1}\bf z_i y_i}{g'(d_{2i|1})}.
  $$

   \begin{eqnarray*}
   \hat Y_{\ell}(\bm \lambda^*, \bm \phi^*) &=&   \hat{Y}_{\pi^*} + 
  \left(\sum_{i \in A_1} w_{1i} \bf x_i  -  \sum_{i \in A_2} w_{1i} \pi_{2i \mid
  1}^{-1} \bf x_i  \right)^T \bm \phi_1^* + \left(\sum_{i \in A_1} w_{1i} g_i  -
  \sum_{i \in A_2} w_{1i} \pi_{2i \mid 1}^{-1}g_i  \right)^T \phi_2^*  
  \end{eqnarray*} 

  and  

  $$
  \begin{pmatrix}
  \bm \phi_1^* \\
  \phi_2^* 
  \end{pmatrix}
  = \left[ \sum_{i \in U} \frac{\pi_{2i \mid 1} }{ g'(d_{2i|1}) q_i} 
  \begin{pmatrix}
  \bx_i \bx_i^T &   \bx_i g_i   \\
  g_i  \bx_i^T   & g_i^2     
  \end{pmatrix} \right]^{-1}
  \sum_{i \in U} \frac{\pi_{2i|1}}{ g'(d_{2i | 1}) q_i} 
  \begin{pmatrix}
    \bx_i \\ g_i 
  \end{pmatrix}y_i 
  $$

with $g_i = g( \pi_{2i |1}^{-1}) q_i$.
\end{theorem}


\begin{proof}
  In this proof, we derive the solution to Equation~\ref{eq:primal} and show
  that it is asymptotically equivalent to a regression estimator. Using the
  method of Lagrange multipliers, to solve Equation~\eqref{eq:primal} we need to
  minimize the Lagrangian in Equation~\eqref{eq:legragedc1}. 
  The first order conditions show that

  $$
  \frac{\partial \mathcal{L}}{\partial w_{2i|1}}: g(w_{2i|1}) w_{1i}q_i -
  w_{1i} \bm \lambda \bf z_i q_i = 0.
  $$

  Hence, $\hat w_{2i}(\bm \lambda) = g^{-1}(\bm \lambda^T \bf z_i)$ and
  $\hat{\bm \lambda}$ is determined by Equation~\eqref{eq:lamdc1}. When the
  sample size gets large, we have $\hat w_{2i|1}(\hat{\bm \lambda}) \to
  d_{2i|1}$ which means that $\hat{\bm \lambda} \to \bm \lambda^*$ where $\bm
  \lambda^* = (\bf 0, 1)$. Then using the linearization technique of
  \cite{randles1982asymptotic}, we can construct a regression estimator, 

  $$
  \hat Y_\ell(\hat{\bm \lambda}, \bm \phi)  = \hat Y_{DCE}(\hat{\bm \lambda}) + 
  \left(\sum_{i \in A_1} w_{1i} \bm z_i q_i - \sum_{i \in A_2} w_{1i} 
  \hat w_{2i|1}(\hat{\bm \lambda}) \bf z_i q_i\right)\bm \phi.
  $$

  Notice that $\hat Y_\ell(\hat{\bm \lambda}, \bm \phi) = \hat Y_{DCE}(\hat{\bm
  \lambda})$ for all $\bm \phi$. We choose $\bm \phi^*$ such that

  $$
  E\left[\frac{\partial}{\partial \bm \lambda} \hat Y_\ell(\bm \lambda^*,
  \bm \phi^*)\right]=0.
  $$

  Using the fact that $g^{-1}(\bm \lambda^* \bf z_i) = g^{-1}(g(d_{2i|1})) = d_{2i|1}$
  and $(g^{-1})'(x) = 1 / g'(g^{-1}(x))$, we have

  \begin{align*}
    \bm \phi^*
    &= E\left[\sum_{i \in A_2} \frac{w_{1i}\bf z_i \bf z_i^T q_i}{g'(d_{2i|1})}
    \right]^{-1}
    E\left[\sum_{i \in A_2} \frac{w_{1i}\bf z_i y_i}{g'(d_{2i|1})}\right]\\
    &= \left[\sum_{i \in U} \frac{\pi_{2i|1} \bf z_i \bf z_i^T q_i}{g'(d_{2i|1})}
    \right]^{-1}
    \left[\sum_{i \in U} \frac{\pi_{2i|1} \bf z_i y_i}{g'(d_{2i|1})}\right]\\
  \end{align*}

  Thus, the linearization estimator is

  $$\hat Y_\ell(\bm \lambda^*, \bm \phi^*) = 
  \sum_{i \in A_1} w_{1i} q_i \bf z_i \bm \phi^* +
  \sum_{i \in A_2} w_{1i} d_{2i|1} (y_i - q_i \bf z_i \bm \phi^*).$$

  By construction using a Taylor expansion yields,

  \begin{align*}
    \hat Y_{DCE}(\hat{\bm \lambda}) 
    &= \hat Y_\ell(\bm \lambda^*, \bm \phi^*) + 
    E\left[\frac{\partial}{\partial \bm \lambda}\hat Y_\ell(\bm \lambda^*,
    \bm \phi^*)\right](\hat{\bm \lambda} - \bm \lambda^*) + \frac{1}{2}
    E\left[\frac{\partial}{\partial \bm \lambda^2} \hat Y_{DCE}(\bm \lambda^*)\right] 
    (\hat{\bm \lambda} - \bm \lambda^*)^2\\
    &= \hat Y_\ell(\bm \lambda^*, \bm \phi^*) + O(N)O_p(n_2^{-1}).
  \end{align*}

  The final equality comes from the fact that 
  $E\left[\frac{\partial}{\partial \bm \lambda}
  \hat Y_\ell(\bm \lambda^*, \bm \phi^*)\right] = 0$,
  $\frac{\partial}{\partial \bm \lambda^2} \hat w_{2i|1}(\bm \lambda^*)$ is
  bounded and $||\hat{\bm \lambda} - \bm \lambda^*|| = O_p(n_2^{-1/2})$, which
  proves our result.
\end{proof}



\textcolor{red}{This is not enough. We need to discuss variance estimation. }
\subsection{Simulation Study 1}

\textcolor{red}{This subsection should be moved to Section 5. }

We run a simulation testing the proposed method. In this approach we have the
following simulation setup:

$$
\begin{aligned}
X_{1i} &\stackrel{ind}{\sim} N(2, 1) \\
X_{2i} &\stackrel{ind}{\sim} Unif(0, 4) \\
X_{3i} &\stackrel{ind}{\sim} N(0, 1) \\
\varepsilon_i &\stackrel{ind}{\sim} N(0, 1) \\
Y_{i} &= 3 X_{1i} + 2 X_{2i} + \delta 0.5 X_{3i} + \varepsilon_i \\
\pi_{1i} &= n_1 / N \\
\pi_{2i|1} &= \max(\min(\Phi_3(x_{3i} - 1), 0.7), 0.02).
\end{aligned}
$$

where $\Phi_3$ is the CDF of a t-distribution with 3 degrees of freedom.
This is a two-phase extension of the setup in \cite{kwon2024debiased}. We
consider a finite population of size $N = 10,000$ with the Phase 1 sampling
being a simple random sample (SRS) and the 
Phase 2 sampling occuring under Poisson sampling. The Phase 1 sample has
$n_1 = 1200$ and the Phase 2 sample has an expected size of
$E[n_2] \approx 2588$. In the Phase 1 sample, we observe 
$(X_1, X_2)$ while in the Phase 2 sample we observe $(X_1, X_2, Y)$. 
The parameter $\delta \in \{0, 1\}$ controls the effect of model misspecification
in the simulation. Let $\bf x_i = (1, x_{1i}, x_{2i})^T$. We compare the
proposed method for the parameter $\bar Y_N$ with four approaches:

\begin{itemize}
  \item[1.] $\pi^*$-estimator (PiStar): $\hat Y_{\pi^*} = N^{-1} \sum_{i \in A_2}
    \frac{y_i}{\pi_{1i} \pi_{2i|1}},$
  \item[2.] Two-Phase Regression estimator (Reg): 
    $\hat Y_{reg} = \sum_{i \in A_1} \frac{\bf x_i' \widehat{\bm \beta}}{\pi_{1i}} + 
    \sum_{i \in A_2} \frac{1}{\pi_{1i}\pi_{2i|1}}(y_i - \bf x_i' \widehat{\bm \beta})$ 
    where $\widehat{\bm \beta} = 
    \left(\sum_{i \in A_2} \bf x_i \bf x_i'\right)^{-1} \sum_{i \in A_2} \bf x_i y_i$.
  \item[3.] Debiased Calibration with Population Constraints (EstPop): This 
    solves 

  \begin{equation}
    \argmin_{w_{2|1}} \sum_{i \in A_2} w_{1i} G(w_{2i}) \text{ such that}
    \sum_{i \in A_2} w_{1i} w_{2i|1} \bf z_i = \sum_{i \in U} \bf z_i.
  \end{equation}

  \item[4.] Debiased Calibration with Estimated Population Constraints (Est):
    This solves Equation~\eqref{eq:primal} with $q_i = 1$.
\end{itemize}

%In addition to estimating the mean parameter $\bar Y_N$, we also construct
%variance estimates $\hat V(\hat Y)$ for each estimate $\hat Y$. For each
%approach we give the variance estimate in Table~\ref{tab:varforms}.
%% Move variance calculations into an appendix?
%
%\begin{table}[ht!]
%  \centering
%  \begin{tabular}{lcc}
%    \toprule
%    Estimator & Estimated Variance & Notes \\
%    \midrule
%    $\pi^*$ 
%    & {\scriptsize$\left(\frac{1}{n_1} - \frac{1}{N}\right) \hat s^2_Y + 
%      \sum\limits_{i \in A_2} \frac{w_{1i}^2}{N^2} 
%        \frac{(1 - \pi_{2i|1})}{\pi_{2i|1}^2}y_i^2$}
%    & {\scriptsize$\hat s^2_Y = \frac{1}{n_2 - 1}\sum\limits_{i \in A_2} (y_i -
%    \bar y_i)^2$} \\
%    TP-Reg  
%    & {\scriptsize$\left(\frac{1}{n_1} - \frac{1}{N}\right) \hat s^2_\eta + 
%      \sum\limits_{i \in A_2} \frac{w_{1i}}{N^2} 
%        \frac{(1 - \pi_{2i|1})}{\pi_{2i|1}^2}\varepsilon_i^2$}
%    & {\scriptsize$ \eta_i = \bf x_i'\bm \beta + \delta_{2i} w_{2i|1}\varepsilon_i,\;
%        \varepsilon_i = (y_i - \bf x_i' \bm \beta)$} \\
%    DC-Pop  
%    & {\scriptsize$\left(\frac{1}{n_1} - \frac{1}{N}\right) \hat s^2_\varepsilon + 
%      \sum\limits_{i \in A_2} \frac{w_{1i}}{N^2} 
%        \frac{(1 - \pi_{2i|1})}{\pi_{2i|1}^2}\varepsilon_i^2$}
%    & {\scriptsize$\eta_i = \bf z_i' \bm \phi + \delta_{2i}w_{2i|1}\varepsilon_i,\;
%        \varepsilon_i = (y_i - \bf z_i'\bm \phi)$} \\
%    DC-Est  
%    & {\scriptsize$\left(\frac{1}{n_1} - \frac{1}{N}\right) \hat s^2_\eta + 
%      \sum\limits_{i \in A_2} \frac{w_{1i}}{N^2} 
%        \frac{(1 - \pi_{2i|1})}{\pi_{2i|1}^2}\varepsilon_i^2$}
%    & {\scriptsize$\eta_i = \bf z_i' \bm \phi + \delta_{2i}w_{2i|1}\varepsilon_i,\;
%        \varepsilon_i = (y_i - \bf z_i' \bm \phi)$} \\
%    \bottomrule
%  \end{tabular}
%  \caption{This table gives the formulas for each variance estimator used in
%  Simulation 1. For the derivation of the estimated variance of the DC-Pop
%estimatator see Appendix A.}
%  \label{tab:varforms}
%\end{table}

We run the simulation $B = 1000$ times for each of these methods and compute the
Bias ($E[\hat Y] - \bar Y_N$), the RMSE ($\sqrt{\Var(\hat Y - \bar Y_N)}$), a 95\%
empirical confidence interval ($\sum_{b = 1}^{1000} |\hat Y^{(b)} - \bar Y_N| \leq 
\Phi(0.975)\sqrt{\hat V(\hat Y^{(b)})^{(b)}}$), and a T-test that assesses the
unbiasedness of each estimator where $\hat Y^{(b)}$ is the result from the $b$th
simulation replicate. We also include the Monte Carlo variance of the estimated
value, ($V_{MC} = (B - 1)^{-1} \sum_{b = 1}^B (\hat Y^{(b)} - \bar Y_N)^2$), the
mean of the estimated variance, 
($\bar{\hat V} = B^{-1} \sum_{b = 1}^B \hat V^{(b)}$), and the relative bias of
the estimated variance, ($(V_{MC} - \bar{\hat V}) / V_{MC}$). The results are in
Table~\ref{tab:tpdc0-mean} and Table~\ref{tab:tpdc1-mean}.

\begin{table}[ht!]
  \centering
  
\begin{tabular}{lrrrrrrrr}
\toprule
Est & Bias & SE & RMSE & EmpCI & Ttest & MCVar & EstVar & RelBias\\
\midrule
Est & 0.003 & 0.135 & 0.134 & 0.945 & 0.70 & 0.018 & 0.017 & -0.074\\
EstPop & 0.000 & 0.082 & 0.082 & 0.975 & 0.07 & 0.007 & 0.009 & 0.317\\
PiStar & -0.031 & 0.829 & 0.829 & 0.943 & 1.17 & 0.688 & 0.692 & 0.006\\
Reg & 0.004 & 0.135 & 0.135 & 0.950 & 0.86 & 0.018 & 0.018 & -0.031\\
\bottomrule
\end{tabular}
\caption{This table shows the results of Simulation Study 1 with $\delta = 0$.
  It displays the Bias, RMSE, empirical 95\% confidence interval, a t-statistic
  assessing the unbiasedness, the Monte Carlo variance, mean estimated variance
  and relative bias of the variance estimator for the estimators: PiStar, Reg,
  EstPop, and Est.}
\label{tab:tpdc0-mean}
\end{table}

\begin{table}[ht!]
  \centering
  
\begin{tabular}{lrrrrrrrr}
\toprule
Est & Bias & SE & RMSE & EmpCI & Ttest & MCVar & EstVar & RelBias\\
\midrule
Est & 0.007 & 0.136 & 0.136 & 0.941 & 1.56 & 0.019 & 0.017 & -0.072\\
EstPop & 0.004 & 0.085 & 0.085 & 0.973 & 1.48 & 0.007 & 0.009 & 0.320\\
PiStar & -0.028 & 0.794 & 0.794 & 0.941 & 1.13 & 0.631 & 0.635 & 0.007\\
Reg & 0.011 & 0.153 & 0.154 & 0.940 & 2.19 & 0.024 & 0.024 & -0.001\\
\bottomrule
\end{tabular}
\caption{This table shows the results of Simulation Study 1 with $\delta = 1$.
  It displays the Bias, RMSE, empirical 95\% confidence interval, a t-statistic
  assessing the unbiasedness, the Monte Carlo variance, mean estimated variance
  and relative bias of the variance estimator for the estimators: PiStar, Reg,
  EstPop, and Est.}
\label{tab:tpdc1-mean}
\end{table}

\section{Topic 2: Non-nested Two-Phase Sampling}

\subsection{Background}

Now we consider the sampling mechanism known as non-nested two-phase sampling 
(\cite{hidiroglou2001double}). In the last section, we considered two-phase sampling
in which the Phase 2 sample was a subset of the Phase 1 sample. With non-nested
two-phase sampling the Phase 2 sample is independent of the Phase 1 sample. 
Like traditional two-phase sampling, we consider the Phase 1 sample, $A_1$, to
consist of observations of $(\bf X_i)_{i = 1}^{n_1}$ and the Phase 2 sample,
$A_2$, to consist of observations of $(\bf X_i, Y_i)_{i = 1}^{n_2}$. 

Whereas the classical two-phase estimator uses a single Horvitz-Thompson
estimator of the Phase 1 sample to construct estimates for calibration totals,
in the non-nested two-phase sample we have two independent Horvitz-Thompson
estimators of the total of $\bf X$,

$$\hat{\bf X_1} = \sum_{i \in A_1}^{n_1} d_{1i} \bf x_i \text{ and } 
\hat{\bf X_2} = \sum_{i \in A_2}^{n_2} d_{2i} \bf x_i $$

where $d_{1i} = \pi_{1i}^{-1}$, $d_{2i} = \pi_{2i}^{-1}$, $\pi_{1i}$ is the
probability of $i \in A_1$ and $\pi_{2i} = \Pr(i \in A_2)$. 
We can combine these estimates using the effective sample size 
(\cite{kish1965survey}) to get
$\hat{\bf X_c} = (n_{1, \rm{eff}} \hat{\bf X_1} + n_{2, \rm{eff}}\hat{\bf X_2}) / 
(n_{1, \rm{eff}} + n_{2, \rm{eff}})$ where $n_{1, \rm{eff}}$ and 
$n_{2, \rm{eff}}$ are the effective sample size for $A_1$ and $A_2$ respectively.
Then we can define a regression estimator as

$$
\hat Y_{NN, reg} = \hat Y_2 + (\hat{\bf X_c} - \hat{\bf X_2})^T \hat{\bm \beta_q} = 
\hat Y_2 + (\hat{\bf X_1} - \hat{\bf X_2})^T W\hat{\bm \beta_q}
$$

where, $W = n_{1, \rm{eff}} / (n_{1, \rm{eff}} + n_{2, \rm{eff}})$, and

$$
\hat{\bm \beta_q} = 
\left(\sum_{i \in A_2} \frac{\bf x_i \bf x_i^T}{q_i}\right)^{-1} 
\sum_{i \in A_2} \frac{\bf x_i y_i}{q_i} \text{ and }
\hat Y_2 = \sum_{i \in A_2} d_{2i} y_i.
$$

From an optimality perspective the choice of using the effective sample size to
weight the estimates from $A_1$ and $A_2$ is reasonable because often the
effective sample sizes are proportional to the variance of the estimates 
of $\bf X$. While we know that the inverse variance weighted estimate is optimal
to combine independent samples for a linear estimate, using the effective sample
size approximates this procedure without requiring the actual variance of an
estimator to be known.

Since the samples $A_1$ and $A_2$ are independent, 

$$
V(\hat Y_{NN, reg}) = 
V\left(\sum_{i \in A_2} \frac{1}{\pi_{2i}}(y_i - \bf x_i^TW\bm \beta^*_q)\right)
+ (\bm \beta^*_q)^T W^T V(\hat{\bf X_1}) W \bm \beta_q^*
$$

where $\bm \beta_q^*$ is the probability limit of $\hat{\bm \beta_q}$. Like the
two-phase sample this regression estimator can be viewed as the solution to the
following calibration equation 

\begin{equation}\label{eq:nncal}
  \hat w_2 = \argmin_{w_2} Q(w_2) = \sum_{i \in A_2} (w_{2i} - d_{2i})^2 q_i 
  \text{ such that } \sum_{i \in A_2} w_{2i} \bf x_i = \hat{\bf X_c}
\end{equation}

and $\hat Y_{NN, reg} = \sum_{i \in A_2} \hat w_{2i} y_i$ where $\hat
w_{2i}$ is the solution to Equation~\ref{eq:nncal}.

We extend the debiased calibration estimator of \cite{kwon2024debiased} to the
non-nested two-phase sampling case where we use a combined estimate 
$\hat{\bm X_c}$ as the calibration totals instead of using the true totals from
the finite population.

\subsection{Methodology}

The methodology for the non-nested two-phase sample is very similar to the setup
described as part of Topic 1. Given a strictly convex differentiable function,
$G: \mathcal{V} \to \mathbb{R}$, the goal is to solve

\begin{equation}\label{eq:nnopt}
\hat w_2 = \argmin_w \sum_{i \in A_2} G\left(w_{2i}\right) q_i 
\text{ with } 
\sum_{i \in A_2} w_{2i} \bf x_i = \hat{\bf X_{c}} \text{ and } 
\sum_{i \in A_2} w_{2i} g(d_{2i}) q_i = \sum_{i \in U} g(d_{2i}) q_i
\end{equation}

for $g(x) = G'(x)$ and a known choice of $q_i \in \mathbb{R}$. 
The difference
between solving Equation~\ref{eq:nnopt} and Equation~\ref{eq:primal} is that the
estimator $\hat{\bf X_c}$ is estimated from the combined sample $A_c = A_1 \cup
A_2$. Before using $\hat{\bf X_c}$ in the debiased calibration estimator, we
need to estimate it from the non-nested samples. We can get multiple estimates
of $\hat{\bf X}$, 

$$
\hat{\bf X_1} = \sum_{i \in A_1} d_{1i} \bf x_i \text{ and }
\hat{\bf X_2} = \sum_{i \in A_2} d_{2i} \bf x_i.
$$

Let $n_{1, \rm{eff}}$ and $n_{2, \rm{eff}}$ be the effective samples sizes for
$A_1$ and $A_2$ respectively. Then the optimal combined estimate is 

$$
\hat{\bf X_c} = (n_{1, \rm{eff}} \hat{\bf X_1} + n_{2, \rm{eff}}\hat{\bf X_2}) / 
(n_{1, \rm{eff}} + n_{2, \rm{eff}})
$$

We can constuct a non-nested two-phase estimator $\hat Y_{NNE}$ for $Y_N$ where
$\hat Y_{NNE} = \sum_{i \in A_2} \hat w_{2i} y_i$ and $\hat w_{2i}$ solves
Equation~\ref{eq:nnopt}. Like the classical two-phase approach, to solve this
setup we minimize the Lagrangian,

\begin{equation}\label{eq:legragedc2}
  L(w_{2i}, \bm \lambda) = \sum_{i \in A_2} G(w_{2i}) q_i + 
  \bm \lambda \left( \hat{\bf T} - \sum_{i \in A_2} w_{2i} \bf z_i q_i\right).
\end{equation}

where $\bm \lambda$ are the Lagrange multipliers, with 

$$\hat{\bf T} = 
\begin{bmatrix}
  \hat{\bf X_c} \\ \sum_{i \in U} g(d_{2i}) q_i
\end{bmatrix}.
$$

Differntiating with respect to $w_{2i}$ and setting this expression equal to
zero, yields the fact that $\hat w_{2i}$ satisfies 

$$ \hat w_{2i}(\hat{\bm \lambda}) = g^{-1}(\hat{\bm \lambda^T} \bf z_i) $$

where $\hat{\bm \lambda}$ is the solution to

\begin{equation}\label{eq:lamdc2}
  \left( \hat{\bf T} - \sum_{i \in A_2} w_{2i}(\hat{\bm \lambda}) \bf z_i
  q_i\right) = 0.
\end{equation}

\subsection{Theoretical Results}

\begin{theorem}[Design Consistency]\label{thm:dc2}
  Allowing $\bm \lambda^*$ to be the probability limit of $\hat{\bm \lambda}$,
  under some regularity conditions, $\hat Y_{NNE} = \hat Y_{\ell, NNE}(\bm \lambda^*,
  \bm \phi^*) + O_p(Nn_2^{-1})$ where

  $$
  \hat Y_{\ell, NNE}(\bm \lambda^*, \bm \phi^*) = \sum_{i \in A_2} \hat w_{2i}(
  \bm \lambda^*) + \left(\hat{\bf T} - \sum_{i \in A_2} \hat w_{2i}(
  \bm \lambda^*) \bf z_i q_i\right) \bm \phi^*
  $$

  and 

  $$
  \bm \phi^* =
  \left(\sum_{i \in U} \frac{\pi_{2i} q_i}{g'(d_{2i})} 
  \begin{bmatrix}
    \bf x_i^2 / q_i & \bf x_i g(d_{2i}) / q_i \\
    \bf x_i g(d_{2i}) / q_i & g(d_{2i})^2
  \end{bmatrix}
  \right)^{-1}
  \sum_{i \in U} \frac{\pi_{2i} y_i}{g'(d_{2i})} 
  \begin{bmatrix} \bf x_i / q_i \\ g(d_i) \end{bmatrix}.
  $$
\end{theorem}

\begin{proof}
  The proof of this result is very similar to the proof the
  Theorem~\ref{thm:dc1}. The biggest difference is that the total for $\bf X$ is
  estimated from both samples using $\hat{\bf X_c}$ instead of $\hat{\bf
  X_{HT}}$ from the Phase 1 sample.

  Since $\hat Y_{NNE} = \sum_{i \in A_2} \hat w_{2i}(\hat{\bm \lambda})y_i$ where
  $\hat{\bm \lambda}$ solves

  \begin{equation}
    \sum_{i \in A_2} \hat w_{2i}(\bm \lambda) q_i
    \underbrace{
    \begin{bmatrix} 
      \bf x_i / q_i \\ g(d_i)
  \end{bmatrix}}_{\bf z_i} = \bf T
  \end{equation}

  we have 

  $$\hat Y_{\ell, NNE} (\hat{\bm \lambda}, \bm \phi) = \sum_{i \in A_2} \hat
  w_{2i}(\hat{\bm \lambda}) + \left(\bf T - \sum_{i \in A_2} \hat
  w_{2i}(\hat{\bm \lambda}) \bf z_i q_i\right) \bm \phi.$$

  If we choose $\bm \phi^*$ such that $E\left[\frac{\partial}{\partial \bm \lambda} 
    \hat Y_{\ell, NNE}(\bm \lambda^*, \bm \phi^*)\right] = 0$, then

  $$\bm \phi^* =
  \begin{bmatrix}
    \bm \phi^*_1 \\ \bm \phi^*_2
  \end{bmatrix} = 
  \left(\sum_{i \in U} \frac{\pi_{2i} q_i}{g'(d_{2i})} 
  \begin{bmatrix}
    \bf x_i^2 / q_i & \bf x_i g(d_{2i}) / q_i \\
    \bf x_i g(d_{2i}) / q_i & g(d_{2i})^2
  \end{bmatrix}
  \right)^{-1}
  \sum_{i \in U} \frac{\pi_{2i} y_i}{g'(d_{2i})} 
  \begin{bmatrix} \bf x_i / q_i \\ g(d_i) \end{bmatrix}.
  $$

  Hence, by a Taylor expansion around $\hat{\bm \lambda}$,
  
  $$\hat Y_{NNE}(\hat{\bm \lambda}) = \hat Y_{\ell, NNE}(\bm \lambda^*, 
  \bm \phi^*) + O_p(Nn_2^{-1}).$$

\end{proof}

\begin{theorem}[Variance Estimation]\label{thm:nnve}
  The variance of $\hat Y_{NNE}$ is 

  \begin{align*}
    \Var(\hat Y_{NNE}(\hat \lambda)
    &= (\bm \phi_1^*)^T \Var(\hat{\bf X_c}) \bm \phi_1^* + 
    \sum_{i \in U} \sum_{j \in U} \frac{\Delta_{2ij}}{\pi_{2i}\pi_{2j}} 
      (y_i - \bf z_i \bm \phi^* q_i)(y_j - \bf z_j \bm \phi^* q_j)\\
    &\qquad + (1 - W)\bm \phi_1^* \sum_{i \in U} \sum_{j \in U} \Delta_{2ij}
    d_{2i} \bf x_i d_{2j}(y_j - \bf z_j \bm \phi^*_1 q_j)\\
  \end{align*}

  We can estimate the variance using

  \begin{align*}
    \hat V_{NNE} 
    &= (\hat{\bm \phi_1})^T \Var(\hat{\bf X_c}) \hat{\bm \phi_1} + 
    \sum_{i \in A_2} \sum_{j \in A_2} \frac{\Delta_{2ij}}{\pi_{2ij}\pi_{2i}\pi_{2j}} 
    (y_i - \bf z_i \hat{\bm \phi} q_i)(y_j - \bf z_j \hat{\bm \phi} q_j) \\
    &\qquad + (1 - W)\hat{\bm \phi_1} \sum_{i \in A_2} \sum_{j \in A_2} \frac{\Delta_{2ij}}{\pi_{2ij}}
    \frac{\bf x_i}{\pi_{2i}} \frac{(y_j - \bf z_j \hat{\bm \phi_1} q_j)}{\pi_{2j}}
  \end{align*}

  where 
  
  $$\hat{\bm \phi} =
  \begin{bmatrix}
    \hat{\bm \phi_1} \\ \hat \phi_2
  \end{bmatrix} = 
  \left(\sum_{i \in A_2} \frac{q_i}{g'(d_{2i})} 
  \begin{bmatrix}
    \bf x_i^2 / q_i & \bf x_i g(d_{2i}) / q_i \\
    \bf x_i g(d_{2i}) / q_i & g(d_{2i})^2
  \end{bmatrix}
  \right)^{-1}
  \sum_{i \in A_2} \frac{y_i}{g'(d_{2i})} 
  \begin{bmatrix} \bf x_i / q_i \\ g(d_i) \end{bmatrix}.
  $$
\end{theorem}

\begin{proof}
  From Theorem~\ref{thm:dc2}, we know that $\hat Y_{NNE}(\hat \lambda) = 
  \hat Y_{\ell, NNE}(\bm \lambda^*, \bm \phi^*) + O_p(Nn_2^{-1})$. Hence, the
  variance of $\hat Y_{NNE}(\hat{\bm \lambda})$ is 

  \begin{align*}
    \Var(\hat Y_{NNE}(\hat{\bm \lambda})) 
    &= \Var(\hat Y_{\ell, NNE}(\bm \lambda^*, \bm \phi^*) + O_p(Nn_2^{-1})) \\ 
    &= \Var\left(\sum_{i \in A_2} \hat w_{2i}(\bm \lambda^*) y_i + 
      \left(\bf T - \sum_{i \in A_2} \hat w_{2i}(\bm \lambda^*) \bf z_i
    q_i\right)\bm \phi^*\right)\\
    &= (\bm \phi_1^*)^T \Var(\hat{\bf X_c}) \bm \phi_1^* + 
    \sum_{i \in U} \sum_{j \in U} \frac{\Delta_{2ij}}{\pi_{2i}\pi_{2j}} 
    (y_i - \bf z_i \bm \phi^* q_i)(y_j - \bf z_j \bm \phi^* q_j)\\
    &\qquad+ 2 \Cov\left(\hat{\bf X_c} \bm \phi_1^*, \sum_{i \in A_2} 
      \frac{(y_i - \bf z_i \bm \phi^* q_i)}{\pi_{2i}}\right) \\
    &= (\bm \phi_1^*)^T \Var(\hat{\bf X_c}) \bm \phi_1^* + 
    \sum_{i \in U} \sum_{j \in U} \frac{\Delta_{2ij}}{\pi_{2i}\pi_{2j}} 
    (y_i - \bf z_i \bm \phi^* q_i)(y_j - \bf z_j \bm \phi^* q_j)\\
    &\qquad + (1 - W)\bm \phi_1^* \sum_{i \in U} \sum_{j \in U} \Delta_{2ij} 
    \frac{x_i}{\pi_{2i}} \frac{(y_j - \bf z_j \bm \phi^*_1 q_j)}{\pi_{2j}}\\
  \end{align*}

  where the last equality comes from the fact that $\hat{\bf X_c} = W\hat{\bf
  X_1} + (1 - W) \hat{\bf X_2}$. To have an unbiased estimator of the variance
  we can use:

  \begin{align*}
    \hat V_{NNE} 
    &= (\hat{\bm \phi_1})^T \Var(\hat{\bf X_c}) \hat{\bm \phi_1} + 
    \sum_{i \in A_2} \sum_{j \in A_2} \frac{\Delta_{2ij}}{\pi_{2ij}\pi_{2i}\pi_{2j}} 
    (y_i - \bf z_i \hat{\bm \phi} q_i)(y_j - \bf z_j \hat{\bm \phi} q_j) \\
    &\qquad + (1 - W)\hat \phi_1 \sum_{i \in A_2} \sum_{j \in A_2} 
    \frac{\Delta_{2ij}}{\pi_{2ij}} \frac{x_i}{\pi_{2i}} 
    \frac{(y_j - \bf z_j \hat{\bm \phi_1} q_j)}{\pi_{2j}}
  \end{align*}

  where 
  
  $$\hat{\bm \phi} =
  \begin{bmatrix}
    \hat{\bm \phi_1} \\ \hat \phi_2
  \end{bmatrix} = 
  \left(\sum_{i \in A_2} \frac{q_i}{g'(d_{2i})} 
  \begin{bmatrix}
    \bf x_i^2 / q_i & \bf x_i g(d_{2i}) / q_i \\
    \bf x_i g(d_{2i}) / q_i & g(d_{2i})^2
  \end{bmatrix}
  \right)^{-1}
  \sum_{i \in A_2} \frac{y_i}{g'(d_{2i})} 
  \begin{bmatrix} \bf x_i / q_i \\ g(d_i) \end{bmatrix}.
  $$

\end{proof}

\subsection{Simulation Study 2}

We run a simulation testing the proposed method. This is very similar to
Simulation 1. We have the following simulation setup:

$$
\begin{aligned}
X_{1i} &\stackrel{ind}{\sim} N(2, 1) \\
X_{2i} &\stackrel{ind}{\sim} Unif(0, 4) \\
X_{3i} &\stackrel{ind}{\sim} N(0, 1) \\
\varepsilon_i &\stackrel{ind}{\sim} N(0, 1) \\
Y_{i} &= 3 X_{1i} + 2 X_{2i} + \delta X_{3i} \varepsilon_i \\
\pi_{1i} &= n_1 / N \\
\pi_{2i} &= \max(\min(\Phi_3(X_{3i} - 2.5), 0.9), 0.01).
\end{aligned}
$$

where $\Phi_3$ is the CDF of a t-distribution with 3 degrees of freedom.
We consider a finite population of size $N = 10,000$ with the Phase 1 
sampling being a simple random sample (SRS) of size $n_1 = 1000$. The Phase 2
sample is a Poisson sample with an expected sample size of about 300.
In the Phase 1 sample, we observe 
$(X_1, X_2)$ while in the Phase 2 sample we observe $(X_1, X_2, Y)$. 
If $\delta = 0$ then there is no model misspecification. However, if $\delta =
1$, then there is some model misspecification. We estimate the
the parameter $\bar Y_N$ with four approaches:

\begin{itemize}
  \item[1.] HT-estimator (HT): $\hat Y_{HT} = N^{-1} \sum_{i \in A_2}
    \frac{y_i}{\pi_{2i}},$
  \item[2.] Regression estimator (Reg): Let $\hat Y_{NN, reg} = \hat Y_{HT} + (
    \hat{\bf X_c} - \hat{\bf X_{2, HT}}) \hat{\bm \beta_2}$ where $\hat Y_{HT} =
    \sum_{i \in A_2} d_{2i} y_i$, $\hat{\bf X_c} = W \hat{\bf X_{1, HT}} + (1 -
    W)\hat{\bf X_{2, HT}}$, $W = n_{1,\rm{eff}} / (n_{1,\rm{eff}} + n_{2, \rm{eff}})$,
    $\hat{\bf X_{1, HT}} = \sum_{i \in A_1} d_{1i} \bf x_{i}$, $\hat{\bf X_{2,
    HT}} = \sum_{i \in A_2} d_{2i} \bf x_{i}$, $\bf x_i = (1, x_{1i}, x_{2i})^T$
    and 

    $$ \hat{\bm \beta_2} = \left(\sum_{i \in A_2} \bf x_i \bf x_i^T\right)^{-1}
    \sum_{i \in A_2} \bf x_i y_i.$$

    Then $\hat{\bar Y}_{NN, reg} = \hat Y_{NN, reg} / N$.
  \item[3.] Debiased Calibration with Population Constraints (EstPop): This 
    solves 

  \begin{equation*}
  \hat w_2 = \argmin_w \sum_{i \in A_2} G(w_{2i}) q_i \text{ with } 
  \sum_{i \in A_2} w_{2i} \bf x_i = \sum_{i \in U} \bf x_i \text{ and } 
  \sum_{i \in A_2} w_{2i} g(d_{2i}) q_i = \sum_{i \in U} g(d_{2i}) q_i
  \end{equation*}

  \item[4.] Debiased Calibration with Estimated Population Constraints (Est):
    This solves Equation~\eqref{eq:nnopt} with $q_i = 1$.
\end{itemize}

In addition to estimating the mean parameter $\bar Y_N$, we also construct
variance estimates $\hat V(\hat Y)$ for each estimate $\hat Y$.

We run the simulation $B = 1000$ times for each of these methods and compute the
Bias ($E[\hat Y] - \bar Y_N$), the RMSE ($\sqrt{\Var(\hat Y - \bar Y_N)}$), a 95\%
empirical confidence interval ($\sum_{b = 1}^{1000} |\hat Y^{(b)} - \bar Y_N| \leq 
\Phi(0.975)\sqrt{\hat V(\hat Y^{(b)})^{(b)}}$), and a T-test that assesses the
unbiasedness of each estimator where $\hat Y^{(b)}$ is the result from the $b$th
simulation replicate. We also include the Monte Carlo variance of the estimated
value, ($V_{MC} = (B - 1)^{-1} \sum_{b = 1}^B (\hat Y^{(b)} - \bar Y_N)^2$), the
mean of the estimated variance, 
($\bar{\hat V} = B^{-1} \sum_{b = 1}^B \hat V^{(b)}$), and the relative bias of
the estimated variance, ($(V_{MC} - \bar{\hat V}) / V_{MC}$). The results are in
Table~\ref{tab:nndc0-mean} and Table~\ref{tab:nndc1-mean}.

\begin{table}[ht!]
  \centering
  
\begin{tabular}{lrrrrrrrr}
\toprule
Est & Bias & SE & RMSE & EmpCI & Ttest & MCVar & EstVar & RelBias\\
\midrule
Est & 0.000 & 0.123 & 0.123 & 0.954 & 0.01 & 0.015 & 0.016 & 0.040\\
EstPop & 0.000 & 0.053 & 0.053 & 0.943 & 0.30 & 0.003 & 0.003 & 0.023\\
HT & 0.024 & 0.565 & 0.566 & 0.954 & 1.33 & 0.320 & 0.325 & 0.017\\
Reg & 0.000 & 0.123 & 0.123 & 0.956 & 0.11 & 0.015 & 0.016 & 0.057\\
\bottomrule
\end{tabular}
\caption{This table shows the results of Simulation Study 2 with $\delta = 0$.
  It displays the Bias, RMSE, empirical 95\% confidence interval, a t-statistic
  assessing the unbiasedness, the Monte Carlo variance, mean estimated variance
  and relative bias of the variance estimator for the estimators: HT, Reg,
  EstPop, and Est.}
\label{tab:nndc0-mean}
\end{table}

\begin{table}[ht!]
  \centering
  
\begin{tabular}{lrrrrrrrr}
\toprule
Est & Bias & SE & RMSE & EmpCI & Ttest & MCVar & EstVar & RelBias\\
\midrule
Est & -0.001 & 0.124 & 0.124 & 0.961 & 0.13 & 0.015 & 0.017 & 0.132\\
EstPop & 0.000 & 0.066 & 0.066 & 0.937 & 0.14 & 0.004 & 0.004 & -0.001\\
HT & 0.019 & 0.529 & 0.529 & 0.956 & 1.14 & 0.280 & 0.285 & 0.017\\
Reg & -0.003 & 0.127 & 0.127 & 0.965 & 0.67 & 0.016 & 0.020 & 0.212\\
\bottomrule
\end{tabular}
\caption{This table shows the results of Simulation Study 2 with $\delta = 1$.
  It displays the Bias, RMSE, empirical 95\% confidence interval, a t-statistic
  assessing the unbiasedness, the Monte Carlo variance, mean estimated variance
  and relative bias of the variance estimator for the estimators: HT, Reg,
  EstPop, and Est.}
\label{tab:nndc1-mean}
\end{table}

\section{Topic 3: Multi-Source Two-Phase Sampling}

\subsection{Background}

When considering non-nested two-phase sampling, we focused on the case of having
two samples. Now, we consider incorporating more than two independent samples
together with a debiasing constraint. We consider the case in which we
want to estimate the $\theta = E[Y]$ and $Y$ is only observed in one sample.

Consider the setup in which we have independent samples $A_0, A_1, \dots, A_M$
where $Y$ is observed only in $A_0$ but $\bf X$ is observed in all of the
samples with elements $\bf X^{(0)}$ observed in $A_0$ and $\bf X^{(m)}$ observed
in $A_m$ for each $m = 1, \dots, M$. Like the non-nested case, we assume that
each survey is sampling independently from the same population sampling frame.

The traditional multi-source approach (\cite{kim2024statistics}) is to use
generalized least squares (GLS) to obtain an optimal estimator of $X$ from the
samples $A_1, \dots, A_M$. Then we can incorporate this information in the
estimation of $\theta$ by using the following estimate 

$$\hat \theta = \sum_{i \in A_0} \hat w_i y_i$$

where 

\begin{equation}\label{eq:mstps}
\hat w = \argmin_w \sum_{i \in A_0} G(w_i)q_i \text{ such that } 
\sum_{i \in A_0} w_i \bf x_i = \hat X_{\rm{GLS}}^{(0)}
\end{equation}

where $\hat{\bf X}_{\rm{GLS}}^{(0)}$ is the GLS estimate of $X$ for all of the
$X$-variables measured in sample $A_0$, and $G$ is a generalized entropy
function. 

\subsection{Methodology}

In order to have a design consistent estimator that incorporates information from
samples $A_0, A_1, \dots, A_M$, we want to add the debiasing constraints, 

\begin{equation}\label{eq:msdebconstr}
\sum_{i \in A_0} w_{i} \bf x_i q_i = \sum_{i \in U} g(d_i) q_i
\end{equation}

where $g(x) = \partial G(x) / \partial x$ and $d_i$ are the design weights for
sample $A_0$. This yields the optimization problem:

\begin{equation}\label{eq:dc3}
  \hat w = \argmin_w \sum_{i \in A_0} G(w_i) q_i \text{ such that }
  \widehat{\bf{T}} = \sum_{i \in A_0} w_i \bf z_i.
\end{equation}

where $\widehat{\bf{T}} = (\widehat{\bf X}_{\rm{GLS}}^{(0)},
\sum_{i \in U} g(d_i) q_i)^T$,
and $\bf z_i = ((\bf x_i^{(0)})^T, g(d_i)q_i)^T$. Then
the estimator is $\hat Y_{MS} = \sum_{i \in A_0} w_{i} y_i$ and we can construct
this estimate minimizing the Lagrangian:

$$
L(w_{0}, \bm \lambda) = \sum_{i \in A_0} G(w_{0i}) q_i + {\bm \lambda}^T
\left(\widehat{\bf{T}} - \sum_{i \in A_0} w_i \bf z_i\right)
$$

where $\bm \lambda$ are the Lagrange multipliers.

\subsection{Theoretical Results}

\begin{theorem}[Design Consistency]
  Let $\bm \lambda^*$ be the probability limit of $\hat{\bm \lambda}$.
  Under regularity conditions,

  $$\hat Y_{MS} = \hat Y_{\ell}(\bm \lambda^*, \bm \phi^*) + O_p(N / n_0)$$

  where

  $$
  \hat Y_\ell(\bm \lambda^*, \bm \phi^*) = \hat Y_{DCE} + \left(
    \widehat{\bf T} - \sum_{i \in A_0} \hat w_{0i}(\bm \lambda^*) \bf z_i q_i
  \right)\bm \phi^*
  $$

  and 

  $$
  \bm \phi^* = 
  \left[\sum_{i \in U} \frac{\pi_{0i} \bf z_i \bf z_i^T q_i}{g'(d_{0i})}\right]^{-1}
  \sum_{i \in U} \frac{\pi_{0i} \bf z_i y_i}{g'(d_{0i})}.
  $$
\end{theorem}

\begin{proof}
  This proof follows from the proof of Theorem~\ref{thm:dc2}. The only
  difference is that we have $\hat X_{\rm GLS}$ instead of $\hat X_c$.
\end{proof}

\begin{theorem}[Variance Estimation]
  Under regularity conditions,

  $$
  \begin{aligned}
    V(\hat Y_{MS}) 
    &= (\bm \phi^{*})^T\Var(\hat{\bf X}_{\rm GLS}^{(0)}) (\bm \phi^{*}) + 
      \sum_{i \in U} \sum_{j \in U} \frac{\Delta_{0ij}}{\pi_{0i}\pi_{0j}}
      (y_i - \bf z_i\bm \phi^{*}q_i)
      (y_j - \bf z_j\bm \phi^{*}q_j). 
  \end{aligned}
  $$

  We can estimate the variance with

  $$
  \begin{aligned}
    \hat V(\hat Y_{MS}) 
    &= (\hat{\bm \phi})^T \hat \Var(\hat{\bf X}_{\rm GLS}^{(0)}) (\hat{\bm \phi})
    + \sum_{i \in A_0} \sum_{j \in A_0} \frac{\Delta_{0ij}}{\pi_{0ij}\pi_{0i}\pi_{0j}}
      (y_i - \bf z_i \hat{\bm \phi}q_i)
      (y_j - \bf z_j \hat{\bm \phi}q_j). 
  \end{aligned}
  $$

\end{theorem}

\begin{proof}
  This result follows the same argument as Theorem~\eqref{thm:nnve}. The result
  holds because each sample $A_m$ is independent from the other $A_{m'}$.
\end{proof}

\subsection{Simulation Study 3}

We have the following superpopulation model with $N = 10000$ elements:

$$
\begin{aligned}
X_{1i} &\stackrel{ind}{\sim} N(2, 1) \\
X_{2i} &\stackrel{ind}{\sim} Unif(0, 4) \\
X_{3i} &\stackrel{ind}{\sim} N(5, 1) \\
Z_i &\stackrel{ind}{\sim} N(0, 1) \\
\varepsilon_i &\stackrel{ind}{\sim} N(0, 1) \\
Y_{i} &= 3 X_{1i} + 2 X_{2i} + \delta Z_{i} + \varepsilon_i \\
\pi_{0i} &= \min(\max(\Phi(Z_i - 2), 0.02), 0.9)\\
\pi_{1i} &= n_1 / N\\
\pi_{2i} &= \Phi(X_{2i} - 2) \\
\end{aligned}
$$

Like the previous simulation studies, when $\delta = 1$, there is model
misspecification for the outcome model because we observe the following columns
in each sample

\begin{table}[ht!]
  \centering
  \begin{tabular}{lrrrr}
    \toprule
    Sample & $X_1$ & $X_2$ & $X_3$ & $Y$ \\
    \midrule
   $A_0$  & \checkmark     & \checkmark     & \checkmark     &  \checkmark \\    
   $A_1$  & \checkmark     &       & \checkmark     &    \\  
   $A_2$  & \checkmark     & \checkmark     &       &    \\  
   \bottomrule
  \end{tabular}
\end{table}

For the sampling mechanism both $A_0$ and $A_2$ are selected using a Poisson
sample with response probabilities $\pi_{0i}$ and $\pi_{1i}$ respectively. The
sample $A_1$ is a simple random sample with $n_1 = 2000$. We compare four
different estimators for $\theta = E[Y]$.

\begin{itemize}
  \item[1.] Horvitz-Thompson estimator (HT): $\hat Y = N^{-1} \sum_{i \in A_0}
    \frac{y_i}{\pi_{0i}}$,
  \item[2.] Non-nested regression (NNReg): This is the non-nested regression from
    Equation~\eqref{eq:nnopt} with only using information from Samples $A_0$ and
    $A_1$,
  \item[3.] Multi-Source proposed (Est): This is the proposed estimator from
    Equation~\eqref{eq:dc3}, and
  \item[4.] Multi-Source population (EstPop): This is the proposed estimator with
    using the true value of $T_1$ from the population.
\end{itemize}

The simulation results are displayed in Table~\ref{tab:msdc0-mean} and
Table~\ref{tab:msdc1-mean}.

\begin{table}[ht!]
  \centering
  
\begin{tabular}{lrrrrrrrr}
\toprule
Est & Bias & SE & RMSE & EmpCI & Ttest & MCVar & EstVar & RelBias\\
\midrule
Est & -0.004 & 0.091 & 0.091 & 0.943 & 1.55 & 0.008 & 0.008 & -0.087\\
EstPop & -0.003 & 0.058 & 0.058 & 0.932 & 1.40 & 0.003 & 0.003 & -0.073\\
HT & -0.006 & 0.596 & 0.596 & 0.951 & 0.34 & 0.356 & 0.353 & -0.008\\
NNReg & -0.005 & 0.099 & 0.099 & 0.941 & 1.45 & 0.010 & 0.009 & -0.080\\
\bottomrule
\end{tabular}
\caption{This table shows the results of Simulation Study 3 with $\delta = 0$.
  It displays the Bias, RMSE, empirical 95\% confidence interval, a t-statistic
  assessing the unbiasedness, the Monte Carlo variance, mean estimated variance
  and relative bias of the variance estimator for the estimators: HT, NNReg,
  EstPop, and Est.}
\label{tab:msdc0-mean}
\end{table}

\begin{table}[ht!]
  \centering
  
\begin{tabular}{lrrrrrrrr}
\toprule
Est & Bias & SE & RMSE & EmpCI & Ttest & MCVar & EstVar & RelBias\\
\midrule
Est & -0.002 & 0.097 & 0.097 & 0.938 & 0.63 & 0.009 & 0.009 & -0.051\\
EstPop & 0.000 & 0.068 & 0.068 & 0.945 & 0.07 & 0.005 & 0.004 & -0.049\\
HT & -0.006 & 0.569 & 0.569 & 0.947 & 0.34 & 0.324 & 0.321 & -0.008\\
NNReg & -0.002 & 0.107 & 0.106 & 0.974 & 0.49 & 0.011 & 0.015 & 0.345\\
\bottomrule
\end{tabular}
\caption{This table shows the results of Simulation Study 3 with $\delta = 1$.
  It displays the Bias, RMSE, empirical 95\% confidence interval, a t-statistic
  assessing the unbiasedness, the Monte Carlo variance, mean estimated variance
  and relative bias of the variance estimator for the estimators: HT, NNReg,
  EstPop, and Est.}
\label{tab:msdc1-mean}
\end{table}

As expected MSPop has the lowest RMSE because it also uses the population
totals. It seems like MSEst and MSReg are almost equivalent, which is good
because they are also supposed to be identical. The MSEst estimator outperforms
NNReg because it also uses information from $A_2$, even though it implicitly
uses a regression estimator with $X_3$ as a covariate, which is unnecessary.

\section{Real-Data Analysis}

We apply our method to Medicare Current Beneficiary Survey (MCBS). The MCBS
is an ongoing representative national survey of the population of people on
Medicare in the United States, and in 2020, NORC started issuing an MCBS
supplement to identify the impact of COVID-19 on the Medicare population
\cite{norc2020mcbs}. This analysis looks at whether
respondents had obtained at least one dose of a COVID-19 vaccine.

Assessing the vaccine rate of Medicare patients is important from a public
health perspective. Older people\footnote{Medicare is available to people ages
65 years and older and select designated populations between the ages of 18 and
65. However, the vast majority of Medicare patients are 65 years old or older.}
are known to have a higher all-cause mortality from
COVID-19 (\cite{bonanad2020effect}). Thus, understanding how older people get
vaccinated can help save lives and reduce medical costs.

The MCBS is a rotating panel survey in which participants are interviewed up to
three times within a four-year period (\cite{cms2021mcbs}). The participants
are selected from Medicare enrollment data and enter a cohort in the fall round
of the survey. They then respond to the winter, summer,
and fall rounds of the survey for three consecutive years before exiting the
survey in the winter round of year four (\cite{cms2021mcbs}). Each year a new
cohort is added. Participants are selected based on a three-stage cluster sample
design. The first stage of the sample consists of major metropolitan areas and
groups of rural counties (\cite{cms2021mcbs}). The second stage are census
tracts within the primary sampling units. Finally, Medicare beneficiaries are
selected using a stratified systematic sample with random starts 
(\cite{cms2021mcbs}). Medicare beneficiaries are selected from strata based on
age and whether the beneficiary is Hispanic. Strata with Hispanic beneficiaries
and strata with beneficiaries over the age of 85 and under the age of 65 are
oversampled.
The COVID supplement was added to the traditional MCBS survey in March of
2020 to assess the impact of COVID on the Medicare
population (\cite{cms2021covid}). These questions about vaccines were added to
participants who were enrolled in the Summer 2021 cohort.

We combine the 2021 Summer MCBS survey with data from the 2021 American
Community Survey (ACS)
issued by the U.S. Census Bureau along with administrative data about people on
Medicare from the Center of Medicare and Medicaid Services (CMS). Like the MCBS
survey, the ACS data contains information about the overall percentage of
people's sex, race and ethnicity, education, and marital status for people on
Medicare. We combine this information with official totals of people on Medicare
by sex from the CMS. A complete display of the data available can be seen in
Table~\ref{tab:mcbscols}.

\begin{table}[ht!]
  \label{tab:mcbscols}
  \centering
  \begin{tabular}{lrrrrr}
  \toprule
  Data & Sex & Race and Ethnicity & Education & Marital Status & COVID Vaccine Dose \\
  \midrule
  MCBS & \checkmark & \checkmark & \checkmark & \checkmark & \checkmark \\
  ACS & \checkmark & \checkmark & \checkmark & \checkmark &  \\
  CMS & \checkmark &  &  &  & \\
  \bottomrule
  \end{tabular}
  \caption{This table shows the data available to us in each data set. A
  checkmark in a particular column indicates that the data set listed in the
  left column contains information about the participant characteristic in the
  given column.}
\end{table}

To make the results between the ACS and MCBS compatible, we recode the 
ACS variables with the map found in Table~\ref{tab:acsvars}.

\begin{table}[ht!]
  \centering
  \label{tab:acsvars}
  \begin{tabular}{lr}
  \toprule
  Variable & Consolidated ACS Variables \\
  \midrule
  Education & SCHL:  01-15, 16-19, 20-24 \\
  Marital Status & MAR: 1, 2, 3-4, 5 \\
  \bottomrule
  \end{tabular}
  \caption{This table shows how we consolidated ACS variables. The right column
  displays the ACS category code and the values used in a particular category to
  match a MCBS category. For example, SCHL: 01-15 means that the values of 01 -
  15 of the SCHL ACS column were combined to match an individual MCBS value.}
\end{table}

While the MCBS data is largely free of missing values, there are a small number
that need to be imputed for the $Y$ variable about whether someone received at
least one COVID vaccination dose. This MCBS data was updated to include
information about if a participant had received a second vaccine dose and then
later to include how many total vaccine doses patients had recieved. If the
participant had received at least one more dose and their answer was missing (because
they refused to answer, did not know, or was just missing) the missing value was
imputed to say that they yes, they had received a COVID vaccine dose. Otherwise,
the missing value was imputed to say no, the participant had not received a
COVID vaccine dose.

We analyze the data with three approaches: a Horvitz-Thompson method, a
regression algorithm, and our preposed multi-source two-phase sampling technique. 
The Horvitz-Thompson estimator uses the design weights, the MCBS observation if
an individual received a COVID-19 vaccine dose, and the population total from
the CMS about the total number of people on Medicare in 2021. The regression
estimator uses all of this and a population level total for the number of males.
Our multi-source algorithm also includes information from the ACS about
education and marital status. Due to the fact that we do not observe the
probability that an individual will be selected into the MCBS survey within the
population of all Medicare patients, we have to estimate the term 
$\sum_{i \in U} g(d_i)$ from Equation~\ref{eq:msdebconstr}. This technique is
adopted from \cite{kwon2024debiased} and described in more details in the
appendix.
The results of the real data analysis are found in Table~\ref{tab:mcbsres}.

\begin{table}[ht!]
  \centering
  \label{tab:mcbsres}
  \begin{tabular}{lrr}
    \toprule
    Method & Point Estimate & Standard Error \\
    \midrule
    HT  & 0.817 & 0.01223 \\
    Reg & 0.877 & 0.00439 \\
    Est & 0.863 & 0.00436 \\ 
    \bottomrule
  \end{tabular}
  \caption{This table shows the point estimate and standard error of the
  percentage of people on Medicare who had at least one COVID-19 vaccine in
  2021 as estimated by a Horvitz-Thompson estimator (HT), a regression estimator
  (Reg) and the proposed multi-source estimator (Est).}
\end{table}

\section{Conclusion}

Overall, using debiased calibration for generalized two-phase sampling seems to
be a promising approach for combining multiple samples efficiently. Not only can
this method be used to combine surveys focusing on the same population frame,
but because it only does not require individual responses levels from the
supplemental surveys, this method can be used to construct domain level
estimates as exemplified in the real data analysis of Medicare patients.

\bibliographystyle{apa}
\bibliography{references.bib}

\section{Appendix A: Estimating the Population Weights}

In Equation~\ref{eq:primal}, Equation~\ref{eq:nnopt}, and
Equation~\ref{eq:msdebconstr}, we see the need for knowing $\sum_{i \in U}
g(d_i)$. (In the case of Equation~\ref{eq:primal}, we only need $\sum_{i \in
A_1} g(d_{2i|1})$.) However, this quantity is not always known. When it is
unknown, we need to estimate it.

We follow the approach of \cite{kwon2024debiased}. Define 

$$N \alpha = \sum_{i \in U} g(d_i).$$

Assuming $N$ is known (or that we can estimate it), we need to estimate
$\alpha$. Since our estimating equation is convex for any fixed $\alpha$, we
adopt a two-step procedure for which we estimate $\alpha$ and $\bf w$. In this
approach we have the following loss function for the case of non-nested sampling:

\begin{equation}\label{eq:nnalpha}
  (\hat \alpha, \hat w_2) = 
  \argmin_{\alpha} \argmin_{w_2} \sum_{i \in A_2} \{G(w_{2i})\} - NK(\alpha),
  \text{ such that } \sum_{i \in A_2} w_{2i} \bf x_i = \hat{\bf X_c},
  \text{ and } \sum_{i \in A_2} w_{2i} g(d_{2i}) = N\alpha.
\end{equation}

The choice of $K(\alpha)$ is up to the analyst. We consider $K(\alpha) =
\alpha$.

\end{document}
