\section*{Estimating the Response Model}

\begin{itemize}
  \item In most (critically \textit{not} Simulation 4) of the simulations of the
    previous section, we used the oracle weights when estimating our proposed
    method. The reasoning for this was straightforward{--}we wanted to ensure
    that the response model was correctly specified and the best case scenario
    is to use the oracle weights which we did.
    
  \item This section focuses more on estimating these response weights. Instead
    of focusing on the proposed method, we will actually be working on
    estimation of the complete case IPW estimator. This model is less complex
    and will thus make it easier to understand where we are making mistakes.

  \item We use the following simulation study:
    \begin{align*}
      x_i &\stackrel{iid}{\sim} N(0, 1)\\
      \varepsilon_{1i} &\stackrel{iid}{\sim} N(0, 1)\\
      \varepsilon_{2i} &\stackrel{iid}{\sim} N(0, 1)\\
      y_{1i} &= x_i + \varepsilon_{1i} \\
      y_{2i} &= x_i + \varepsilon_{2i} \\
    \end{align*}

    To select a missingness pattern for each $i$, we have sequence: first, we
    select the first variable to observe (or neither), then we either select the
    second variable or we do not. In the first step, we select $R_{1i} = 1$ with
    probability $0.4$, $R_{2i} = 1$ with probability $0.4$ and neither variable
    with probability $0.2$. For the second step, we have the probability of
    observing the other variable be $\logistic(x_i)$. This yields the following:

    \begin{table}[h!]

\caption{True Value is 0. Cor(Y1, Y2) = 0}
\centering
\begin{tabular}[t]{lrrrr}
\toprule
algorithm & bias & sd & tstat & pval\\
\midrule
oracle & 0.000 & 0.032 & -0.318 & 0.375\\
ipw.or & 0.002 & 0.067 & 1.140 & 0.127\\
ipw.0 & 0.083 & 0.032 & 116.911 & 0.000\\
\bottomrule
\end{tabular}
\end{table}

    % Extension?: estimate IPW with \pi_{2|1} and \pi_{1|2}.

    The problem with this is that our estimate is biased. For one particular
    realization of the simulation, here is the distribution of the difference
    between the estimated and true probabilities of $\pi_{11}$:

    \includegraphics[width = 0.7\linewidth]{diffhist.png}
  
\end{itemize}

\newpage

