% Unofficial ISU Poster template.
% A fork of https://github.com/anishathalye/gemini
% also refer to https://github.com/yhkwon/gemini-isu

% .pptx or illustrator version is available on https://www.print.iastate.edu/PrintingResources/DesignTemplates/

\documentclass[final]{beamer}

% ====================
% Packages
% ====================

\usepackage[T1]{fontenc}
\usepackage{lmodern}
\usepackage[size=custom,width=120,height=72,scale=1.0]{beamerposter}
\usetheme{gemini}
\usecolortheme{gemini}
\usepackage{graphicx}
\usepackage{booktabs}
\usepackage{tikz}
\usepackage{pgfplots}
\usepackage{amsmath}
\usepackage{bm}
\pgfplotsset{compat=1.14}
\usepackage{anyfontsize}
\usepackage[backend=biber, style=bwl-FU]{biblatex}
\addbibresource{poster.bib}

\DeclareMathOperator*{\argmin}{arg\,min}
\DeclareMathOperator*{\argmax}{arg\,max}
\DeclareMathOperator*{\suchthat}{such\; that}
% ====================
% Lengths
% ====================

% If you have N columns, choose \sepwidth and \colwidth such that
% (N+1)*\sepwidth + N*\colwidth = \paperwidth
\newlength{\sepwidth}
\newlength{\colwidth}
\setlength{\sepwidth}{0.025\paperwidth}
\setlength{\colwidth}{0.3\paperwidth}

\newcommand{\separatorcolumn}{\begin{column}{\sepwidth}\end{column}}

% ====================
% Title
% ====================

\title{\hspace{25cm}Debiased Calibration for Generalized Two-Phase Sampling}

%\author{Your Name}
\author{Caleb Leedy \inst{1} \and Jae-Kwang Kim \inst{1}}

%\institute[shortinst]{Some Institute}
\institute[shortinst]{\inst{1} Iowa State University}

% ====================
% Footer (optional)
% ====================

\footercontent{
  \href{https://github.com}{Github: https://github.com/calebleedy} \hfill
JSM Poster Session \hfill
  \href{mailto:cleedy@iastate.edu}{cleedy@iastate.edu}}
% (can be left out to remove footer)

% ====================
% Logo (optional)
% ====================

% use this to include logos on the left and/or right side of the header:
% \logoright{\includegraphics[height=7cm]{logo1.pdf}}
% \logoleft{\includegraphics[height=7cm]{logo2.pdf}}

% ====================
% Body
% ====================

\begin{document}
\addtobeamertemplate{headline}{}
{
    \begin{tikzpicture}[remember picture,overlay]
        \node [anchor=north west, inner sep=3cm] at ([xshift=0.0cm,yshift=1.0cm]current page.north west){\includegraphics[height=2.5cm]{logos/watermark.png}}; % Iowa State University watermark
        % ISU watermark can be found on https://www.brandmarketing.iastate.edu/wordmark/primary/
        \node [anchor=north west, inner sep=3cm] at ([xshift=0.0cm,yshift=-2.5cm]current page.north west){\huge \textbf{Statistics Department}}; % Department name
        %\node [anchor=north east, inner sep=3cm] at ([xshift=0.0cm,yshift=1.0cm]current page.north east){\includegraphics[height=5cm]{logos/logo.png}}; % Use this if you want to add a logo on your right top corner.
    \end{tikzpicture}
}

\begin{frame}[t]
\begin{columns}[t]
\separatorcolumn

\begin{column}{\colwidth}

  \begin{block}{Introduction}

    In classical two-phase sampling, one first selects a sample $A_1$ from a
    finite population $U$ of size $N$, and observes $({\bf X}_i)_{i = 1}^{n_1}$.
    Then one selects a sample $A_2$ from $A_1$ and observes 
    $({\bf X}_i, Y_i)_{i = 1}^{n_2}$. We take the
    framework of two-phase sampling and view data integration as specific case
    of observing a variable of interest $Y$ within a single data set while we
    also observe common covariates $({\bf X}_i)$ between both the data set with
    $Y$ as well as an outside auxilary sample. This is an important practical
    problem \cite{yang2020statistical}, 
    \cite{dagdoug2023model}. We focus on the case:

    \begin{itemize}
      \item Where all of the surveys are probability samples, 
      \item Where we only have summary level information instead of individual
        observation values,
      \item When we want the estimate the finite population total of $Y$,
        $$Y_T = \sum_{i \in U} y_i.$$
    \end{itemize}

    \heading{Notation}

    \begin{itemize}
      \item Let $\pi_{1i}$ be the probability that element $i$ is selected into
        the Phase 1 sample, $A_1$.
      \item Let $\pi_{2i|1}$ be the probability that element $i$ is selected into
        the Phase 1 sample, $A_2$ conditional on the fact that $i \in A_1$.
      \item Let $d_{1i} = 1 / \pi_{1i}$ and $d_{2i|1} = 1 / \pi_{2i|1}$.
      \item We use $\delta_{1i}$ and $\delta_{2i}$ to indicate if an observation
        in contained within $A_1$ and $A_2$ respectively.
    \end{itemize}

  \end{block}

  \begin{block}{Goal}

    We want a two-phase sampling framework to

    \begin{itemize}
      \item[1.] Combine information from multiple data sources,
      \item[2.] In a way that is efficient, and
      \item[3.] Approximately design unbiased.
    \end{itemize}

  \end{block}

  % NOTE: Use the "alertblock" to highlight a particular section in green.
  \begin{block}{Comparable Methods}

    \begin{itemize}
      \item The \textbf{double expansion estimator} of \cite{kott1997can}
        is a Horvitz-Thompson like estimator that is design unbiased, but
        inefficient:

        $$
        \hat Y_{\rm{DEE}} = \sum_{i \in A_2} \frac{y_i}{\pi_{1i}\pi_{2i|1}}.
        $$
      \item The \textbf{two-phase regression estimator} is approximately design
        unbiased but not as efficient as our proposed method:
        $$
        \hat Y_{\rm{Reg}} = 
        \sum_{i \in A_1} \frac{{\bf{x}}_i \hat{\bm{\beta}}_q}{\pi_{1i}} + 
        \sum_{i \in A_2} \frac{1}{\pi_{1i}\pi_{2i|1}}(y_i - 
        {\bf{x}}_i \hat{\bm{\beta}}_q)
        $$

        where $q_i = q(\bf x_i)$ and 

        $$
        \hat{\bm \beta}_q = \left(\sum_{i \in A_2} 
        \frac{{\bf{x}}_i {\bf{x}}_i'}{\pi_{1i} q_i}\right)^{-1} 
        \sum_{i \in A_2} \frac{{\bf{x}}_i y_i}{\pi_{1i} q_i}.
        $$

    \end{itemize}

  \end{block}

\end{column}

\separatorcolumn

\begin{column}{\colwidth}

  \begin{block}{Methodology}

    \heading{Generalized Calibration}

    \begin{itemize}
      \item In a seminal paper, \cite{deville1992calibration} generalized the
        regression estimator to other loss functions besides squared-error loss
        for a sample $A$ with auxilary information about $X$. Their generalized
        loss function minimizes,

        $$
        \sum_{i \in A} G(w_{i}, d_{i}) \suchthat \sum_{i \in
        A} d_{i} w_{i} {\bf x_i} = \sum_{i \in U} {\bf x_i}.
        $$

        for a non-negative, strictly convex function with respect to $w$
        function $G$, with a minimum at $g(w_i, d_i) = \frac{\partial G}{\partial
        w}$ defined on an interval containing $d_{i}$ with $g(w_i, d_i)$
        continuous.
    \end{itemize}

    \heading{Calibration with Generalized Entropy}

    \begin{itemize}
      \item Recently, \cite{kwon2024debiased} proposed a calibration estimator that
        uses a generalized entropy function $G(w)$ (\cite{gneiting2007strictly})
        instead of the generalized loss function $G(w, d)$ of
        \cite{deville1992calibration}. They separate the bias calibration from
        the minimization term and solve the following equation for estimated
        sample weights:

        $$
        \hat w_{i} = \argmin_{w} \sum_{i \in A} G(w_i)
        \suchthat \sum_{i \in A} w_{i} {\bf{x}}_i = 
        \sum_{i \in U} {\bf{z}}_i,
        \sum_{i \in A} w_i g(d_i) = \sum_{i \in U} g(d_i) .
        $$

      \item Our proposed method extends their result to two-phase sampling.
    \end{itemize}

  \end{block}

  \begin{alertblock}{Proposal}

    Let ${\bf{z}}_i = ({\bf x}_i / q_i, g(d_{2i|1}))^T$, the proposed debiased
    calibration estimator is 
    \vspace{-1.0cm}

    \begin{equation}
    \hat Y_{\rm{DCE}} = \sum_{i \in A_2} d_{1i} \hat w_{2i|1} y_i
    \end{equation}

    \vspace{-1.0cm}

    where 
    \vspace{-0.5cm}

    \begin{equation}\label{eq:wmin}
    \hat w_{2i|1} = \argmin_{w_{2|1}} \sum_{i \in A_2} w_{1i} G(w_{2i|1}) q_i
    \suchthat \sum_{i \in A_2} d_{1i} w_{2i|1} {\bf{z}}_i q_i = 
    \sum_{i \in A_1} d_{1i} {\bf{z}}_i q_i.
    \end{equation}
    
    \vspace{-0.5cm}
  \end{alertblock}

  \begin{exampleblock}{Theoretical Results: Asymptotic Design Consistency}

  %\begin{theorem}[Design Consistency]\label{thm:dc1}
    Let $\bm \lambda^*$ be the probability limit of $\hat{\bm \lambda}$,
    where $\hat{\bm \lambda}$ are the corresponding Lagrange multipliers from
    Equation~\ref{eq:wmin}.
    Under some regularity conditions, $\bm \lambda^* = ({\bf 0}^T, 1)^T$ and

    $$\hat Y_{\rm{DCE}} = \hat Y_\ell(\bm \lambda^*, \bm \phi^*) + O_p(N / n_2)$$

    where
    \vspace{-1.3cm}

    $$\hat Y_{\ell}(\bm \lambda^*, \bm \phi^*) = \hat Y_{\rm{DEE}}+ 
    \left(\sum_{i \in A_1} d_{1i} {\bf z}_i q_i - \sum_{i \in A_2} d_{1i} 
    \pi_{2i \mid 1}^{-1}  {\bf z}_i q_i\right)\bm \phi^*$$

    \vspace{-1.3cm}
    and
    \vspace{-0.3cm}

    $$
    \bm \phi^* = 
    \left[\sum_{i \in U} \frac{\pi_{2i|1}{\bf z}_i {\bf z}_i^T q_i}{g'(d_{2i|1})}\right]^{-1}
    \sum_{i \in U} \frac{\pi_{2i|1}{\bf z}_i y_i}{g'(d_{2i|1})}.
    $$

    \vspace{-0.5cm}
  %\end{theorem}

  \end{exampleblock}

\end{column}

\separatorcolumn

\begin{column}{\colwidth}

  \begin{block}{Simulation Study}

  For a finite population of size $N = 10,000$, and $n_1 = 1000$,

  \begin{itemize}
    \item $X_{1i} \stackrel{ind}{\sim} N(2, 1)$,
      $X_{2i} \stackrel{ind}{\sim} \rm{Unif}(0, 4)$,
      $Z_{i} \stackrel{ind}{\sim} N(0, 1)$,
      $\varepsilon_i \stackrel{ind}{\sim} N(0, 1)$
    \item $Y_{i} = 3 X_{1i} + 2 X_{2i} + 0.5 Z_i + \varepsilon_i$
    \item $\pi_{1i} = n_1 / N$,
      $\pi_{2i|1} = \max(\min(\Phi_3(z_{i} - 1), 0.7), 0.02)$.
  \end{itemize}

  where $\Phi_3$ is the CDF of a t-distribution with 3 degrees of freedom. We
  compare the following algorithms:

  \begin{itemize}
    \item[1.] Double Expansion Estimator (DEE)
    \item[2.] Two-Phase Regression estimator (TP-Reg)
    \item[3.] Debiased Calibration with Population Constraints (DC-Pop): This 
      solves 

    \begin{equation*}
      \argmin_{w_{2|1}} \sum_{i \in A_2} d_{1i} G(w_{2i}) \suchthat
      \sum_{i \in A_2} d_{1i} w_{2i|1} {\bf z}_i = \sum_{i \in U} {\bf z}_i.
    \end{equation*}

    \item[4.] Debiased Calibration with Estimated Population Constraints (DC-Est):
      This solves Equation~\eqref{eq:wmin} with $q_i = 1$.
  \end{itemize}

  \begin{table}[ht!]
    \centering
  
\begin{tabular}{lrrrr}
\toprule
Est & Bias & RMSE & EmpCI & Ttest\\
\midrule
$\pi^*$ & 0.003 & 0.251 & 0.936 & 0.344\\
TP-Reg & 0.004 & 0.149 & 0.949 & 0.793\\
DC-Pop & 0.001 & 0.015 & 0.961 & 2.425\\
DC-Est & 0.005 & 0.148 & 0.946 & 1.014\\
\bottomrule
\end{tabular}
  % This table was generated from ../../src/explore/proto-dctpmcmc.R
  \caption{This table shows the results of the simulation study. It displays the
  Bias, RMSE, empirical 95\% confidence interval, and a t-statistic assessing the
  unbiasedness of each estimator for the estimators: DEE, TP-Reg, DC-Pop, and
  DC-Est.}
  \label{tab:tpdc-mean}
  \end{table}
  \end{block}

  \begin{block}{Extensions}

    We also consider the following extensions:

    \begin{itemize}
      \item Non-nested two-phase sampling: when $A_1$ and $A_2$ are independent.
      \item Multi-source sampling: when $Y$ is contained in a sample $A_0$ that
        shares common covariates $\bf X$, with samples $A_1$, $A_2$, \dots, $A_M$.
    \end{itemize}
  \end{block}

  \vspace{-1.0cm}

  \begin{block}{References}

    %\footnotesize{\bibliographystyle{plainnat}\bibliography{poster}}
    \printbibliography

  \end{block}

\end{column}

\separatorcolumn
\end{columns}
\end{frame}

\end{document}

