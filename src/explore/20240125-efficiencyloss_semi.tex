\documentclass[12pt]{article}

\usepackage{amsmath, amssymb, mathrsfs, fancyhdr}
\usepackage{syntonly, lastpage, hyperref, enumitem, graphicx}
\usepackage{biblatex}
\usepackage{booktabs}
\usepackage{float}

\addbibresource{references.bib}

\hypersetup{colorlinks = true, urlcolor = black}

\headheight     15pt
\topmargin      -1.5cm   % read Lamport p.163
\oddsidemargin  -0.04cm  % read Lamport p.163
\evensidemargin -0.04cm  % same as oddsidemargin but for left-hand pages
\textwidth      16.59cm
\textheight     23.94cm
\parskip         7.2pt   % sets spacing between paragraphs
\parindent         0pt   % sets leading space for paragraphs
\pagestyle{empty}        % Uncomment if don't want page numbers
\pagestyle{fancyplain}

\newcommand{\MAP}{{\text{MAP}}}
\newcommand{\argmax}{{\text{argmax}}}
\newcommand{\argmin}{{\text{argmin}}}
\newcommand{\Cov}{{\text{Cov}}}
\newcommand{\Var}{{\text{Var}}}
\newcommand{\logistic}{{\text{logistic}}}

\renewcommand{\arraystretch}{1.2}

\begin{document}

\lhead{Caleb Leedy}
\chead{Non-monotone Missingness}
%\chead{STAT 615 - Advanced Bayesian Methods}
%\rhead{Page \thepage\ of \pageref{LastPage}}
\rhead{\today}

\section*{The Efficiency Loss of Semiparametric Inference}

Inspired by Dr. Fuller's note, I set out to see if we could assess the amount 
of efficiency loss due to using semiparametric inference. First, let's define 
some notation.

\begin{table}[ht!]
\centering
\begin{tabular}{lrrr}
  \toprule
  Segment & Quantity of Interest & Estimator & Coefficient \\
  \midrule
  $A_{00}$ & $E[g \mid X]$ & $\hat \gamma_{11}$ & $c_{11}$ \\
  $A_{10}$ & $E[g \mid X]$ & $\hat \gamma_{21}$ & $c_{21}$ \\
  $A_{10}$ & $E[g \mid X, Y_1]$ & $\hat \gamma_{22}$ & $c_{22}$ \\
  $A_{01}$ & $E[g \mid X]$ & $\hat \gamma_{31}$ & $c_{31}$ \\
  $A_{01}$ & $E[g \mid X, Y_2]$ & $\hat \gamma_{33}$ & $c_{33}$ \\
  $A_{11}$ & $E[g \mid X]$ & $\hat \gamma_{41}$ & $c_{41}$ \\
  $A_{11}$ & $E[g \mid X, Y_1]$ & $\hat \gamma_{42}$ & $c_{42}$ \\
  $A_{11}$ & $E[g \mid X, Y_2]$ & $\hat \gamma_{43}$ & $c_{43}$ \\
  $A_{11}$ & $E[g \mid X, Y_1, Y_2]$ & $\hat \gamma_{44}$ & $c_{44}$ \\
  \bottomrule
\end{tabular}
\caption{This table shows the relationship between different quantities of
interest in each segment, their estimators, and the corresponding coefficients
of their estimators.}
\label{tab:reftab}
\end{table}

This is a similar framework to understanding the class of estimators 
that we discussed in \verb|optest_update.pdf|, where we write the estimator 
as a weighted average of different expectation with functional coefficients.

\[\hat \theta = \frac{\delta_{11}}{\pi_{11}}g(Z) + \beta_0(\delta, c_0)E[g(Z)
\mid X] + \beta_1(\delta, c_1)E[g(Z) \mid X, Y_1] + \beta_2(\delta, c_2) E[g(Z)
\mid X, Y_2].\]

However, now we write the estimator as 

\[\hat \theta = \sum_{k,t} c_{kt}\hat \gamma_{kt}\]

where $\hat \gamma_{kt} = \frac{\delta_{ij}}{\pi_{ij}}E[g \mid G_{ij}(X, Y_1,
Y_2)]$ and $ij$ corresponds to the segment $A_{ij}$ associated with $\hat
\gamma_{kt}$ in Table~\ref{tab:reftab}.

\newpage

\subsection*{Different Classes of Models}

As noted by Dr. Fuller, we can understand parametric estimation as solving 
the following constrained optimization problems:

\begin{align*}
  \min &\Var\left(\sum_{k, t} c_{kt} \hat \gamma_{kt}\right) \text{ such that } 
  \sum_{k,t} c_{kt} = 1 \\
       &\text{or} \\
  \min &\Var\left(\sum_{k, t} c_{kt} \hat \gamma_{kt}\right) \text{ such that } 
  \sum_{(k,t): (k,t) \neq (4, 4)} c_{kt} = 0 \text{ and } c_{44} = 1. \\
\end{align*}

To be robust to the outcome model, we can construct a similar optimization
problem with different constraints:

\begin{align*}
  \min \Var\left(\sum_{k, t} c_{kt} \hat \gamma_{kt}\right) \text{ such that }& 
  c_{11} + c_{21} + c_{31} + c_{41} = 0, c_{22} + c_{42} = 0, c_{33} + c_{43} = 0,\\
       &\text{and } c_{44} = 1.
\end{align*}

Likewise to be robust to the response model, we can have the problem:

\begin{align*}
  \min \Var\left(\sum_{k, t} c_{kt} \hat \gamma_{kt}\right) \text{ such that }& 
  c_{11} = \pi_{00}, c_{21} + c_{22} = \pi_{10}, c_{31} + c_{33} = \pi_{01}, \\
       &\text{ and } c_{41} + c_{42} + c_{43} + c_{44} = \pi_{11}.
\end{align*}

To be double robust (robust to the outcome and response model) we can combine 
the last two constraints. This is summarized in Table~\ref{tab:mods}.

\begin{table}[ht!]
  \centering
  \caption{This table identifies the different constraints for each model type.}
  \label{tab:mods}
  \begin{tabular}{lp{12cm}}
    \toprule
    Type & Constraints \\
    \midrule
    Parametric 1 & $\sum_{k, t} c_{kt} = 1$ \\
    Parametric 2 &  $\sum_{k, t: (k, t) \neq (4, 4)} c_{kt} = 0, c_{44} = 1$ \\
    Outcome Robust & $c_{11} + c_{21} + c_{31} + c_{41} = 0, c_{22} + c_{42} =
    0, c_{33} + c_{43} = 0, \text{ and } c_{44} = 1.$ \\
    Response Robust & $c_{11} = \pi_{00}, c_{21} + c_{22} = \pi_{10}, c_{31} +
    c_{33} = \pi_{01}, \text{ and } c_{41} + c_{42} + c_{43} + c_{44} =
    \pi_{11}$ \\
    Double Robust & $c_{11} + c_{21} + c_{31} + c_{41} = 0, c_{22} + c_{42} =
    0, c_{33} + c_{43} = 0, c_{44} = 1,$ \hspace{3cm} $
    c_{11} = \pi_{00}, c_{21} + c_{22} = \pi_{10}, c_{31} +
    c_{33} = \pi_{01}, \text{ and } c_{41} + c_{42} + c_{43} + c_{44} = \pi_{11}$ \\
    \bottomrule
  \end{tabular}
\end{table}

\subsection*{Simulation Study}

\subsubsection*{Simulation 1}

We use the following simulation setup

\begin{align*}
  \begin{bmatrix} x \\ e_1 \\ e_2 \end{bmatrix} 
  &\stackrel{ind}{\sim} N\left(\begin{bmatrix} 0 \\ 0 \\ 0 \end{bmatrix}, 
  \begin{bmatrix} 1 & 0 & 0 \\ 0 & 1 & \rho \\ 0 & \rho & 1 \end{bmatrix}\right) \\
  y_1 &= x + e_1 \\
  y_2 &= \theta + x + e_2 \\
\end{align*}

Furthermore, $\pi_{11} = 0.2$ and $\pi_{00} = 0.3, \pi_{10} = 0.4$, and 
$\pi_{01} = 0.1$.
The goal of this simulation study is to find $\theta = E[Y_2]$. In other words,
$g(Z) = Y_2$. There are several algorithms for comparison which are defined 
as the following:

\begin{align*}
  Oracle &= n^{-1} \sum_{i = 1}^n g(Z_i)\\
  CC &= \frac{\sum_{i = 1}^n \delta_{11} g(Z_i)}{\sum_{i = 1}^n \delta_{11}} \\
  IPW &= \sum_{i = 1}^n \frac{\delta_{11}}{\pi_{11}} g(Z_i)\\
\end{align*}

Furthermore, we include four existing estimators that we have already 
proposed in the past: WLS, Prop, PropInd, and SemiDelta. 

The estimator WLS is a a weight least square estimator derived in the 
following manner. We now consider a normal model:

    \[ 
      \begin{pmatrix}
        x_i \\ e_{1i} \\ e_{2i}
      \end{pmatrix} \stackrel{ind}{\sim}
      N \left(
      \begin{bmatrix}
        0 \\ 0 \\ 0
      \end{bmatrix},
      \begin{bmatrix}
        1 & 0 & 0 \\
        0 & \sigma_{11} & \sigma_{12} \\ 
        0 & \sigma_{12} & \sigma_{22}
      \end{bmatrix},
      \right)
    \]

    and define $y_{1i} = \theta_1 + x_i + e_{1i}$ and $y_{2i} = \theta_2 + x_i
    + e_{2i}$. Then $b_1 = b_2 = 1$. We define $\bar z_k^{(ij)}$ as the mean of
    $y_k$ in segment $A_{ij}$. This means that we have means $\bar z_1^{(11)}$,
    $\bar z_2^{(11)}$, $\bar z_1^{(10)}$, and $\bar z_2^{(01)}$. Let $W = [\bar
    z_1^{(11)}, \bar z_2^{(11)}, \bar z_1^{(10)}, \bar z_2^{(01)}]'$, then for
    $n_{ij} = |A_{ij}|$, we have

    \[Z - M \mu \sim N(\vec 0, V)\]

    where 

    \[M = 
      \begin{bmatrix}
        1 & 0 \\
        0 & 1 \\
        1 & 0 \\
        0 & 1 \\
      \end{bmatrix}
      \text{ and }
      V = 
      \begin{bmatrix}
        \frac{\sigma_{11}}{n_{11}} & \frac{\sigma_{12}}{n_{11}} & 0 & 0 \\
        \frac{\sigma_{12}}{n_{11}} & \frac{\sigma_{22}}{n_{11}} & 0 & 0 \\
        0 & 0 & \frac{\sigma_{11}}{n_{10}} & 0 \\
        0 & 0 & 0 & \frac{\sigma_{22}}{n_{01}} \\
      \end{bmatrix}.
    \]

    Thus, the BLUE for $\mu = [\mu_1, \mu_2]'$ is 

    \[\hat \mu = (M' V^{-1} M)^{-1} M' V^{-1} W.\]

    Hence, WLS is $\mu_2$ as $g(X, Y_1, Y_2) = Y_2$.


The remaining three estimators are drived from the following expression
where the values for $\beta$ are provided in Table~\ref{tab:beta}. These 
are good estimators to compare to because Prop is the original proposed 
estimator. PropInd has the same form as Prop except the values for $\beta$ 
is different. PropInd shares the same values of $\beta$ as all of the new 
models with constraints. SemiDelta is useful because it is the best 
estimator in general so far.

\[\hat \theta = \frac{\delta_{11}}{\pi_{11}}g(Z) + \beta_0(\delta, c_0)E[g(Z)
\mid X] + \beta_1(\delta, c_1)E[g(Z) \mid X, Y_1] + \beta_2(\delta, c_2) E[g(Z)
\mid X, Y_2].\]

\begin{table}[ht!]
  \centering
  \caption{This table displays the values of $\beta$ for different estimator 
  types.}
  \label{tab:beta}
\begin{tabular}{lrr}
  \toprule
  Estimator & $\beta_0(\delta, c_0)$ & $\beta_1(\delta, c_1)$ \\
  \midrule
  Prop & $\left(1 - \frac{(\delta_{10} + \delta_{11})}{(\pi_{10} + \pi_{11})} - 
  \frac{(\delta_{01} + \delta_{11})}{(\pi_{01} + \pi_{11})} + \frac{\delta_{11}}{\pi_{11}}\right)$ &
  $\left(\frac{\delta_{10} + \delta_{11}}{\pi_{10} + \pi_{11}} - \frac{\delta_{11}}{\pi_{11}}\right)$ \\
  PropInd & $\left(1 - \frac{(\delta_{10})}{(\pi_{10})} - 
  \frac{(\delta_{01})}{(\pi_{01})} + \frac{\delta_{11}}{\pi_{11}}\right)$ &
  $\left(\frac{\delta_{10}}{\pi_{10}} - \frac{\delta_{11}}{\pi_{11}}\right)$ \\
%  $\hat \theta_{c}$ & $c_0\left(1 - \frac{(\delta_{10} + \delta_{11})}{(\pi_{10} + \pi_{11})} - 
%  \frac{(\delta_{01} + \delta_{11})}{(\pi_{01} + \pi_{11})} + \frac{\delta_{11}}{\pi_{11}}\right)$ &
%  $c_1\left(\frac{\delta_{10} + \delta_{11}}{\pi_{10} + \pi_{11}} - \frac{\delta_{11}}{\pi_{11}}\right)$ \\
%  $\hat \theta_{c, ind}$ & $c_0\left(1 - \frac{(\delta_{10})}{(\pi_{10})} - 
%  \frac{(\delta_{01})}{(\pi_{01})} + \frac{\delta_{11}}{\pi_{11}}\right)$ &
%  $c_1\left(\frac{\delta_{10}}{\pi_{10}} - \frac{\delta_{11}}{\pi_{11}}\right)$ \\
  SemiDelta & $c_0\left(\frac{\delta_{11}}{\pi_{11}} - \frac{\delta_{00}}{\pi_{00}}\right)$ &
  $c_1\left(\frac{\delta_{11}}{\pi_{11}} - \frac{\delta_{10}}{\pi_{10}}\right)$ \\
  \bottomrule
\end{tabular}
\end{table}


In the simulation results in Table~\ref{tab:sim1}, the new results have the 
same label as the value from the ``Type'' column in Table~\ref{tab:mods}.


\begin{table}[ht!]
  \caption{Results from Simulation 1. True values: $\theta = 5, \rho = 0.5$.
  The test conducted for the T-statistic and P-value is a two sample test to 
  see if the estimator is unbiased.}
  \label{tab:sim1}
  \centering
  \begin{tabular}[t]{lrrrr}
    \toprule
    Algorithm & Bias & SD & Tstat & P-value\\
    \midrule
    Oracle & -0.002 & 0.045 & -1.669 & 0.048\\
    CC & 0.001 & 0.083 & 0.263 & 0.396\\
    IPW & 0.010 & 0.357 & 0.872 & 0.192\\
    WLS & 0.000 & 0.054 & -0.065 & 0.474\\
    %WLS (with true $\theta$) & -0.001 & 0.035 & -1.158 & 0.124\\
    Prop & -0.002 & 0.063 & -0.768 & 0.221\\
    PropInd & -0.003 & 0.120 & -0.900 & 0.184\\
    SemiDelta & -0.001 & 0.064 & -0.342 & 0.366\\
    Parametric 1 & -0.001 & 0.061 & -0.509 & 0.305\\
    Parametric 2 & -0.001 & 0.061 & -0.509 & 0.305\\
    Outcome Robust & -0.001 & 0.061 & -0.543 & 0.294\\
    Response Robust & -0.001 & 0.061 & -0.509 & 0.305\\
    Double Robust & -0.001 & 0.061 & -0.543 & 0.294\\
    \bottomrule
  \end{tabular}
\end{table}

\newpage 

\subsubsection*{Simulation 2}

Next, we used the same simulation setup except that we are interested in 
estimating $\theta = E[g] = E[Y_1^2 Y_2]$. We do not use the WLS estimator 
because it does not work.

\begin{table}[ht!]
  \caption{Results from Simulation 2. True values: $\theta = 10, \rho = 0.5$.
  The test conducted for the T-statistic and P-value is a two sample test to 
  see if the estimator is unbiased.}
  \label{tab:sim2}
  \centering
  \begin{tabular}[t]{lrrrr}
    \toprule
    Algorithm & Bias & SD & Tstat & P-value\\
    \midrule
    Oracle & -0.037 & 0.528 & -2.214 & 0.014\\
    CC & 0.007 & 1.196 & 0.179 & 0.429\\
    IPW & 0.023 & 1.405 & 0.520 & 0.302\\
    Prop & -0.044 & 0.691 & -2.027 & 0.021\\
    SemiDelta & -0.032 & 0.674 & -1.521 & 0.064\\
    Parametric 1 & -0.004 & 0.301 & -0.460 & 0.323\\
    Parametric 2 & -0.004 & 0.344 & -0.341 & 0.367\\
    Outcome Robust & -0.040 & 0.654 & -1.950 & 0.026\\
    Response Robust & -0.004 & 0.301 & -0.461 & 0.322\\
    Double Robust & -0.040 & 0.654 & -1.954 & 0.025\\
    \bottomrule
  \end{tabular}
\end{table}


\newpage

\subsection*{Discussion}

\begin{itemize}
  \item In the first simulation there does not seem to be much of a cost 
    to using a semiparametric estimator with the Double Robust model 
    performing as well as the Outcome Robust, Response Robust and 
    Parametric models. However, in the second simulation the Parametric 
    models and Response Robust estimators perform the best.
  \item Strangly, the Double Robust estimator in the second simulation 
    has picked up the bad characteristics of the Outcome Model. This 
    can make sense if the optimal Outcome Model is contained in the 
    space of the Double Robust model but the optimal Response Model 
    is not.
\end{itemize}

\end{document}
