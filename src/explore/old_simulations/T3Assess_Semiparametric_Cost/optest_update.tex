\documentclass[12pt]{article}

\usepackage{amsmath, amssymb, mathrsfs, fancyhdr}
\usepackage{syntonly, lastpage, hyperref, enumitem, graphicx}
\usepackage{biblatex}
\usepackage{booktabs}
\usepackage{float}

\addbibresource{references.bib}

\hypersetup{colorlinks = true, urlcolor = black}

\headheight     15pt
\topmargin      -1.5cm   % read Lamport p.163
\oddsidemargin  -0.04cm  % read Lamport p.163
\evensidemargin -0.04cm  % same as oddsidemargin but for left-hand pages
\textwidth      16.59cm
\textheight     23.94cm
\parskip         7.2pt   % sets spacing between paragraphs
\parindent         0pt   % sets leading space for paragraphs
\pagestyle{empty}        % Uncomment if don't want page numbers
\pagestyle{fancyplain}

\newcommand{\MAP}{{\text{MAP}}}
\newcommand{\argmax}{{\text{argmax}}}
\newcommand{\argmin}{{\text{argmin}}}
\newcommand{\Cov}{{\text{Cov}}}
\newcommand{\Var}{{\text{Var}}}
\newcommand{\logistic}{{\text{logistic}}}

\renewcommand{\arraystretch}{1.2}

\begin{document}

\lhead{Caleb Leedy}
\chead{Non-monotone Missingness}
%\chead{STAT 615 - Advanced Bayesian Methods}
%\rhead{Page \thepage\ of \pageref{LastPage}}
\rhead{\today}

\section*{Setup}

Consider the following setup. Let $(X, Y_1, Y_2, \delta)  \stackrel{ind}{\sim}
F$ for some distribution $F$ that is unknown. We define $Z = (X, Y_1, Y_2$ and
we observe the following:

\begin{table}[!ht]
  \centering
  \caption{This table shows some of our notation and some of the corresponding
  notation from \cite{tsiatis2006semiparametric}.}
\begin{tabular}{cccccccc}
  \toprule
  Segments & $X$ & $Y_1$ & $Y_2$ & Prob. Element in Segment & $\delta$ & $C$ & $G_C(Z)$ \\
  \midrule
  $A_{00}$ & $\checkmark$ &               &         & $\pi_{00}$ & $\delta_{00}$ & 1 & $\{X\}$ \\
  $A_{10}$ & $\checkmark$ & $\checkmark$  &         & $\pi_{10}$ & $\delta_{10}$ & 2 & $\{X, Y_1\}$ \\
  $A_{01}$ & $\checkmark$ &               & $\checkmark$ & $\pi_{01}$ & $\delta_{01}$ & 3 & $\{X, Y_2\}$ \\
  $A_{11}$ & $\checkmark$ & $\checkmark$  & $\checkmark$ & $\pi_{11}$ & $\delta_{11}$ & $\infty$ & $\{X, Y_1, Y_2\}$ \\
  \bottomrule
\end{tabular}
\end{table}

Define $\varpi(r, Z) = \Pr(C = r \mid Z)$. For now, we assume that $\varpi(r,
Z)$ is known. Notice that $\varpi(\infty, Z) = \pi_11$, $\varpi(3, Z) = \pi_{01}$,
$\varpi(2, Z) = \pi_{10}$, and $\varpi(1, Z) = \pi_{00}$. Since $\pi_{00} +
\pi_{10} + \pi_{01} + \pi_{11} = 1$, we only need to define three inclusion 
probabilities.\\

Suppose that we want to estimate $\theta = E[g(X, Y_1, Y_2)]$ for a known
function $g$. 
The goal is the show that over all functions $b_1(X, Y_1)$ and $b_2(X, Y_2)$,
we have the optimal estimator in the form:

\begin{align}
  &\hat \theta =\\ \nonumber
  &n^{-1} \sum_{i = 1}^n E[g_i \mid X_i] + n^{-1} \sum_{i = 1}^n
  \frac{\delta_{10}}{\pi_{10}} (b_1(X_i, Y_{1i}) - E[g_i \mid X_i]) 
  + n^{-1} \sum_{i = 1}^n \frac{\delta_{01}}{\pi_{01}} (b_2(X_i, Y_{2i}) - E[g_i \mid X_i]) \\ \nonumber
  &+ n^{-1} \sum_{i = 1}^n \frac{\delta_{11}}{\pi_{11}} (g_i - b_1(X_i, Y_{1i}) - b_2(X_i, Y_{2i}) + E[g_i \mid X_i])
\end{align}


%The proposed estimator is 
%
%\begin{align}\label{eq:propind}
%  &\hat \theta^{ind}_{prop} =\\ \nonumber
%  &n^{-1} \sum_{i = 1}^n E[g_i \mid X_i] + n^{-1} \sum_{i = 1}^n
%  \frac{\delta_{10}}{\pi_{10}} (E[g_i \mid X_i, Y_{1i}] - E[g_i \mid X_i]) 
%  + n^{-1} \sum_{i = 1}^n \frac{\delta_{01}}{\pi_{01}} (E[g_i \mid X_i, Y_{2i}] - E[g_i \mid X_i]) \\ \nonumber
%  &+ n^{-1} \sum_{i = 1}^n \frac{\delta_{11}}{\pi_{11}} (g_i - E[g_i \mid X_i, Y_{1i}] - E[g_i \mid X_i, Y_{2i}] + E[g_i \mid X_i]).
%\end{align}

\section*{Semiparametric Inference}

We know from Theorem 7.2 of \cite{tsiatis2006semiparametric} that if $\Pr(C =
\infty \mid Z) = \pi_{11} > 0$ 
then the semiparametric influence function has the form (see page 20 of notes):

\begin{equation}\label{lambda2}
 \frac{I(C = \infty) g(Z)}{\varpi(\infty, Z)} + \frac{I(C =
  \infty)}{\varpi(\infty, Z)} \left(\sum_{r \neq \infty} \varpi(r, G_r(Z))
  L_{2r}(G_r(Z))\right) - \sum_{r \neq \infty} I(C = r)L_{2r}(G_r(Z))
\end{equation}

where $L_{2r}(G_r(Z))$ is an arbitrary function of $G_r(Z)$. Notice, that this
form does not identify \textit{the} optimal estimator but a class of 
semiparametric functions. A reasonable choice of an estimator for $L_{2r}(G_r(Z))$ is

\[L_{2r}(G_r(Z)) = c_r E[g(Z) \mid G_r(Z)]\]

where each $c_r$ is a constant. This actually suggests a variety of estimators 
that each have different values of coefficients of $\delta$. Initially, we tried 
a direct projection onto the nuisance tangent space, but I got stuck. See Appendix A 
for this work.

\subsection*{Different Classes of Estimators}

All of the considered estimators have a similar form:

\[\hat \theta = \frac{\delta_{11}}{\pi_{11}}g(Z) + \beta_0(\delta, c_0)E[g(Z)
\mid X] + \beta_1(\delta, c_1)E[g(Z) \mid X, Y_1] + \beta_2(\delta, c_2) E[g(Z)
\mid X, Y_2].\]

In this case, $c$ is a vector of constants $c = (c_0, c_1, c_2)$ that are 
chosen to minimize the variance of $\hat \theta$. See Appendix B for details on 
an example about choosing the values of $\hat c$.

\begin{tabular}{lrrr}
  \toprule
  Estimator & $\beta_0(\delta, c_0)$ & $\beta_1(\delta, c_1)$ & Implemented \\
  \midrule
  $\hat \theta_{prop}$ & $\left(1 - \frac{(\delta_{10} + \delta_{11})}{(\pi_{10} + \pi_{11})} - 
  \frac{(\delta_{01} + \delta_{11})}{(\pi_{01} + \pi_{11})} + \frac{\delta_{11}}{\pi_{11}}\right)$ &
  $\left(\frac{\delta_{10} + \delta_{11}}{\pi_{10} + \pi_{11}} - \frac{\delta_{11}}{\pi_{11}}\right)$ & \checkmark \\
  $\hat \theta^{ind}_{prop}$ & $\left(1 - \frac{(\delta_{10})}{(\pi_{10})} - 
  \frac{(\delta_{01})}{(\pi_{01})} + \frac{\delta_{11}}{\pi_{11}}\right)$ &
  $\left(\frac{\delta_{10}}{\pi_{10}} - \frac{\delta_{11}}{\pi_{11}}\right)$ & \checkmark \\
  $\hat \theta_{c}$ & $c_0\left(1 - \frac{(\delta_{10} + \delta_{11})}{(\pi_{10} + \pi_{11})} - 
  \frac{(\delta_{01} + \delta_{11})}{(\pi_{01} + \pi_{11})} + \frac{\delta_{11}}{\pi_{11}}\right)$ &
  $c_1\left(\frac{\delta_{10} + \delta_{11}}{\pi_{10} + \pi_{11}} - \frac{\delta_{11}}{\pi_{11}}\right)$ & \checkmark \\
  $\hat \theta_{c, ind}$ & $c_0\left(1 - \frac{(\delta_{10})}{(\pi_{10})} - 
  \frac{(\delta_{01})}{(\pi_{01})} + \frac{\delta_{11}}{\pi_{11}}\right)$ &
  $c_1\left(\frac{\delta_{10}}{\pi_{10}} - \frac{\delta_{11}}{\pi_{11}}\right)$ & \checkmark \\
  $\hat \theta_{\delta}$ & $c_0\left(\frac{\delta_{11}}{\pi_{11}} - \frac{\delta_{00}}{\pi_{00}}\right)$ &
  $c_1\left(\frac{\delta_{11}}{\pi_{11}} - \frac{\delta_{10}}{\pi_{10}}\right)$ & \checkmark \\
  \bottomrule
\end{tabular}

I wanted to give a brief comment on the origin of each of these estimators. The 
first estimator $\hat \theta_{prop}$ is the estimator proposed by Dr. Kim. The 
original goal was to show that this is optimal. The second estimator, 
$\hat \theta^{ind}_{prop}$ is the first estimator with different functions of 
$\delta$ to try to make each component more independent from each other. The 
problem with this is that the components are still dependent and it has worse 
properties in simulation studies (next section). The estimator $\hat \theta_c$ 
is the originial estimator $\hat \theta_{prop}$ with different coefficients that
hopefully make it slightly more efficient. The estimator $\hat \theta_{c, ind}$ 
does the same thing with $\hat \theta_{prop}^{ind}$. Finally, 
$\hat \theta_{\delta}$ uses the functional form of Theorem 7.2 from 
\cite{tsiatis2006semiparametric} and optimized coefficients.

%It turns out that $\hat \theta_{prop}$ from Equation~\ref{eq:prop} is contained within 
%the previous class of estimators with 
%
%\[L_{2r}(G_r(Z)) = \varpi(r, G_r(Z)) E[g(Z) \mid G_r(Z)].\]
%
%(See pages 81-82 of notes for proof.) \\

\section*{Simulation Studies}

\subsection*{Simulation 1}

To see which of these estimators is the best, we run a simulation study. In this 
study, we have the following variables:

\begin{align*}
  \begin{bmatrix} x \\ e_1 \\ e_2 \end{bmatrix} 
  &\stackrel{ind}{\sim} N\left(\begin{bmatrix} 0 \\ 0 \\ 0 \end{bmatrix}, 
  \begin{bmatrix} 1 & 0 & 0 \\ 0 & 1 & \rho \\ 0 & \rho & 1 \end{bmatrix}\right) \\
  y_1 &= x + e_1 \\
  y_2 &= \theta + x + e_2 \\
\end{align*}

Furthermore, $\pi_{11} = 0.4$ and $\pi_{00} = \pi_{10} = \pi_{01} = 0.2$.
The goal of this simulation study is to find $\theta = E[Y_2]$. In other words,
$g(Z) = Y_2$. There are several algorithms for comparison which are defined 
as the following:

\begin{align*}
  Oracle &= n^{-1} \sum_{i = 1}^n g(Z_i)\\
  CC &= \frac{\sum_{i = 1}^n \delta_{11} g(Z_i)}{\sum_{i = 1}^n \delta_{11}} \\
  IPW &= \sum_{i = 1}^n \frac{\delta_{11}}{\pi_{11}} g(Z_i)\\
\end{align*}


\begin{table}[ht!]
\caption{The true value of $\theta$ is 5. $\rho = 0.5$. The test statistic and 
p-value are from a one-sample t-test to see if the estimator is biased.}
\centering
\begin{tabular}[t]{lrrrr}
\toprule
Algorithm & Bias & SD & Tstat & Pval\\
\midrule
Oracle & 0.001 & 0.044 & 1.546 & 0.061\\
CC & 0.002 & 0.058 & 1.549 & 0.061\\
IPW & 0.006 & 0.210 & 1.504 & 0.066\\
$\hat \theta_{prop}$ & 0.002 & 0.051 & 1.911 & 0.028\\
$\hat \theta_{c}$ & 0.002 & 0.051 & 1.920 & 0.027\\
$\hat \theta_{prop}^{ind}$ & 0.002 & 0.091 & 0.983 & 0.163\\
$\hat \theta_{c, ind}$ & 0.001 & 0.074 & 0.784 & 0.216\\
$\hat \theta_{\delta}$ & 0.002 & 0.051 & 1.964 & 0.025\\
\bottomrule
\end{tabular}
\end{table}

The main takeaway from this simulation study is that $\hat \theta_\delta$,
$\hat \theta_c$
and $\hat\theta_{prop}$ have the smallest standard deviation among the 
non-oracle estimators. In fact $\hat \theta_c$ and $\hat \theta_{prop}$ are 
virtually identical because the optimal values of $c$ in $\hat \theta_c$ 
are $(c_0, c_1, c_2)' \approx (1, 1, 1)'$.

\subsection*{Simulation 2}

We now test a second simulation study. The setup of this simulation study 
is the same as the previous simulation. The only difference is that we 
are interested in estimating $\theta = E[Y_1^2 Y_2]$. In other words, now 
we have $g(Z) = Y_1^2 Y_2$.

\begin{table}[ht!]
  \caption{The true value of $E[g(Z)]$ is 10. $\rho = 0.5$. The test statistic and 
p-value are from a one-sample t-test to see if the estimator is biased.}
\centering
\begin{tabular}[t]{lrrrr}
\toprule
Algorithm & Bias & SD & Tstat & Pval\\
\midrule
Oracle & 0.007 & 0.529 & 0.741 & 0.229\\
CC & 0.023 & 0.824 & 1.518 & 0.065\\
IPW & 0.031 & 0.915 & 1.846 & 0.032\\
$\hat \theta_{prop}$ & 0.011 & 0.635 & 0.912 & 0.181\\
$\hat \theta_{c}$ & 0.012 & 0.632 & 1.010 & 0.156\\
$\hat \theta_{c, ind}$ & 0.013 & 0.662 & 1.060 & 0.145\\
$\hat \theta_{\delta}$ & 0.011 & 0.636 & 0.983 & 0.163\\
\bottomrule
\end{tabular}
\end{table}

Like the first simulation, $\hat \theta_{prop}$, $\hat \theta_c$
and $\hat \theta_{\delta}$ performed the best.

\section*{Next Steps}

\begin{itemize}
  \item[1.] It turns out that \cite{tsiatis2006semiparametric} contains 
    a general technique for solving non-monotone missing data. They show 
    that the optimal estimator is found via the following double projection:

    \[L_2 = \Pi\left(\Pi\left(\frac{\delta_{11}}{\pi_{11}}g(Z) \mid C,
    G_C(Z)\right) \mid Z\right)\]

    I can show that $\hat \theta_\delta$ is NOT such a double projection. 
    However, I think it is worth exploring what such an estimator (especially 
    if we consider a linear estimator) looks like.
  \item[2.] Similar to the previous point, \cite{tsiatis2006semiparametric} 
    discuss what the optimal estimator of a non-monotone problem looks like 
    and they determine that it solves the estimating equation

    \[\sum_{i = 1}^n \mathcal{L}(\mathcal{M}^{-1}(g(Z)))\]

    where $\mathcal{M}(g(Z)) = E[\mathcal{L}(g(Z)) \mid Z]$ and 
    $\mathcal{L}(g(Z)) = E[g(Z) \mid C, G_C(Z)]$. The difficult part is 
    understanding $\mathcal{M}^{-1}$, but there is conveniently the fact that

    \[\mathcal{M}^{-1}(g(Z)) = \lim_{n \to \infty} \phi_{n + 1}(Z) \text{ where } 
      \phi_{n + 1}(Z) = (I - \mathcal{M}) \phi_n(Z) + g(Z) \text{ and }
    \phi_0(Z) = g(Z).\]

    I have been investigating what this estimating equation yields if we 
    approximate $\mathcal{M}^{-1}$ with $\phi_1$. However, I have not 
    finished this yet.
  \item[3.] A different method to find the projection onto $\Lambda_2$ 
    could be to try to minimize the (KL) divergence between 
    $L_2$ and $\frac{\delta_{11}}{\pi_{11}}g(Z)$.
\end{itemize}

\newpage 

\printbibliography

\newpage

\appendix

\section{Linear Expectations}

To simplify this problem, we consider the following estimator\footnote{This
estimator has slightly different coefficients compared to the initial
estimator.}:

\begin{align}
  &\hat \theta_c \\ \nonumber
  &= \frac{\delta_{11}}{\pi_{11}} g(Z) + \left(1 -
    \left(\frac{\delta_{10} + \delta_{11}}{\pi_{10} + \pi_{11}}\right) -
    \left(\frac{\delta_{01} + \delta_{11}}{\pi_{01} + \pi_{11}}\right) + 
  \frac{\delta_{11}}{\pi_{11}}\right) c_0 E[g \mid X] \\ \nonumber
  &+
  \left(\left(\frac{\delta_{10} + \delta_{11}}{\pi_{10} + \pi_{11}}\right) - 
  \frac{\delta_{11}}{\pi_{11}}\right) c_1 E[g \mid X, Y_1] +
  \left(\left(\frac{\delta_{01} + \delta_{11}}{\pi_{01} + \pi_{11}}\right) - 
  \frac{\delta_{11}}{\pi_{11}}\right) c_2 E[g \mid X, Y_2]
\end{align}

\subsection{Projection onto Nuisance Tangent Space}

The goal is now to find the coefficients $c_0, c_1$, and $c_2$ such that 
$\langle \hat \theta_c, L_2\rangle \equiv E[\hat \theta_c L_2] = 0$ for 
all $L_2 \in \Lambda_2$ (see \cite{tsiatis2006semiparametric} for definition 
of $\Lambda_2$). If we can find such coefficients that the estimator $\hat
\theta_c$ will be orthogonal to $\Lambda_2$ and hence by Theorem 10.1 of 
\cite{tsiatis2006semiparametric} semiparametrically optimal. The good news 
is that we know (from Theorem 7.2) that any element $L_2 \in \Lambda_2$ has a
form:

\begin{align}
  L_2 &= 
  \left(\frac{\delta_{11}}{\pi_{11}}\pi_{00} - \delta_{00}\right) L_{20}(X) +
  \left(\frac{\delta_{11}}{\pi_{11}}\pi_{10} - \delta_{10}\right) L_{21}(X, Y_1) +
  \left(\frac{\delta_{11}}{\pi_{11}}\pi_{01} - \delta_{01}\right) L_{22}(X, Y_2).
\end{align}

Then expanding and solving $E[\hat \theta_c L_2] = 0$ yields:

{\tiny
\begin{align*}
  0 &= E[\hat \theta_c L_2] \\ 
    &E\left[\left(\frac{\pi_{00}}{\pi_{11}} + \left(\frac{\pi_{10}}{\pi_{10} + \pi_{11}}\right)\frac{\pi_{00}c_1}{\pi_{11}} +
      \left(\frac{\pi_{01}}{\pi_{01} + \pi_{11}}\right) \frac{\pi_{00}c_2}{\pi_{11}} + \frac{\pi_{00}(\pi_{10}\pi_{01} -
  \pi_{11}^2)c_0}{(\pi_{10} + \pi_{11})(\pi_{01} + \pi_{11})\pi_{11}}\right) E[g \mid X] L_{20}(X)\right. \\ 
    &+ \frac{\pi_{10}}{\pi_{11}} \left(E[g(Z) L_{21}(X, Y_1) \mid X] - c_1 E[E[g(Z) \mid X, Y_1] L_{21}(X, Y_1) \mid X] +
      \frac{\pi_{01}c_2}{\pi_{10} + \pi_{11}} E[E[g \mid X, Y_2] L_{21}(X, Y_1) \mid X] +
      \frac{\pi_{10} \pi_{01}}{\pi_{11}(\pi_{01} + \pi_{11})} E[g \mid X]
      E[L_{21}(X, Y_1) \mid X]c_0\right)\\
    &+ \left.\frac{\pi_{01}}{\pi_{11}} \left(E[g(Z) L_{22}(X, Y_2) \mid X] + \frac{\pi_{10} c_1}{\pi_{10} + \pi_{11}} E[E[g(Z) \mid X, Y_1] L_{22}(X, Y_2) \mid X] -
      c_2 E[E[g \mid X, Y_2] L_{22}(X, Y_2) \mid X] +
      \frac{\pi_{10}}{(\pi_{01} + \pi_{11})} E[g \mid X]
      E[L_{22}(X, Y_2) \mid X]c_0\right)\right]
\end{align*}
}

To solve for $c_0, c_1$, and $c_2$ we need the following to hold for any
$L_{21}(X, Y_1)$ and $L_{22}(X, Y_2)$:

{\small
\begin{align*}
  &1 + c_0 \frac{\pi_{01} \pi_{10} - \pi_{11}^2}{(\pi_{10} + \pi_{11})(\pi_{01}
  + \pi_{11})} + c_1 \frac{\pi_{10}}{\pi_{01} + \pi_{11}} + c_2
  \frac{\pi_{01}}{\pi_{01} + \pi_{11}} = 0\\ 
  &E\left[\left(g(Z) + c_0 \frac{\pi_{01}}{\pi_{01} + \pi_{11}}E[g(Z) \mid X] - c_1 E[g(Z) \mid X, Y_1] + c_2 \frac{\pi_{01}}{\pi_{10} + \pi_{11}}E[g(Z) \mid X, Y_2]\right)L_{21}(X, Y_1) \mid X\right] = 0 \\ 
  &E\left[\left(g(Z) + c_0 \frac{\pi_{10}}{\pi_{10} + \pi_{11}}E[g(Z) \mid X] + c_1 \frac{\pi_{10}}{\pi_{10} + \pi_{11}} E[g(Z) \mid X, Y_1] - c_2 E[g(Z) \mid X, Y_2]\right)L_{22}(X, Y_2) \mid X\right] = 0
\end{align*}
}

Unfortunately, I am now stuck because it is unclear to me how to use any actual 
data to solve this problem.

\section{Solving for $\hat c$}

We can find the values of 
$c_0$, $c_1$, and $c_2$ in $\hat \theta_c$ that minimize the variance the 
estimator. We can find these values by differentiating by $c_i$ and solving 
for $c_i$:

\begin{align*}
  \begin{bmatrix} \hat c_0 \\ \hat c_1 \\ \hat c_2 \end{bmatrix} 
  &= -
  \begin{bmatrix} 
    M_{11} & M_{12} & M_{13} \\
    M_{21} & M_{22} & M_{23} \\
    M_{31} & M_{32} & M_{33} \\
  \end{bmatrix}^{-1}
  \times 
  \begin{bmatrix}
    E[E[g \mid X]^2] \left(1 + \frac{\pi_{10}\pi_{01} - \pi_{11}^2}{\pi_{11}(\pi_{10} + \pi_{11})(\pi_{01} + \pi_{11})}\right) \\
    E[E[g \mid X, Y_1]^2] \left(\frac{-\pi_{10}}{\pi_{11}(\pi_{10} + \pi_{11})}\right) \\
    E[E[g \mid X, Y_2]^2] \left(\frac{-\pi_{01}}{\pi_{11}(\pi_{01} + \pi_{11})}\right) \\
  \end{bmatrix}
\end{align*}

where 

\begin{align*}
  M_{11} &= E[E[g \mid X]^2] \left(\frac{\pi_{11}^2 + \pi_{10}\pi_{01}}{\pi_{11}(\pi_{10} + \pi_{11})(\pi_{01} + \pi_{11})} - 1\right) \\
  M_{12} &= E[E[g \mid X]^2] \left(\frac{-\pi_{10}\pi_{01}}{\pi_{11}(\pi_{10} + \pi_{11})(\pi_{01} + \pi_{11})}\right) \\
  M_{13} &= E[E[g \mid X]^2] \left(\frac{-\pi_{10}\pi_{01}}{\pi_{11}(\pi_{10} + \pi_{11})(\pi_{01} + \pi_{11})}\right) \\
  M_{22} &= E[V(E[g \mid X, Y_1] \mid X)] \left(\frac{\pi_{10}}{\pi_{11}(\pi_{10} + \pi_{11})}\right) \\
  M_{23} &= E[E[g \mid X, Y_1]E[g \mid X, Y_2]] \left(\frac{\pi_{10}\pi_{01}}{\pi_{11}(\pi_{10} + \pi_{11})(\pi_{01} + \pi_{11})}\right) \\
  M_{33} &= E[V(E[g \mid X, Y_2] \mid X)] \left(\frac{\pi_{01}}{\pi_{11}(\pi_{01} + \pi_{11})}\right)
\end{align*}

This estimator is similar to several other estimators:

\begin{align}\label{thetaind}
  \hat \theta^{ind}_c 
  &= \frac{\delta_{11}}{\pi_{11}} g(Z) + \left(1 -
    \left(\frac{\delta_{10}}{\pi_{10}}\right) -
    \left(\frac{\delta_{01}}{\pi_{01}}\right) + 
  \frac{\delta_{11}}{\pi_{11}}\right) c_0 E[g \mid X] \\ \nonumber
  &+
  \left(\frac{\delta_{10}}{\pi_{10}} - 
  \frac{\delta_{11}}{\pi_{11}}\right) c_1 E[g \mid X, Y_1] +
  \left(\frac{\delta_{01}}{\pi_{01}} - 
  \frac{\delta_{11}}{\pi_{11}}\right) c_2 E[g \mid X, Y_2]
\end{align}

\begin{align}\label{thetadelta}
  \hat \theta^{\delta}_c 
  &= \frac{\delta_{11}}{\pi_{11}} g(Z) 
  + \left(\frac{\delta_{11}}{\pi_{11}} - \frac{\delta_{00}}{\pi_{00}}\right) 
   c_0 E[g \mid X] \\ \nonumber
  &+
  \left(\frac{\delta_{11}}{\pi_{11}} - 
  \frac{\delta_{10}}{\pi_{10}}\right) c_1 E[g \mid X, Y_1] +
  \left(\frac{\delta_{11}}{\pi_{11}} - 
  \frac{\delta_{01}}{\pi_{01}}\right) c_2 E[g \mid X, Y_2]
\end{align}

The first expression (Equation~\ref{thetaind}) is the proposed estimator 
with independent differences in each segment, while the second expression 
(Equation~\ref{thetadelta}) is the optimal estimator with values of $\delta$
such that $\hat \theta^{\delta}_c \in \Lambda_2$, which means that it has 
the form of the semiparametric in Equation~\ref{lambda2}.

\end{document}
